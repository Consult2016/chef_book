\section{Knife}
\label{sec:server-knife}

After installation Chef Server with default settings, Chef will generate pem keys, which will be used for knife and Chef clients for authentication with server. We should copy its from our Chef Server to <<.chef>> directory in project:

\begin{lstlisting}[language=Bash,label=lst:my-server-cloud-knife1]
$ vagrant ssh chef_server
Welcome to Ubuntu 12.04.1 LTS (GNU/Linux 3.2.0-23-generic x86_64)

vagrant@precise64:~$ sudo cp /etc/chef-server/*.pem /vagrant/.chef/
\end{lstlisting}

On real (production) Chef Server you can use \lstinline!scp! command.

Next we should create for knife configuration file. As you remember from chapter <<\ref{sec:solo-chef-folder}~\nameref{sec:solo-chef-folder}>>, we already have <<.chef>> folder with \lstinline!knife.rb! config. But for Chef server we should define additional params. We can use for this \lstinline!configure! command of knife:

\begin{lstlisting}[language=Bash,label=lst:my-server-cloud-knife2]
$ knife configure -i
Overwrite .../my-server-cloud/.chef/knife.rb? (Y/N) y
Please enter the chef server URL: [https://macbookproleo:443] https://10.33.33.33
Please enter a name for the new user: [leo]
Please enter the existing admin name: [admin]
Please enter the location of the existing admin's private key: [/etc/chef-server/admin.pem] .chef/admin.pem
Please enter the validation clientname: [chef-validator]
Please enter the location of the validation key: [/etc/chef-server/chef-validator.pem] .chef/chef-validator.pem
Please enter the path to a chef repository (or leave blank):
Creating initial API user...
Please enter a password for the new user:
Created user[leo]
Configuration file written to .../my-server-cloud/.chef/knife.rb
\end{lstlisting}

Now our file look like this:

\begin{lstlisting}[label=lst:my-server-cloud-knife3,title=my-server-cloud/.chef/knife.rb]
log_level                :info
log_location             STDOUT
node_name                'leo'
client_key               '.chef/leo.pem'
validation_client_name   'chef-validator'
validation_key           '.chef/chef-validator.pem'
chef_server_url          'https://10.33.33.33'
syntax_check_cache_path  '.chef/syntax_check_cache'
\end{lstlisting}

Let's little modify it:

\begin{lstlisting}[label=lst:my-server-cloud-knife4,title=my-server-cloud/.chef/knife.rb]
current_dir = File.dirname(__FILE__)

log_level                :info
log_location             STDOUT
node_name                'leo'
client_key               "#{current_dir}/leo.pem"
syntax_check_cache_path  "#{current_dir}/syntax_check_cache"
validation_client_name   "chef-validator"
validation_key           "#{current_dir}/chef-validator.pem"
chef_server_url          "https://10.33.33.33"
cookbook_path            ["#{current_dir}/../cookbooks", "#{current_dir}/../site-cookbooks"]
node_path                 "#{current_dir}/../nodes"
role_path                 "#{current_dir}/../roles"
data_bag_path             "#{current_dir}/../data_bags"
environment_path          "#{current_dir}/../environments"
#encrypted_data_bag_secret "data_bag_key"

knife[:berkshelf_path] = "cookbooks"
\end{lstlisting}

Let's consider an options (part of this options already considered in chapter <<\ref{sec:solo-chef-folder}~~\nameref{sec:solo-chef-folder}>>):

\begin{itemize}
  \item \lstinline!chef_server_url! - the URL for the Chef Server
  \item \lstinline!node_name! - the name of the node. This is typically also the same name as the computer from which Knife is run
  \item \lstinline!client_key! - the location of the file which contains the client key. This key used to authenticate with Chef Server
  \item \lstinline!validation_key! - the location of the file which contains the key used when a chef-client is registered with a server. A validation key is signed using the \lstinline!validation_client_name! for authentication. Default value is \lstinline!/etc/chef/validation.pem!
  \item \lstinline!validation_client_name! - the name of the server that–along with the \lstinline!validation_key! – is used to determine whether a chef-client may register with a server. The \lstinline!validation_client_name! located in the server and client configuration files must match
  \item \lstinline!syntax_check_cache_path! - all files in a cookbook must contain valid Ruby syntax. Use this setting to specify the location in which Knife caches information about files that have been checked for valid Ruby syntax
  \item \lstinline!log_level! - log level for knife
  \item \lstinline!log_location! - log location for knife
\end{itemize}

Now we check what knife can communicate with Chef server:

\begin{lstlisting}[language=Bash,label=lst:my-server-cloud-knife5]
$ knife user list
admin
leo
$ knife client list
chef-validator
chef-webui
\end{lstlisting}

As you can see we successfully get list of users and clients from our Chef Server.
