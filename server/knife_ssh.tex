\section{Knife ssh}
\label{sec:server-knife-ssh}

The knife ssh subcommand is used to invoke SSH commands (in parallel) on a subset of nodes within an organization, based on the results of a search query. We already use it to run chef-client on <<first.example.com>> node. Let's consider an examples.

To find the uptime of all of web servers (all node, which have role <<web>>):

\begin{lstlisting}[language=Bash,label=lst:my-server-cloud-knife-ssh1]
$ knife ssh "role:web" "uptime" -i ../keys/production.pem -x ubuntu
***.com    13:18:28 up 55 days, 14 min,  1 user,  load average: 0.00, 0.01, 0.05
***.com    13:18:28 up 75 days, 23:49,  1 user,  load average: 0.00, 0.01, 0.05
***.com    13:18:28 up 55 days, 13 min,  1 user,  load average: 0.08, 0.03, 0.05
\end{lstlisting}

To run the chef-client on all nodes:

\begin{lstlisting}[language=Bash,label=lst:my-server-cloud-knife-ssh2]
$ knife ssh 'name:*' 'sudo chef-client' -i ../keys/production.pem -x ubuntu
...
\end{lstlisting}

To run the chef-client on all nodes, which name begin from <<second>> string:

\begin{lstlisting}[language=Bash,label=lst:my-server-cloud-knife-ssh3]
$ knife ssh 'name:second*' 'sudo chef-client' -i ../keys/production.pem -x ubuntu
...
\end{lstlisting}

To upgrade all nodes (don't do this on real production nodes):

\begin{lstlisting}[language=Bash,label=lst:my-server-cloud-knife-ssh4]
$ knife ssh 'name:*' 'sudo aptitude upgrade -y' -i ../keys/production.pem -x ubuntu
...
\end{lstlisting}

To get memory information from all nodes in staging environment:

\begin{lstlisting}[language=Bash,label=lst:my-server-cloud-knife-ssh5]
$ knife ssh "chef_environment:staging" "free -m" -i ../keys/production.pem -x ubuntu
***.com              total       used       free     shared    buffers     cached
***.com Mem:          1692       1182        509          0        181        491
***.com -/+ buffers/cache:        509       1183
***.com Swap:          895          6        889
...
\end{lstlisting}

\subsection{Chef-client cookbook}

Sometimes you may want update you servers automaticaly (instead using \inline{knife ssh}). For example, you just updated new cookbooks, roles and nodes and all nodes should automaticaly fetch new cookbooks and execute its, if it updated (and for you not critical update speed). We can use special cookbook \href{http://community.opscode.com/cookbooks/chef-client}{chef-client} fo this. It allow for use bluepill, daemontools, runit or cron to configure your systems to run Chef Client as a service. First of all we need add this cookbook in Berksfile and run \inline{berks install}:

\begin{lstlisting}[label=lst:my-server-cloud-knife-ssh6,title=my-server-cloud/Berksfile]
source "http://api.berkshelf.com"

cookbook 'chef-server'
cookbook 'nginx'
cookbook 'yum', '~> 3.0'
cookbook 'chef-client'
\end{lstlisting}

Next we will create new node:

\begin{lstlisting}[language=JSON,label=lst:my-server-cloud-knife-ssh7,title=my-server-cloud/roles/chef-client.json]
{
  "name": "chef-client",
  "description": "The base role for chef-client",
  "chef_type": "role",
  "json_class": "Chef::Role",
  "default_attributes": {
    "chef_client": {
      "interval": 1800,
      "init_style": "upstart",
      "config": {
        "client_fork": true
      }
    }
  },
  "run_list": [
    "recipe[chef-client]",
    "recipe[chef-client::config]"
  ]
}
\end{lstlisting}

We set by attributes to check Chef each 1800 sec (30 min) and use <<fork>> method for execution of chef-client.

Next add \inline{chef-client} in node \inline{run_list}:

\begin{lstlisting}[language=JSON,label=lst:my-server-cloud-knife-ssh8,title=my-server-cloud/nodes/second.example.com.json]
{
  "name": "second.example.com",
  "json_class": "Chef::Node",
  "chef_type": "node",
  "chef_environment": "development",
  "normal": {
    "fqdn": "10.33.33.35"
  },
  "default": {},
  "override": {},
  "run_list": [
    "role[chef-client]",
    "role[nginx]"
  ]
}
\end{lstlisting}

After upload coobooks, role and nodes on Chef Server, we can check what chef-client will be added to upstart:

\begin{lstlisting}[language=Bash,label=lst:my-server-cloud-knife-ssh9]
$ vagrant provision chef_second_client
[chef_second_client] Running provisioner: chef_client...
Creating folder to hold client key...
Uploading chef client validation key...
...
INFO: service[chef-client] restarted
INFO: Chef Run complete in 9.599237616 seconds
INFO: Running report handlers
INFO: Report handlers complete
\end{lstlisting}

By command \inline{knife status} you can check chef-client status on nodes:

\begin{lstlisting}[language=Bash,label=lst:my-server-cloud-knife-ssh10]
$ knife status
3 minutes ago, first.example.com, precise64, 10.0.2.14, ubuntu 12.04.
2 minutes ago, second.example.com, precise64, 10.0.2.15, ubuntu 12.04.
\end{lstlisting}

Of course, you can filter this list, if you have too many nodes (filter by role, environment, etc.):

\begin{lstlisting}[language=Bash,label=lst:my-server-cloud-knife-ssh11]
$ knife status 'name:second*'
4 minutes ago, second.example.com, precise64, 10.0.2.15, ubuntu 12.04.
$ knife status 'role:nginx'
6 minutes ago, first.example.com, precise64, 10.0.2.14, ubuntu 12.04.
5 minutes ago, second.example.com, precise64, 10.0.2.15, ubuntu 12.04.
\end{lstlisting}
