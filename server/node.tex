\section{Bootstrap first node}

Once the Chef Server workstation is configured, it can be used to install Chef on one (or more) nodes across the organization using a Knife bootstrap operation. The <<knife bootstrap>> command is used to SSH into the target machine, and then do what is needed to allow the chef-client to run on the node. It will install the chef-client executable (if necessary), generate keys, and register the node with the Chef Server. The bootstrap operation requires the IP address or FQDN of the target system, the SSH credentials (username, password or identity file) for an account that has root access to the node, and (if the operating system is not Ubuntu, which is the default distribution used by knife bootstrap) the operating system running on the target system.

First, let's add new server in Vagrantfile:

\begin{lstlisting}[label=lst:my-server-cloud-node1,title=my-server-cloud/Vagrantfile]
...

Vagrant.configure(VAGRANTFILE_API_VERSION) do |config|

  ...

  config.vm.define :chef_client do |chef_client|
    chef_client.vm.box = "precise64"
    chef_client.vm.network "private_network", ip: "10.33.33.34"
  end

end
\end{lstlisting}

And reload vagrant servers:

\begin{lstlisting}[language=Bash,label=lst:my-server-cloud-node2]
$ vagrant halt chef_server
[chef_server] Attempting graceful shutdown of VM...
$ vagrant up
Bringing machine 'chef_server' up with 'virtualbox' provider...
Bringing machine 'chef_client' up with 'virtualbox' provider...
...
\end{lstlisting}

And now we can bootstrap node:

\begin{lstlisting}[language=Bash,label=lst:my-server-cloud-node3]
$ knife bootstrap localhost -x vagrant -p 2200 -i ~/.vagrant.d/insecure_private_key -N first.example.com --sudo
Bootstrapping Chef on localhost
localhost --2014-01-05 16:01:33--  https://www.opscode.com/chef/install.sh
...
localhost Chef Client finished, 0 resources updated
$ knife node list
first.example.com
\end{lstlisting}

If you have error like this:

\begin{lstlisting}[language=Bash,label=lst:my-server-cloud-node4]
localhost Network Error:
localhost --------------
localhost There was a network error connecting to the Chef Server:
localhost Error connecting to https://chef.server-cloud.dev/clients - getaddrinfo: Name or service not known
\end{lstlisting}

It is mean, what node couldn't find your chef server. Just add info hosts on client IP of Chef Server:

\begin{lstlisting}[language=Bash,label=lst:my-server-cloud-node5]
$ vagrant ssh chef_client

vagrant@precise64:~$ cat /etc/hosts
127.0.0.1	localhost
127.0.1.1	precise64

10.33.33.33		chef.server-cloud.dev
vagrant@precise64:~$ ping chef.server-cloud.dev
PING chef.server-cloud.dev (10.33.33.33) 56(84) bytes of data.
64 bytes from chef.server-cloud.dev (10.33.33.33): icmp_req=1 ttl=64 time=0.623 ms
64 bytes from chef.server-cloud.dev (10.33.33.33): icmp_req=2 ttl=64 time=0.466 ms
\end{lstlisting}

And we can check what node created on server:

\begin{lstlisting}[language=Bash,label=lst:my-server-cloud-node6]
$ knife node list
first.example.com
$ knife client show first.example.com
admin:      false
chef_type:  client
json_class: Chef::ApiClient
name:       first.example.com
public_key: -----BEGIN PUBLIC KEY-----
MIIBIjANBgkqhkiG9w0BAQEFAAOCAQ8AMIIBCgKCAQEAw/dmDkNpaVeL7FVB72Hp
bRn6iQgMU0dGrrO8kJUzmQaCTwuqjD09Pa8FeaEcpz8WZ1M28F0/Bp00uuQcXqZr
M/0noxFkTZWLJkw+wQAB4X93mOq2v5vEGvs/tsz7YYfCAztd9ktYwfl3GVRaPgu8
BcsH0tbP4RRCj6pd8+oGpVI+w9LETmtt1Tl5FcNyOgVpZD+PbCmb6HkeDf0qJQss
7PJjqgIrTNx6hm1KWSWNGNAivl1VGYVMXpg0ULdHVCjQE1Fmz7Amy/9WTBDYtCI1
fP5j0ti6WujVMfhgLRTQqRlFkc0SMwDlaaPjc3u4iX4pSCwkds+pM3773sI42ekd
CQIDAQAB
-----END PUBLIC KEY-----

validator:  false
\end{lstlisting}