\section{Bootstrap first node}

Once the Chef Server workstation is configured, it can be used to install Chef on one (or more) nodes across the organization using a Knife bootstrap operation. The <<knife bootstrap>> command is used to SSH into the target machine, and then do what is needed to allow the chef-client to run on the node. It will install the chef-client executable (if necessary), generate keys, and register the node with the Chef Server. The bootstrap operation requires the IP address or FQDN of the target system, the SSH credentials (username, password or identity file) for an account that has root access to the node, and (if the operating system is not Ubuntu, which is the default distribution used by knife bootstrap) the operating system running on the target system.

First, let's add new server in Vagrantfile:

\begin{lstlisting}[label=lst:my-server-cloud-node1,title=my-server-cloud/Vagrantfile]
...

Vagrant.configure(VAGRANTFILE_API_VERSION) do |config|

  ...

  config.vm.define :chef_first_client do |chef_client|
    chef_client.vm.box = "precise64"
    chef_client.vm.network "private_network", ip: "10.33.33.34"
  end

end
\end{lstlisting}

And reload vagrant servers:

\begin{lstlisting}[language=Bash,label=lst:my-server-cloud-node2]
$ vagrant halt chef_server
[chef_server] Attempting graceful shutdown of VM...
$ vagrant up
Bringing machine 'chef_server' up with 'virtualbox' provider...
Bringing machine 'chef_client' up with 'virtualbox' provider...
...
\end{lstlisting}

And now we can bootstrap node:

\begin{lstlisting}[language=Bash,label=lst:my-server-cloud-node3]
$ knife bootstrap localhost -x vagrant -p 2200 -i ~/.vagrant.d/insecure_private_key -N first.example.com --sudo
Bootstrapping Chef on localhost
localhost --2014-01-05 16:01:33--  https://www.opscode.com/chef/install.sh
...
localhost Chef Client finished, 0 resources updated
$ knife node list
first.example.com
\end{lstlisting}

And we can check what node created on server:

\begin{lstlisting}[language=Bash,label=lst:my-server-cloud-node6]
$ knife node list
first.example.com
$ knife client show first.example.com
admin:      false
chef_type:  client
json_class: Chef::ApiClient
name:       first.example.com
public_key: -----BEGIN PUBLIC KEY-----
...
-----END PUBLIC KEY-----

validator:  false
\end{lstlisting}

\subsection{Node in Vagrant}

You can automate registration of node with your Chef Server in Vagrant. Let's add new node in Vagrantfile:

\begin{lstlisting}[label=lst:my-server-cloud-node7,title=my-server-cloud/Vagrantfile]
...

Vagrant.configure(VAGRANTFILE_API_VERSION) do |config|
  ...
  config.vm.define :chef_second_client do |chef_client|
    chef_client.vm.box = "precise64"
    chef_client.vm.network "private_network", ip: "10.33.33.35"
    chef_client.vm.provision :chef_client do |chef|
      chef.chef_server_url = Chef::Config[:chef_server_url]
      chef.validation_key_path = Chef::Config[:validation_key]
      chef.validation_client_name = Chef::Config[:validation_client_name]
      chef.node_name = 'second.example.com'

      chef.delete_node = true
      chef.delete_client = true
    end
  end

end
\end{lstlisting}

As you can see, options for <<chef\_client>> the same as we set in knife.rb. After command <<vagrant up>> you can check what new node registered:

\begin{lstlisting}[language=Bash,label=lst:my-server-cloud-node10]
$ knife node list
first.example.com
second.example.com
\end{lstlisting}

When you provision your Vagrant virtual machine with Chef server, it creates a new Chef node entry and Chef client entry on the Chef server, using the hostname of the machine. After you tear down your guest machine, Vagrant can be configured to do it automatically with the following settings:

\begin{lstlisting}[label=lst:my-server-cloud-node8]
chef.delete_node = true
chef.delete_client = true
\end{lstlisting}

If you don't specify it or set it to false, you must explicitly delete these entries from the Chef server before you provision a new one with Chef server. For example, using Chef's built-in knife tool:

\begin{lstlisting}[language=Bash,label=lst:my-server-cloud-node9]
$ knife node delete second.example.com
$ knife client delete second.example.com
\end{lstlisting}

And how works cleanup for destroy of server:

\begin{lstlisting}[language=Bash,label=lst:my-server-cloud-node11]
$ vagrant destroy chef_second_client
Are you sure you want to destroy the 'chef_second_client' VM? [y/N] y
[chef_second_client] Forcing shutdown of VM...
[chef_second_client] Destroying VM and associated drives...
[chef_second_client] Running cleanup tasks for 'chef_client' provisioner...
Deleting client "second.example.com" from Chef server...
Deleting node "second.example.com" from Chef server...
$ knife node list
first.example.com
$ knife client list
chef-validator
chef-webui
first.example.com
\end{lstlisting}

As you can see node and client removed from Chef server.
