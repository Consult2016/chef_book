\section{Role}
\label{sec:server-role}

Role work in the same way, as in Chef Solo (chapter <<\ref{sec:solo-role}~\nameref{sec:solo-role}>>). For example, we will install on <<first.example.com>> \href{http://community.opscode.com/cookbooks/nginx}{nginx}. First of all, we need add nginx cookbook in Berksfile:

\begin{lstlisting}[label=lst:my-serer-cloud-role1,title=my-server-cloud/Berksfile]
source "http://api.berkshelf.com"

cookbook 'chef-server'
cookbook 'nginx'
cookbook 'yum', '~> 3.0'
\end{lstlisting}

After command \inline{berks install} we will create <<nginx>> role:

\begin{lstlisting}[language=JSON,label=lst:my-serer-cloud-role2,title=my-server-cloud/roles/nginx.json]
{
  "name": "nginx",
  "description": "The base role for systems that serve web server",
  "chef_type": "role",
  "json_class": "Chef::Role",
  "default_attributes": {
    "nginx": {
      "install_method": "source",
      "version": "1.4.4",
      "default_site_enabled": true,
      "source": {
        "url": "http://nginx.org/download/nginx-1.4.4.tar.gz"
      },
      "worker_rlimit_nofile": 30000,
      "worker_connections": 4000
    }
  },
  "run_list": [
    "recipe[nginx]"
  ]
}
\end{lstlisting}

And <<first.example.com>> node:

\begin{lstlisting}[language=JSON,label=lst:my-serer-cloud-role3,title=my-server-cloud/nodes/first.example.com.json]
{
  "name": "first.example.com",
  "json_class": "Chef::Node",
  "chef_type": "node",
  "normal": {
    "fqdn": "10.33.33.34"
  },
  "default": {},
  "override": {},
  "run_list": [
    "role[nginx]"
  ]
}
\end{lstlisting}

As you can see, node have different format, than Chef Solo. Keys \inline{normal}, \inline{default} and \inline{override} contain attributes. Difference of this attributes you can read in chapter <<\ref{sec:server-attributes}~\nameref{sec:server-attributes}>>. Now we should upload cookbooks, role and node in Chef Server. For this we can use knife and berkshelf:

\begin{lstlisting}[language=Bash,label=lst:my-serer-cloud-role4]
// upload all cookbooks from path 'cookbooks' and 'site-cookbooks' (use --force if cookbook frozen).
// But for vendor cookbooks before you need execute 'berks install --path cookbooks'
$ knife cookbook upload -a
// or upload all cookbooks by berks
$ berks upload
// create/update role from file
$ knife role from file roles/nginx.json
// update node from file
$ knife node from file nodes/first.example.com.json
\end{lstlisting}

Next we should can use \inline{knife ssh} command to do something on nodes. The \inline{knife ssh} subcommand is used to invoke SSH commands (in parallel) on a subset of nodes within an organization, based on the results of a search query. Example:

\begin{lstlisting}[language=Bash,label=lst:my-serer-cloud-role5]
// execute on 'first.example.com' chef-client
$ knife ssh 'name:first.example.com' 'sudo chef-client' -i ~/.vagrant.d/insecure_private_key -x vagrant
10.33.33.34 Starting Chef Client, version 11.8.2
10.33.33.34 resolving cookbooks for run list: ["nginx::source"]
...
10.33.33.34 Recipe: nginx::source
10.33.33.34   * service[nginx] action nothing (up to date)
10.33.33.34   * service[nginx] action reload
10.33.33.34     - reload service service[nginx]
10.33.33.34
10.33.33.34 Chef Client finished, 4 resources updated
\end{lstlisting}

After first run of this command next time you can get errors like <<Failed to connect to>>. It is because FQDN set by Ohai in chef-client will not visible hostname (in my example nodes have <<precise64>> hostname). In real cluster you will not have such problems, because you will use real hostnames. But in our case we can use Vagrant <<chef\_client>> stuff. Let's create node <<second.example.com>> and upload it on server:

\begin{lstlisting}[language=JSON,label=lst:my-serer-cloud-role6,title=my-server-cloud/nodes/second.example.com.json]
{
  "name": "second.example.com",
  "json_class": "Chef::Node",
  "chef_type": "node",
  "normal": {
    "fqdn": "10.33.33.35"
  },
  "default": {},
  "override": {},
  "run_list": [
    "role[nginx]"
  ]
}
\end{lstlisting}

\begin{lstlisting}[language=Bash,label=lst:my-serer-cloud-role7]
$ knife node from file nodes/second.example.com.json
\end{lstlisting}

Now you can just run \inline{vagrant provision} for second node:

\begin{lstlisting}[language=Bash,label=lst:my-serer-cloud-role8]
$ vagrant provision chef_second_client
[chef_second_client] Running provisioner: chef_client...
Creating folder to hold client key...
Uploading chef client validation key...
Generating chef JSON and uploading...
[chef_second_client] Warning: Chef run list is empty. This may not be what you want.
Running chef-client...
stdin: is not a tty
INFO: Forking chef instance to converge...
INFO: *** Chef 11.8.2 ***
INFO: Chef-client pid: 1198
...
INFO: Running report handlers
INFO: Report handlers complete
\end{lstlisting}

It will run chef-client on server, which will get all needed info how to <<cook>> node from Chef Server. After this you can check what on url \href{http://10.33.33.35/}{http://10.33.33.35/} running nginx.
