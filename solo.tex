\chapter{Chef Solo}

Chef Solo is a simple way to begin working with Chef. It is an open source version of the chef-client that allows using cookbooks with nodes without requiring access to a server. Chef Solo runs locally and requires that a cookbook (and any of its dependencies) be on the same physical disk as the node. It is a limited-functionality version of the chef-client and does not support the following:

\begin{itemize}
  \item Node data storage
  \item Search indexes
  \item Centralized distribution of cookbooks
  \item A centralized API that interacts with and integrates infrastructure components
  \item Authentication or authorization
  \item Persistent attributes
\end{itemize}

We will learn Chef Solo by practical examples in this chapter.

\section{Required software}

To get started working with Chef, you need to install the required software:

\begin{itemize}
  \item \href{https://www.virtualbox.org/}{Virtual Box} to provide a local virtual machine to manage using Chef
  \item \href{http://www.vagrantup.com/}{Vagrant} to give a command line interface to manage Virtual Box
  \item \href{http://git-scm.com/}{Git} to revision a Chef code
  \item \href{https://www.ruby-lang.org}{Ruby} (since Chef runs on it). For simple installation of it you can use the \href{https://rvm.io/}{RVM} or \href{https://github.com/sstephenson/rbenv}{Rbenv}
\end{itemize}

Next, create a directory called <<my-cloud>> and inside it create file <<Gemfile>> with this content:

\begin{lstlisting}[label=lst:my-cloud-required1,title=my-cloud/Gemfile]
source "https://rubygems.org"

gem 'knife-solo'
gem 'berkshelf'
\end{lstlisting}

and run command \inline{bundle} in terminal. As a result, by using \href{http://bundler.io/}{bundler} we install a required rubygems. Why do we need these rubygems?

\subsection{Knife-solo}

\href{http://matschaffer.github.io/knife-solo/}{Knife-solo} adds 5 subcommands to knife tool:

\begin{itemize}
  \item <<knife solo init>> is used to create a new directory structure (i.e. kitchen) that fits with Chef's standard structure and can be used to build and store recipes.
  \item <<knife solo prepare>> installs Chef on a given host. It's structured to auto-detect the target OS and change the installation process accordingly.
  \item <<knife solo cook>> uploads the current kitchen to the target host and runs chef-solo on that host.
  \item <<knife solo bootstrap>> combines the two previous ones (prepare and cook).
  \item <<knife solo clean>> removes the uploaded kitchen from the target host.
\end{itemize}

Knife-solo also integrates with Berkshelf and Librarian-Chef.

\subsection{Berkshelf}

\href{http://berkshelf.com/}{Berkshelf} used as well as the bundler for rubygems - it manage a cookbooks and its dependencies. Also, there is exists a \href{https://github.com/applicationsonline/librarian-chef}{librarian-chef}, which performs a similar functions. I prefer to use berkshelf (it have a bigger set of features and integrations).

\section{Creation of kitchen (chef-repo)}
\label{sec:solo-kitchen}

Working with Chef Solo starts with creating kitchen (chef-repo). The kitchen is located on a workstation (the location from which most users will do most of their work, ie. your computer) and should be synchronized with a version control system. To create a kitchen use knife-solo rubygems:

\begin{lstlisting}[language=Bash,label=lst:my-cloud-kitchen1,title=my-cloud]
$ cd my-cloud
$ knife solo init .
WARNING: No knife configuration file found
Creating kitchen...
Creating knife.rb in kitchen...
Creating cupboards...
Setting up Berkshelf...
$ ls -o
total 32
-rw-r--r--  1 leo    14 Dec 14 00:36 Berksfile
-rw-r--r--@ 1 leo    63 Dec 14 00:36 Gemfile
-rw-r--r--  1 leo  4427 Dec 14 00:36 Gemfile.lock
drwxr-xr-x  3 leo   102 Dec 14 00:36 cookbooks
drwxr-xr-x  3 leo   102 Dec 14 00:36 data_bags
drwxr-xr-x  3 leo   102 Dec 14 00:36 environments
drwxr-xr-x  3 leo   102 Dec 14 00:36 nodes
drwxr-xr-x  3 leo   102 Dec 14 00:36 roles
drwxr-xr-x  3 leo   102 Dec 14 00:36 site-cookbooks
\end{lstlisting}

Let's consider the directory structure:

\begin{itemize}
  \item \lstinline!.chef! - a hidden directory that is used to store .pem files and the knife.rb file
  \item \lstinline!cookbooks! - directory for Chef cookbooks. This directory will be used for vendor cookbooks, that will be installed with the help of berkshelf
  \item \lstinline!data_bags! - directory for Chef Data Bags
  \item \lstinline!environments! - directory for Chef environments
  \item \lstinline!nodes! - directory for Chef nodes
  \item \lstinline!roles! - directory for Chef roles
  \item \lstinline!site-cookbooks! - directory for your custom Chef cookbooks
  \item \lstinline!Berksfile! - file contains a list of sources identifying which cookbooks to retrieve and where to get them for berkshelf (like Gemfile for rubygems)
\end{itemize}

<<Cookbooks>> directory added into <<.gitignore>>, because it contains only vendor cookbooks. Vendor cookbook data will not be modified, so there is no reason to keep them in VCS (git, mercurial, etc).

\section{.Chef folder}\label{sec:solo-chef-folder}

After creating the kitchen you should find the <<.chef>> directory. The <<.chef>> directory will contain the configuration and credentials it use knife, the command line interface for Chef.

For Chef Solo in this directory mostly contained only knife.rb file. A knife.rb file is used to specify the chef-repo-specific configuration details for Knife. This file is the default configuration file and is loaded every time this executable is run. The configuration file is located at: <<~/.chef/knife.rb>>. If a knife.rb file is present in the .chef/knife.rb directory in the chef-repo, the settings contained within that file will override the default configuration settings.

My knife.rb have such content:

\begin{lstlisting}[label=lst:my-cloud-chef-filder1,title=my-cloud/.chef/knife.rb]
cookbook_path    ["cookbooks", "site-cookbooks"]
node_path        "nodes"
role_path        "roles"
environment_path "environments"
data_bag_path    "data_bags"
#encrypted_data_bag_secret "data_bag_key"

knife[:berkshelf_path] = "cookbooks"
\end{lstlisting}

Let's consider an options:

\begin{itemize}
  \item \textbf{cookbook\_path} - the sub-directory for cookbooks on the chef-client
  \item \textbf{node\_path} - the sub-directory for nodes on the chef-client
  \item \textbf{role\_path} - the sub-directory for roles on the chef-client
  \item \textbf{environment\_path} - the sub-directory for environments on the chef-client
  \item \textbf{data\_bag\_path} - the sub-directory for Data Bags on the chef-client
  \item \textbf{knife[:berkshelf\_path]} - as you remember, knife-solo gem have integration with Berkshelf and Librarian-Chef. By this option we set directory in which knife will install vendor cookbooks from Berksfile before cook node
\end{itemize}

\section{Vendor cookbooks and berkshelf}

Suppose, that our task to install Apache2 on node. For this purpose we can use vendor cookbook through berkshelf. Huge amount cookbooks you can find at Chef community website\footnote{http://community.opscode.com/cookbooks}. To install Apache2 cookbook we need add it to Berksfile:

\begin{lstlisting}[label=lst:my-cloud-berkshelf1,title=my-cloud/Berksfile]
site :opscode

cookbook 'apache2'
\end{lstlisting}

After running the command <<berks install>> this cookbook will be installed with dependencies.

\begin{lstlisting}[language=Bash,label=lst:my-cloud-berkshelf2,title=my-cloud/Berksfile]
$ berks install
Using apache2 (1.7.0)
\end{lstlisting}

By default, you will not find this cookbook in <<cookbooks>> directory. Berkshelf install cookbooks in <<~/.berkshelf>> directory (to avoid duplication of cookbooks). To install it in special path you can use \texttt{-\--path} option in berkshelf:

\begin{lstlisting}[language=Bash,label=lst:my-cloud-berkshelf3,title=my-cloud/Berksfile]
$ berks install --path cookbooks
Using apache2 (1.7.0)
$ ls cookbooks
apache2
\end{lstlisting}

Option <<knife[:berkshelf\_path]>> in knife.rb file set to install our cookbooks into cookbooks directory, so no need to run <<berks install -\--path cookbooks>> each time, when you need cook a node - knife will do it automatically.




\section{Defining nodes}
\label{sec:solo-node}

Node itself represent any physical, virtual, or cloud machine. In most cases number of nodes in kitchen equal of number of machines in your cloud. To install apache2 to the correct server, we need to create a file in the nodes folder. Basically, this file is named as machine domain. For example, I attach to the machine domain <<web1.example.com>> and create node for it:

\begin{lstlisting}[language=JSON,label=lst:my-cloud-node1,title=my-cloud/nodes/web1.example.com.json]
{
  "run_list": []
}
\end{lstlisting}

Node file should contain valid JSON document. Main key in this json is \inline{run_list}. This key contain array of recipes and roles, which should be executed on mashine. It always executed in the same order as listed in this key. As you can see right now \inline{run_list} is empty. To install apache2 you need to add the recipe in this array. This information can be found in README (if well written cookbook), in metadata.rb or just open directory recipes - the name of the file will mean recipe name. \inline{default.rb} file mean a recipe that will be executed when you call the cookbook in \inline{run_list} without designation of recipe. Examples:

\begin{lstlisting}[language=JSON,label=lst:my-cloud-node2,title=my-cloud/nodes/web1.example.com.json]
{
  "run_list": [
    "recipe[apache2]"
  ]
}
\end{lstlisting}

This \inline{run_list} will execute default recipe from apache2 cookbook.

Now all you need to check this kitchen.

\section{Vagrant}

For testing our chef kitchen in most cases we are using Vagrant. So what is Vagrant?

Vagrant is free and open-source software for creating and configuring virtual development environments. It can be considered a wrapper around VirtualBox and configuration management software such as Chef. Since version 1.1, Vagrant is no longer tied to VirtualBox and also works with other virtualization software such as VMware and Amazon EC2.

Instead of building a virtual machine from scratch, which would be a slow and tedious process, Vagrant uses a base image to quickly clone a virtual machine. These base images are known as boxes in Vagrant, and specifying the box to use for your Vagrant environment is always the first step after creating a new Vagrantfile. For testing we will use precise64 (Ubuntu 12.04 LTS 64-bit) box.

\begin{lstlisting}[label=lst:my-cloud-vagrant1]
$ vagrant box add precise64 http://files.vagrantup.com/precise64.box
\end{lstlisting}

More boxes you can find on \href{http://www.vagrantbox.es/}{vagrantbox.es}.

The first step for any project to use Vagrant is to configure Vagrant using a Vagrantfile. We should execute <<vagrant init precise64>> inside kitchen directory:

\begin{lstlisting}[label=lst:my-cloud-vagrant2]
$ vagrant init precise64
A `Vagrantfile` has been placed in this directory. You are now
ready to `vagrant up` your first virtual environment! Please read
the comments in the Vagrantfile as well as documentation on
`vagrantup.com` for more information on using Vagrant.
\end{lstlisting}

By default, Vagrantfile have such content:

\begin{lstlisting}[language=Ruby,label=lst:my-cloud-vagrant3,title=my-cloud/nodes/Vagrantfile]
# -*- mode: ruby -*-
# vi: set ft=ruby :

# Vagrantfile API/syntax version. Don't touch unless you know what you're doing!
VAGRANTFILE_API_VERSION = "2"

Vagrant.configure(VAGRANTFILE_API_VERSION) do |config|
  config.vm.box = "precise64"
end
\end{lstlisting}

We can check, what vagrant is working fine by command <<vagrant up>>:

\begin{lstlisting}[label=lst:my-cloud-vagrant4]
$ vagrant up
Bringing machine 'default' up with 'virtualbox' provider...
[default] Importing base box 'precise64'...
[default] Matching MAC address for NAT networking...
[default] Setting the name of the VM...
[default] Clearing any previously set forwarded ports...
[default] Creating shared folders metadata...
[default] Clearing any previously set network interfaces...
[default] Preparing network interfaces based on configuration...
[default] Forwarding ports...
[default] -- 22 => 2222 (adapter 1)
[default] Booting VM...
[default] Waiting for machine to boot. This may take a few minutes...
[default] Machine booted and ready!
[default] Mounting shared folders...
[default] -- /vagrant
\end{lstlisting}

After this you can use command <<vagrant ssh>> to SSH into a running Vagrant machine and give you access to a shell. Command <<vagrant halt>> shuts down the running machine Vagrant is managing. <<vagrant destroy>> command stops the running machine Vagrant is managing and destroys all resources that were created during the machine creation process.

By default, in most cases, your mashine is not contain chef client and we should install it. As you remember, we have knife with command <<prepare>>. Let use this command:

\begin{lstlisting}[label=lst:my-cloud-vagrant5]
$ knife solo prepare vagrant@localhost -i ~/.vagrant.d/insecure_private_key -p 2222 -N web1.example.com
Bootstrapping Chef...
--2013-12-27 19:12:56--  https://www.opscode.com/chef/install.sh
Resolving www.opscode.com (www.opscode.com)... 184.106.28.91
...
Installing Chef 11.8.2
installing with dpkg...
Selecting previously unselected package chef.
(Reading database ... 51095 files and directories currently installed.)
Unpacking chef (from .../chef_11.8.2_amd64.deb) ...
Setting up chef (11.8.2-1.ubuntu.12.04) ...
Thank you for installing Chef!
\end{lstlisting}

As you can see, we use options <<-i>> to set ssh key for mashine, <<-N>> to set node name (if it different from host name) and <<-p>> to set SSH port. In most cases, this port is 2222, but if you running several mashines from vagrant, it will be different. You can read what port is used for SSH by mashine from output of command <<vagrant up>>.

Vagrantfile inside use ruby syntax, so we can use Ruby to \href{http://en.wikipedia.org/wiki/Dont\_repeat\_yourself}{DRY} our config. We should install chef gem inside vagrant:

\begin{lstlisting}[label=lst:my-cloud-vagrant6]
$ vagrant plugin install chef
Installing the 'chef' plugin. This can take a few minutes...
Installed the plugin 'chef (11.8.2)'!
\end{lstlisting}

Next, we should define chef solo in Vagrantfile:

\begin{lstlisting}[language=Ruby,label=lst:my-cloud-vagrant7,title=my-cloud/nodes/Vagrantfile]
# -*- mode: ruby -*-
# vi: set ft=ruby :

require 'chef'
require 'json'

Chef::Config.from_file(File.join(File.dirname(__FILE__), '.chef', 'knife.rb'))
vagrant_json = JSON.parse(Pathname(__FILE__).dirname.join('nodes', (ENV['NODE'] || 'web1.example.com.json')).read)

# Vagrantfile API/syntax version. Don't touch unless you know what you're doing!
VAGRANTFILE_API_VERSION = "2"

Vagrant.configure(VAGRANTFILE_API_VERSION) do |config|
  config.vm.box = "precise64"

  config.vm.provision :chef_solo do |chef|
    chef.cookbooks_path = Chef::Config[:cookbook_path]
    chef.roles_path = Chef::Config[:role_path]
    chef.data_bags_path = Chef::Config[:data_bag_path]

    chef.environments_path = Chef::Config[:environment_path]
    #chef.environment = ENV['ENVIRONMENT'] || 'development'

    chef.run_list = vagrant_json.delete('run_list')
    chef.json = vagrant_json
  end
end
\end{lstlisting}

Now consider that we have added.

On lines 4-5 required chef and json gem. JSON gem is as part of Vagrant, but chef gem we installed by previous command <<vagrant plugin install chef>>. After this we load knife.rb in chef config and parse json from node file <<web1.example.com.json>>. After this we have <<Chef::Config>> ruby hash with knife configuration and <<vagrant\_json>> ruby hash will attributes from node. On lines 16-26 defined chef solo configuration for vagrant (we commented <<environment>>, because we don't have it right now, but we will use it later). More information about this you can find in \href{http://docs.vagrantup.com/v2/provisioning/chef\_solo.html}{vagrant website}.

Basically, to run our kitchen on server we are using <<knife solo cook>> command:

\begin{lstlisting}[label=lst:my-cloud-vagrant11]
$ knife solo cook vagrant@localhost -i ~/.vagrant.d/insecure_private_key -p 2222 -N web1.example.com
Running Chef on localhost...
Checking Chef version...
Installing Berkshelf cookbooks to 'cookbooks'...
Using apache2 (1.7.0)
Uploading the kitchen...
...
 * service[apache2] action start (up to date)
Chef Client finished, 1 resources updated
\end{lstlisting}

But in Vagrant we can use build in command <<vagrant provision>>. This command runs any configured provisioners against the running Vagrant managed machine (we changed Vagrantfile, what is why we run <<vagrant reload>> before it):

\begin{lstlisting}[label=lst:my-cloud-vagrant8]
$ vagrant reload
[default] Attempting graceful shutdown of VM...
...
[default] -- /vagrant
[default] -- /tmp/vagrant-chef-1/chef-solo-3/roles
[default] -- /tmp/vagrant-chef-1/chef-solo-2/cookbooks
[default] -- /tmp/vagrant-chef-1/chef-solo-1/cookbooks
[default] -- /tmp/vagrant-chef-1/chef-solo-4/data_bags
[default] -- /tmp/vagrant-chef-1/chef-solo-5/environments
$ vagrant provision
[default] Running provisioner: chef_solo...
Generating chef JSON and uploading...
Running chef-solo...
stdin: is not a tty
[2013-12-27T19:35:59+00:00] INFO: Forking chef instance to converge...
[2013-12-27T19:35:59+00:00] INFO: *** Chef 11.8.2 ***
...
[2013-12-27T19:36:18+00:00] INFO: service[apache2] restarted
[2013-12-27T19:36:18+00:00] INFO: Chef Run complete in 18.115479422 seconds
[2013-12-27T19:36:18+00:00] INFO: Running report handlers
[2013-12-27T19:36:18+00:00] INFO: Report handlers complete
\end{lstlisting}

To verify that the apache2 successfully installed into mashine, we can forward the 80 apache2 port. Modify <<Vagrantfile>>:

\begin{lstlisting}[language=Ruby, label=lst:my-cloud-vagrant9,title=my-cloud/nodes/Vagrantfile]
...

Vagrant.configure(VAGRANTFILE_API_VERSION) do |config|
  config.vm.box = "precise64"
  config.vm.network :forwarded_port, guest: 80, host: 8085 # <== add port forwarding

...
\end{lstlisting}

And reload vagrant:

\begin{lstlisting}[label=lst:my-cloud-vagrant10]
$ vagrant reload
...
[default] Preparing network interfaces based on configuration...
[default] Forwarding ports...
[default] -- 22 => 2222 (adapter 1)
[default] -- 80 => 8085 (adapter 1)
...
\end{lstlisting}

Now in any of your browser you can open url \href{http://localhost:8085/}{http://localhost:8085/} and see page 404 from apache2 server.
\section{Idempotence}

As you should read from previous chapter, one of the main idea of Chef is idempotence. It mean, what Chef can safely be run multiple times on the same mashine. Once you develop your configuration, your machines will apply the configuration and Chef will only make any changes to the system if the system state does not match the configured state.

For now we have mashine which contain apache2 running inside it. Let's run <<vagrant provision>> again:

\begin{lstlisting}[label=lst:my-cloud-idempotence1,title=my-cloud/nodes/Vagrantfile]
$ vagrant provision
[default] Running provisioner: chef_solo...
Generating chef JSON and uploading...
Running chef-solo...
stdin: is not a tty
...
[2013-12-27T19:54:10+00:00] INFO: Chef Run complete in 1.092279021 seconds
...
\end{lstlisting}

As you can see it did nothing, because configuration of the server the same as in chef kitchen (what is why execution time also so small).
\section{Defining roles}
\label{sec:solo-role}

Roles help classify the same server group. For example, in your project you can have web, queue and db servers. In this case you can create such type of roles, which will include the same attributes and \inline{run_list} for nodes. Let's look at an example.

For example, in our project we have a web application servers, load balancer server and database server. First we will create roles <<web>>:

\begin{lstlisting}[label=lst:my-cloud-role1,title=my-cloud/roles/web.json]
{
  "name": "web",
  "description": "The base role for systems that serve web server",
  "chef_type": "role",
  "json_class": "Chef::Role",
  "default_attributes": {
    "apache": {
      "listen_ports": ["80", "443"]
    }
  },
  "run_list": [
    "recipe[apache2]"
  ]
}
\end{lstlisting}

Let's consider current json structure:

\begin{itemize}
  \item \inline{name} - a unique name of role. In most cases the same as name of file without extension
  \item \inline{description} - a description of the functionality that is covered by role
  \item \inline{chef_type} - this should always be set to <<role>>
  \item \inline{json_class} - this should always be set to <<Chef::Role>>
  \item \inline{default_attributes} - a set of attributes that should be applied to all nodes, assuming the node does not already have a value for the attribute. This is useful for setting global defaults that can then be overridden for specific nodes. If more than one role attempts to set a default value for the same attribute, the last role applied will be the role to set the attribute value. This attribute is optional
  \item \inline{override_attributes} - a set of attributes that should be applied to all nodes, even if the node already has a value for an attribute. This is useful for ensuring that certain attributes always have specific values. If more than one role attempts to set an override value for the same attribute, the last role applied will win. This attribute is optional
  \item \inline{run_list} - a list of recipes and/or roles that will be applied and the order in which those recipes and/or roles will be applied
  \item \inline{env_run_lists} - a list of environments, each specifying a recipe or a role that will be applied to that environment. This attribute is optional
\end{itemize}

To use this role, we can create new node <<web2.example.com>> with content:

\begin{lstlisting}[label=lst:my-cloud-role2,title=my-cloud/nodes/web2.example.com.json]
{
  "run_list": [
    "role[web]"
  ]
}
\end{lstlisting}

And check that everything is working:

\begin{lstlisting}[language=Bash,label=lst:my-cloud-role3]
$ NODE=web2.example.com.json vagrant provision
[default] Running provisioner: chef_solo...
Generating chef JSON and uploading...
Running chef-solo...
stdin: is not a tty
INFO: Forking chef instance to converge...
INFO: *** Chef 11.8.2 ***
INFO: Chef-client pid: 1224
INFO: Setting the run_list to ["role[web]"] from JSON
INFO: Run List is [role[web]]
INFO: Run List expands to [apache2]
...
INFO: Chef Run complete in 1.437157496 seconds
INFO: Running report handlers
INFO: Report handlers complete
\end{lstlisting}

As you can see, role defined in \inline{run_list} by command \inline{role} and role command replaced to run list of this role by chef client. This allow for use use several roles with recipes in the same node. For example, if web and database role should exists on the same node (example: staging environment), you can define \inline{run_list} in node in such way:

\begin{lstlisting}[label=lst:my-cloud-role4,title=my-cloud/nodes/web2.example.com.json]
{
  "run_list": [
    "role[web]",
    "role[db]"
  ]
}
\end{lstlisting}

BTW, role can contain in run list another roles. For example:

\begin{lstlisting}[label=lst:my-cloud-role5,title=my-cloud/roles/test.json]
{
  "name": "test",
  "description": "The test role, it is not used in kitchen",
  "chef_type": "role",
  "json_class": "Chef::Role",
  "run_list": [
    "role[web]",
    "recipe[postgresql]"
  ]
}
\end{lstlisting}

Role can contain attributes inside \inline{default_attributes} or \inline{override_attributes} keys. In our example, <<web>> role can contain general settings for http listen ports, timeout of web server, etc. So let's look in more detail about the use of attributes.

\section{Attributes}

An attribute is a specific detail about a node. Attributes are used by the chef-client to understand:

\begin{itemize}
  \item The current state of the node
  \item What the state of the node was at the end of the previous chef-client run
  \item What the state of the node should be at the end of the current chef-client run
\end{itemize}

As you read from previous sections of book, attributes can be defined in node and role, but it's also can be defined by environments and cookbooks.



\section{Defining environments}

An environment is a way to map an organization's real-life workflow to what can be configured and managed when using server. Every organization begins with a single environment called the <<\_default>> environment, which cannot be modified (or deleted). Additional environments can be created to reflect each organization's patterns and workflow. For example, creating production, staging, testing, and development environments. Generally, an environment is also associated with one (or more) cookbook versions.

We create for our example development environment:

\begin{lstlisting}[label=lst:my-cloud-chef-environment1,title=my-cloud/environments/development.json]
{
  "name": "development",
  "description": "development environment",
  "chef_type": "environment",
  "json_class": "Chef::Environment",
  "default_attributes": {
    "apache2": {
      "listen_ports": ["80","443"]
    }
  }
}
\end{lstlisting}

Let's consider a json structure:

\begin{itemize}
  \item \textbf{name} - a unique name of environment
  \item \textbf{description} - a description of the environment
  \item \textbf{chef\_type} - this should always be set to environment
  \item \textbf{json\_class} - this should always be set to Chef::Environment
  \item \textbf{default\_attributes} - a set of attributes that should be applied to all nodes, assuming the node does not already have a value for the attribute. This is useful for setting global defaults that can then be overridden for specific nodes. This attribute is optional
  \item \textbf{override\_attributes} - a set of attributes that should be applied to all nodes, even if the node already has a value for an attribute. This is useful for ensuring that certain attributes always have specific values. This attribute is optional
\end{itemize}

As you can see environment doesn't have run\_list, but it have attributes. Attributes in most cases contain information, which specific for environment: connection information to databases, cluster settings for database or queue, etc.

Now we can activate this development environment in Vagrantfile:

\begin{lstlisting}[label=lst:my-cloud-chef-environment2,title=my-cloud/Vagrantfile]
Vagrant.configure(VAGRANTFILE_API_VERSION) do |config|
  ...
    chef.environments_path = Chef::Config[:environment_path]
    chef.environment = ENV['ENVIRONMENT'] || 'development'
  ...
end
\end{lstlisting}

and check how it works:

\begin{lstlisting}[language=Bash,label=lst:my-cloud-chef-environment3]
$ vagrant provision
...
[2013-12-31T21:53:57+00:00] INFO: Chef Run complete in 1.105324838 seconds
[2013-12-31T21:53:57+00:00] INFO: Running report handlers
[2013-12-31T21:53:57+00:00] INFO: Report handlers complete
\end{lstlisting}

To <<cook>> server by knife with environment you should use <<-E>> argument:

\begin{lstlisting}[language=Bash,label=lst:my-cloud-chef-environment4]
$ knife solo cook vagrant@localhost -i ~/.vagrant.d/insecure_private_key -p 2222 -N web1.example.com -E development
Running Chef on localhost...
Checking Chef version...
...
  * service[apache2] action start (up to date)
Chef Client finished, 1 resources updated
\end{lstlisting}

As you can see in Chef Solo environment can be used only for setting attributes. In Chef Server it have additional feature for locking cookbook versions, which we will consider in Chef Server chapter.
\section{Defining data bags}
\label{sec:solo-data-bag}

A data bag is a global variable that is stored as JSON data. The contents of a data bag can vary, but they often include sensitive information (such as database passwords).

Knife is not working with Chef Solo data bags, but we can use \href{http://thbishop.com/knife-solo\_data\_bag/}{knife-solo\_data\_bag} rubygem. Just add this gem in Gemfile:

\begin{lstlisting}[label=lst:my-cloud-chef-databag1,title=my-cloud/Gemfile]
source "https://rubygems.org"

gem 'knife-solo'
gem 'knife-solo_data_bag'
gem 'berkshelf'
\end{lstlisting}

and run command \inline{bundle} to install it.

After instalation you knife should have new commands for working with Chef Solo:

\begin{lstlisting}[language=Bash,label=lst:my-cloud-chef-databag2]
$ knife --help | grep solo
  knife solo cook [USER@]HOSTNAME [JSON] (options)
  knife solo init DIRECTORY
  knife solo prepare [USER@]HOSTNAME [JSON] (options)
knife solo bootstrap [USER@]HOSTNAME [JSON] (options)
knife solo clean [USER@]HOSTNAME
knife solo cook [USER@]HOSTNAME [JSON] (options)
knife solo init DIRECTORY
knife solo prepare [USER@]HOSTNAME [JSON] (options)
knife solo data bag create BAG [ITEM] (options)
knife solo data bag edit BAG ITEM (options)
knife solo data bag list (options)
knife solo data bag show BAG [ITEM] (options)
  knife solo clean [USER@]HOSTNAME
\end{lstlisting}

For beginning, let's create a plain text data bag:

\begin{lstlisting}[language=Bash,label=lst:my-cloud-chef-databag3]
$ EDITOR=vim knife solo data bag create pass mysql
Created data_bag_item[mysql]
\end{lstlisting}

\inline{EDITOR} environment variable is used to set editor, which will open data bag file. I add in this JSON password for database and save it. Now we can see the result:

\begin{lstlisting}[language=Bash,label=lst:my-cloud-chef-databag4]
$ knife solo data bag show pass
mysql:
  id:       mysql
  password: secret
\end{lstlisting}

If you open data bag file, you will see this JSON:

\begin{lstlisting}[language=JSON,label=lst:my-cloud-chef-databag10,title=my-cloud/data\_bags/pass/mysql.json]
{
  "name":"data_bag_item_pass_mysql",
  "chef_type":"data_bag_item",
  "json_class":"Chef::DataBagItem",
  "data_bag":"pass",
  "raw_data":{
    "id":"mysql",
    "password":"secret"
  }
}
\end{lstlisting}

Let's consider a json structure:

\begin{itemize}
  \item \inline{name} - a unique name of data bag
  \item \inline{chef_type} - this should always be set to \inline{data_bag_item}
  \item \inline{json_class} - this should always be set to Chef::DataBagItem
  \item \inline{data_bag} - name of data bag
  \item \inline{raw_data} - values of data bag
\end{itemize}

The contents of a data bag can be encrypted using shared secret encryption. This allows a data bag to store confidential information (such as a database password) or to be managed in a source control system (without plain-text data appearing in revision history).

Encrypting a data bag requires a secret key. A secret key can be created in any number of ways. For example, OpenSSL can be used to generate a random number, which can then be used as the secret key:

\begin{lstlisting}[language=Bash,label=lst:my-cloud-chef-databag5]
$ openssl rand -base64 512 | tr -d '\r\n' > .chef/encrypted_data_bag_secret
\end{lstlisting}

The \inline{tr} command eliminates any trailing line feeds. Doing so avoids key corruption when transferring the file between platforms with different line endings.

A data bag can be encrypted using a Knife command similar to:

\begin{lstlisting}[language=Bash,label=lst:my-cloud-chef-databag6]
$ EDITOR=vim knife solo data bag create passwords mysql --secret-file .chef/encrypted_data_bag_secret
Created data_bag_item[mysql]
\end{lstlisting}

As a result we obtain the encrypted data bag:

\begin{lstlisting}[language=Bash,label=lst:my-cloud-chef-databag7]
$ knife solo data bag show passwords mysql
id:       mysql
password:
  cipher:         aes-256-cbc
  encrypted_data: qsoqyxXYSrQ29EJZZ8dPLx3dhlDsVpalLmkm5IDr4jk=

  iv:             QwXUtR8P5/5e7xb+M0lgJw==

  version:        1
\end{lstlisting}

An encrypted data bag item can be decrypted with a Knife command similar to:

\begin{lstlisting}[language=Bash,label=lst:my-cloud-chef-databag8,title=my-cloud/Gemfile]
$ knife solo data bag show passwords mysql --secret-file .chef/encrypted_data_bag_secret
id:       mysql
password: secret
\end{lstlisting}

You can set \inline{encrypted_data_bag_secret} in knife.rb file:

\begin{lstlisting}[label=lst:my-cloud-chef-databag9,title=my-cloud/.chef/knife.rb]
cookbook_path    ["cookbooks", "site-cookbooks"]
node_path        "nodes"
role_path        "roles"
environment_path "environments"
data_bag_path    "data_bags"
encrypted_data_bag_secret ".chef/encrypted_data_bag_secret"

knife[:berkshelf_path] = "cookbooks"
\end{lstlisting}

and in this case no need to define \inline{secret-file} for knife data bag commands:

\begin{lstlisting}[language=Bash,label=lst:my-cloud-chef-databag11]
$ knife solo data bag show passwords mysql
id:       mysql
password: secret
\end{lstlisting}

For Vagrant you should set \inline{encrypted_data_bag_secret_key_path}:

\begin{lstlisting}[label=lst:my-cloud-chef-databag12,title=my-cloud/Vagrantfile]
...
Vagrant.configure(VAGRANTFILE_API_VERSION) do |config|
  ...
    chef.encrypted_data_bag_secret_key_path = Chef::Config[:encrypted_data_bag_secret]
  ...
end
\end{lstlisting}

The Recipe DSL provides access to data bags and data bag items with the following methods: \inline{{data_bag('bag')}, where bag is the name of the data bag and \inline{data_bag_item('bag', 'item')}, where bag is the name of the data bag and item is the name of the data bag item. Examples:

\begin{lstlisting}[label=lst:my-cloud-chef-databag13]
data_bag("pass")
# => ["mysql"]
item = data_bag_item("pass", "mysql")
item["password"]
# => "secret"
\end{lstlisting}

A recipe can access encrypted data bag items as long as the recipe is running on a node that has access to the shared-key that is required to decrypt the data. A secret can be specified by using the Chef::EncryptedDataBagItem.load method. For example:

\begin{lstlisting}[label=lst:my-cloud-chef-databag14]
mysql_creds = Chef::EncryptedDataBagItem.load("passwords", "mysql", secret_key)
mysql_creds["password"]
# => "secret"
\end{lstlisting}

where \inline{secret_key} is the argument that specifies the location of the file that contains the encryption key. An encryption key can be configured so that the chef-client knows where to look using the \inline{Chef::Config[:encrypted_data_bag_secret]} method, which defaults to \inline{/etc/chef/encrypted_data_bag_secret}. When the default location is used, the argument that specifies the secret key file location is assumed to be the default and does not need to be explicitly specified in the recipe. For example:

\begin{lstlisting}[label=lst:my-cloud-chef-databag15]
mysql_creds = Chef::EncryptedDataBagItem.load("passwords", "mysql")
mysql_creds["password"]
# => "secret"
\end{lstlisting}


\section{Summary}

Chef Solo is a most simple way to begin working with Chef. Also it is very good choice, if your environment small (several servers) and you don't need setup or buy separate Chef Server. But if you have huge numbers of servers or you don't like limited functionality of Chef Solo, in this case you should thinking to setup or buy own Chef Server.
