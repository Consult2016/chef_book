\section{Creation of kitchen (chef-repo)}

Working with Chef Solo starts with creating kitchen (chef-repo). The kitchen is located on a workstation (location from which most users will do most of their work, ie. your computer) and should be synchronized with a version control system. To create a kitchen used knife-solo rubygems:

\begin{lstlisting}[language=Bash,label=lst:my-cloud-kitchen1,title=my-cloud]
$ cd my-cloud
$ knife solo init .
WARNING: No knife configuration file found
Creating kitchen...
Creating knife.rb in kitchen...
Creating cupboards...
Setting up Berkshelf...
$ ls -o
total 32
-rw-r--r--  1 leo    14 Dec 14 00:36 Berksfile
-rw-r--r--@ 1 leo    63 Dec 14 00:36 Gemfile
-rw-r--r--  1 leo  4427 Dec 14 00:36 Gemfile.lock
drwxr-xr-x  3 leo   102 Dec 14 00:36 cookbooks
drwxr-xr-x  3 leo   102 Dec 14 00:36 data_bags
drwxr-xr-x  3 leo   102 Dec 14 00:36 environments
drwxr-xr-x  3 leo   102 Dec 14 00:36 nodes
drwxr-xr-x  3 leo   102 Dec 14 00:36 roles
drwxr-xr-x  3 leo   102 Dec 14 00:36 site-cookbooks
\end{lstlisting}

Let's consider a directory structure:

\begin{itemize}
  \item \textbf{cookbooks} - directory for Chef cookbooks. This directory will be used for vendor cookbooks, that will be installed with the help of berkshelf
  \item \textbf{data\_bags} - directory for Chef Data Bags
  \item \textbf{environments} - directory for Chef environments
  \item \textbf{nodes} - directory for Chef nodes
  \item \textbf{roles} - directory for Chef roles
  \item \textbf{site-cookbooks} - directory for your custom Chef cookbooks
  \item \textbf{Berksfile} - file contains a list of sources identifying what cookbooks to retrieve and where to get them for berkshelf
\end{itemize}

Cookbooks directory added into <<.gitignore>>, because it contains only vendor cookbooks. Need to remember about it when using other version control system.

