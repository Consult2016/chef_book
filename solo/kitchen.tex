\section{Creation of kitchen (сhef-repo)}

Working with Chef Solo starts with creating kitchen(сhef-repo). The kitchen is located on a workstation (location from which most users will do most of their work, ie. your computer) and should be synchronized with a version control system. To get started, you need to install the required software:

\begin{itemize}
  \item Virtual Box\footnote{https://www.virtualbox.org/} to provide a local virtual machine to manage using Chef
  \item Vagrant\footnote{http://www.vagrantup.com/} to give a command line interface to manage Virtual Box
  \item Git\footnote{http://git-scm.com/} to revision a Chef code
  \item Ruby\footnote{https://www.ruby-lang.org} (since Chef runs on it). For simple installation of it you can use the RVM\footnote{https://rvm.io/} or Rbenv\footnote{https://github.com/sstephenson/rbenv}
\end{itemize}

Next, create a directory called <<my-cloud>> and inside it create file <<Gemfile>> with this content:

\begin{lstlisting}[label=lst:my-cloud-kitchen1,title=my-cloud/Gemfile]
source "https://rubygems.org"

gem 'knife-solo'
gem 'berkshelf'
\end{lstlisting}

and run command <<bundle>> in terminal. Upon completion of work should create a file <<Gemfile.lock>>. As a result, by using bundler\footnote{http://bundler.io/} we install a rubygems. Why do we need these gems?

\subsection{Knife-solo}

Knife-solo\footnote{http://matschaffer.github.io/knife-solo/} adds 5 subcommands to knife tool:

\begin{itemize}
  \item <<knife solo init>> is used to create a new directory structure (i.e. kitchen) that fits with Chef's standard structure and can be used to build and store recipes.
  \item <<knife solo prepare>> installs Chef on a given host. It's structured to auto-detect the target OS and change the installation process accordingly.
  \item <<knife solo cook>> uploads the current kitchen to the target host and runs chef-solo on that host.
  \item <<knife solo bootstrap>> combines the two previous ones (prepare and cook).
  \item <<knife solo clean>> removes the uploaded kitchen from the target host.
\end{itemize}

Knife-solo also integrates with Berkshelf and Librarian-Chef.

\subsection{Berkshelf}

Berkshelf\footnote{http://berkshelf.com/} used as well as the bundler for rubygems - it manage a cookbooks and its dependencies. Also, there is exists a librarian-chef\footnote{https://github.com/applicationsonline/librarian-chef}, which performs a similar functions. I prefer to use berkshelf (it have a bigger set of features and integrations).

