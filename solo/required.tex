\section{Required software}

To get started working with Chef, you need to install the required software:

\begin{itemize}
  \item \href{https://www.virtualbox.org/}{Virtual Box} to provide a local virtual machine to manage using Chef
  \item \href{http://www.vagrantup.com/}{Vagrant} to give a command line interface to manage Virtual Box
  \item \href{http://git-scm.com/}{Git} to revision a Chef code
  \item \href{https://www.ruby-lang.org}{Ruby} (since Chef runs on it). For simple installation of it you can use the \href{https://rvm.io/}{RVM} or \href{https://github.com/sstephenson/rbenv}{Rbenv}
\end{itemize}

Next, create a directory called <<my-cloud>> and inside it create file <<Gemfile>> with this content:

\begin{lstlisting}[label=lst:my-cloud-required1,title=my-cloud/Gemfile]
source "https://rubygems.org"

gem 'knife-solo'
gem 'berkshelf'
\end{lstlisting}

and run command \inline{bundle} in terminal. As a result, by using \href{http://bundler.io/}{bundler} we install a required rubygems. Why do we need these rubygems?

\subsection{Knife-solo}

\href{http://matschaffer.github.io/knife-solo/}{Knife-solo} adds 5 subcommands to knife tool:

\begin{itemize}
  \item \inline{knife solo init} is used to create a new directory structure (i.e. kitchen) that fits with Chef's standard structure and can be used to build and store recipes.
  \item \inline{knife solo prepare} installs Chef on a given host. It's structured to auto-detect the target OS and change the installation process accordingly.
  \item \inline{knife solo cook} uploads the current kitchen to the target host and runs chef-solo on that host.
  \item \inline{knife solo bootstrap} combines the two previous ones (prepare and cook).
  \item \inline{knife solo clean} removes the uploaded kitchen from the target host.
\end{itemize}

Knife-solo also integrates with Berkshelf and Librarian-Chef.

\subsection{Berkshelf}

\href{http://berkshelf.com/}{Berkshelf} used as well as the bundler for rubygems - it manage a cookbooks and its dependencies. Also, there is exists a \href{https://github.com/applicationsonline/librarian-chef}{librarian-chef}, which performs a similar functions. I prefer to use berkshelf (it have a bigger set of features and integrations).
