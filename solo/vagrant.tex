\section{Vagrant}
\label{sec:solo-vagrant}

For testing our chef kitchen in most cases we are using Vagrant. So what is Vagrant?

\href{http://www.vagrantup.com/}{Vagrant} is free and open-source software for creating and configuring virtual development environments. It can be considered a wrapper around VirtualBox and configuration management software such as Chef. Since version 1.1, Vagrant is no longer tied to VirtualBox and also works with other virtualization software such as VMware and Amazon EC2.

Instead of building a virtual machine from scratch, which would be a slow and tedious process, Vagrant uses a base image to quickly clone a virtual machine. These base images are known as boxes in Vagrant, and specifying the box to use for your Vagrant environment is always the first step after creating a new Vagrantfile. For testing will be used Ubuntu 12.04 LTS 64-bit (precise64 box).

\begin{lstlisting}[language=Bash,label=lst:my-cloud-vagrant1]
$ vagrant box add precise64 http://files.vagrantup.com/precise64.box
\end{lstlisting}

More boxes you can find on \href{http://www.vagrantbox.es/}{Vagrantbox.es} or \href{https://vagrantcloud.com/}{Vagrant Cloud}.

The first step for any project to use Vagrant is to configure Vagrant using a Vagrantfile. We should execute \lstinline!vagrant init precise64! inside kitchen directory:

\begin{lstlisting}[label=lst:my-cloud-vagrant2]
$ vagrant init precise64
A `Vagrantfile` has been placed in this directory. You are now
ready to `vagrant up` your first virtual environment! Please read
the comments in the Vagrantfile as well as documentation on
`vagrantup.com` for more information on using Vagrant.
\end{lstlisting}

By default, Vagrantfile have such content:

\begin{lstlisting}[language=Ruby,label=lst:my-cloud-vagrant3,title=my-cloud/nodes/Vagrantfile]
# -*- mode: ruby -*-
# vi: set ft=ruby :

# Vagrantfile API/syntax version. Don't touch unless you know what you're doing!
VAGRANTFILE_API_VERSION = "2"

Vagrant.configure(VAGRANTFILE_API_VERSION) do |config|
  config.vm.box = "precise64"
end
\end{lstlisting}

We can check, that vagrant is working fine by command <<vagrant up>>:

\begin{lstlisting}[label=lst:my-cloud-vagrant4]
$ vagrant up
Bringing machine 'default' up with 'virtualbox' provider...
[default] Importing base box 'precise64'...
[default] Matching MAC address for NAT networking...
[default] Setting the name of the VM...
[default] Clearing any previously set forwarded ports...
[default] Creating shared folders metadata...
[default] Clearing any previously set network interfaces...
[default] Preparing network interfaces based on configuration...
[default] Forwarding ports...
[default] -- 22 => 2222 (adapter 1)
[default] Booting VM...
[default] Waiting for machine to boot. This may take a few minutes...
[default] Machine booted and ready!
[default] Mounting shared folders...
[default] -- /vagrant
\end{lstlisting}

After this you can use command \lstinline!vagrant ssh! to SSH into a running Vagrant machine and give you access to a shell. Command <<vagrant halt>> shuts down the running machine Vagrant is managing. \lstinline!vagrant destroy! command stops the running machine Vagrant is managing and destroys all resources that were created during the machine creation process.

Tell vagrant to output the ssh connection details by doing:

\begin{lstlisting}[label=lst:my-cloud-vagrant-ssh-config]
$ vagrant ssh-config
Host default
  HostName 127.0.0.1
  User vagrant
  Port 2222
  UserKnownHostsFile /dev/null
  StrictHostKeyChecking no
  PasswordAuthentication no
  IdentityFile /../.vagrant/machines/default/virtualbox/private_key
  IdentitiesOnly yes
  LogLevel FATAL
\end{lstlisting}

Vagrant 1.7 changed how it handled private ssh keys. Starting with 1.7, Vagrant generates a new private key for each machine. Earlier versions used the same key, which was in the default location of \lstinline!~/.vagrant.d/insecure_private_key!. The examples in this book are from the newer version of Vagrant.

In most cases, your machine will not contain chef client. We should install it before using our kitchen. As you remember, we have knife with command \lstinline!prepare!. Let use this command:

\begin{lstlisting}[label=lst:my-cloud-vagrant5]
$ knife solo prepare vagrant@localhost -i ./.vagrant/machines/default/virtualbox/private_key -p 2222 -N web1.example.com
Bootstrapping Chef...
--2013-12-27 19:12:56--  https://www.opscode.com/chef/install.sh
Resolving www.opscode.com (www.opscode.com)... 184.106.28.91
...
Installing Chef 11.8.2
installing with dpkg...
Selecting previously unselected package chef.
(Reading database ... 51095 files and directories currently installed.)
Unpacking chef (from .../chef_11.8.2_amd64.deb) ...
Setting up chef (11.8.2-1.ubuntu.12.04) ...
Thank you for installing Chef!
\end{lstlisting}

We used some options for the \lstinline!knife prepare! command. \lstinline!-i! option specifies ssh key for machine, \lstinline!-N! option specifies node name (if it different from host name) and \lstinline!-p! option specifies SSH port. In most cases, this port is 2222, but if you running several machines from vagrant, it will be different. You can read what port is used for SSH by machine from output of command \lstinline!vagrant up!. All this credentials used to login by ssh on the node and install chef client.

Basically, to run our kitchen on node we are using \lstinline!knife solo cook! command:

\begin{lstlisting}[label=lst:my-cloud-vagrant11]
$ knife solo cook vagrant@localhost -i ./.vagrant/machines/default/virtualbox/private_key -p 2222 -N web1.example.com
Running Chef on localhost...
Checking Chef version...
Installing Berkshelf cookbooks to 'cookbooks'...
Using apache2 (1.7.0)
Uploading the kitchen...
...
 * service[apache2] action start (up to date)
Chef Client finished, 1 resources updated
\end{lstlisting}

But in Vagrant we can use build in command \lstinline!vagrant provision!. This command runs any configured provisioners against the running Vagrant managed machine.

Vagrantfile inside use ruby syntax, so we can use Ruby to \href{http://en.wikipedia.org/wiki/Dont\_repeat\_yourself}{DRY} our config. We should install chef gem inside vagrant:

\begin{lstlisting}[label=lst:my-cloud-vagrant6]
$ vagrant plugin install chef
Installing the 'chef' plugin. This can take a few minutes...
Installed the plugin 'chef (11.8.2)'!
\end{lstlisting}

After this we can use chef gem inside Vagrantfile config:

\begin{lstlisting}[language=Ruby,label=lst:my-cloud-vagrant7,title=my-cloud/nodes/Vagrantfile]
# -*- mode: ruby -*-
# vi: set ft=ruby :

require 'chef'
require 'json'

Chef::Config.from_file(File.join(File.dirname(__FILE__), '.chef', 'knife.rb'))
vagrant_json = JSON.parse(Pathname(__FILE__).dirname.join('nodes', (ENV['NODE'] || 'web1.example.com.json')).read)

# Vagrantfile API/syntax version. Don't touch unless you know what you're doing!
VAGRANTFILE_API_VERSION = "2"

Vagrant.configure(VAGRANTFILE_API_VERSION) do |config|
  config.vm.box = "precise64"

  config.vm.provision :chef_solo do |chef|
    chef.cookbooks_path = Chef::Config[:cookbook_path]
    chef.roles_path = Chef::Config[:role_path]
    chef.data_bags_path = Chef::Config[:data_bag_path]

    chef.environments_path = Chef::Config[:environment_path]
    #chef.environment = ENV['ENVIRONMENT'] || 'development'

    chef.run_list = vagrant_json.delete('run_list')
    chef.json = vagrant_json
  end
end
\end{lstlisting}

Now consider that we have added.

On lines 4-5 required chef and json gem. JSON gem is as part of Vagrant and chef gem was installed by previous command \lstinline!vagrant plugin install chef!. Further, we load knife.rb file in chef config and parse json from node file <<web1.example.com.json>>. After these manipulations we will have \lstinline!Chef::Config! ruby hash with knife configuration and \lstinline!vagrant_json! ruby hash with attributes from node.

On lines 16-26 defined Chef Solo configuration for Vagrant (\lstinline!environment! is commented, because we don't have it right now, but we will use it later). More information about this you can find in \href{http://docs.vagrantup.com/v2/provisioning/chef\_solo.html}{vagrant website}.

Because we changed the Vagrantfile configuration, we need to restart the test node by using \lstinline!vagrant reload! command:

\begin{lstlisting}[label=lst:my-cloud-vagrant8]
$ vagrant reload
[default] Attempting graceful shutdown of VM...
...
[default] -- /vagrant
[default] -- /tmp/vagrant-chef-1/chef-solo-3/roles
[default] -- /tmp/vagrant-chef-1/chef-solo-2/cookbooks
[default] -- /tmp/vagrant-chef-1/chef-solo-1/cookbooks
[default] -- /tmp/vagrant-chef-1/chef-solo-4/data_bags
[default] -- /tmp/vagrant-chef-1/chef-solo-5/environments
$ vagrant provision
[default] Running provisioner: chef_solo...
Generating chef JSON and uploading...
Running chef-solo...
stdin: is not a tty
INFO: Forking chef instance to converge...
INFO: *** Chef 11.8.2 ***
...
INFO: service[apache2] restarted
INFO: Chef Run complete in 18.115479422 seconds
INFO: Running report handlers
INFO: Report handlers complete
\end{lstlisting}

To verify that the apache2 successfully installed into node, we can forward the 80 apache2 port to our machine. To do this, we modify the Vagrantfile:

\begin{lstlisting}[language=Ruby, label=lst:my-cloud-vagrant9,title=my-cloud/nodes/Vagrantfile]
...

Vagrant.configure(VAGRANTFILE_API_VERSION) do |config|
  config.vm.box = "precise64"
  config.vm.network :forwarded_port, guest: 80, host: 8085 # <== add port forwarding

...
\end{lstlisting}

And again reload node:

\begin{lstlisting}[label=lst:my-cloud-vagrant10]
$ vagrant reload
...
[default] Preparing network interfaces based on configuration...
[default] Forwarding ports...
[default] -- 22 => 2222 (adapter 1)
[default] -- 80 => 8085 (adapter 1)
...
\end{lstlisting}

Now in any of your browser you can open url \href{http://localhost:8085/}{http://localhost:8085/} and see 404 page from apache2 server (because we only install it).
