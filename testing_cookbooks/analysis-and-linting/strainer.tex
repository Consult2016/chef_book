\subsection{Strainer}

\href{https://github.com/customink/strainer}{Strainer} is a gem for isolating and testing individual chef cookbooks. It allows you to keep all your cookbooks in a single repository (saving you time and money), while having all the benefits of individual repositories. But also you can use Strainer in <<standalone>> mode. This allows you to use Strainer file from within a cookbook.

First, add it in Gemfile:

\begin{lstlisting}[label=lst:testing-strainer1]
source 'https://rubygems.org'

gem 'berkshelf'
gem 'foodcritic'
gem 'thor-foodcritic'
gem 'chefspec'
gem 'test-kitchen'
gem 'chef-zero'

group :integration do
  gem 'kitchen-vagrant'
  gem 'cucumber'
  gem 'rspec-expectations'
  gem 'leibniz', '~> 0.2.1'
end

gem 'rubocop', require: false
gem 'strainer', require: false
\end{lstlisting}

And run <<bundle>> command to install it.

Next you must create file Strainerfile with such content:

\begin{lstlisting}[label=lst:testing-strainer2]
chefspec:   bundle exec rspec --color
kitchen:    bundle exec thor kitchen:default-ubuntu-1204
\end{lstlisting}

Strainer have similar functionality as \href{http://ddollar.github.io/foreman/}{Foreman}, but for running tests inside cookbook(s). Now we cat test our cookbook:

\begin{lstlisting}[language=Bash,label=lst:testing-strainer3]
$ bundle exec strainer test
# Straining 'my_cool_app (v0.1.0)'
chefspec             | bundle exec rspec --color
chefspec             | ..................
chefspec             | Finished in 1 minute 22.43 seconds
chefspec             | 23 examples, 0 failures
chefspec             | SUCCESS!
kitchen              | bundle exec thor kitchen:default-ubuntu-1204
kitchen              | -----> Cleaning up any prior instances of <default-ubuntu-1204>
kitchen              | -----> Destroying <default-ubuntu-1204>...

...

kitchen              |        Vagrant instance <default-ubuntu-1204> destroyed.
kitchen              |        Finished destroying <default-ubuntu-1204> (0m6.87s).
kitchen              |
kitchen              | SUCCESS!
Strainer marked build OK
\end{lstlisting}

Also you can use it with Rake or Thor. And integrate with CI.