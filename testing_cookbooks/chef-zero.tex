\section{Chef Zero}
\label{sec:testing-chef-zero}

Chef Zero is a simple, easy-install, in-memory Chef server that can be useful for Chef Client testing and chef-solo-like tasks that require a full Chef Server. It IS intended to be simple, Chef 11 compliant, easy to run and fast to start. It is NOT intended to be secure, scalable, performant or persistent. It does NO input validation, authentication or authorization (it will not throw a 400, 401 or 403). It does not save data, and will start up empty each time you start it.

Because Chef Zero runs in memory, it's super fast and lightweight. This makes it perfect for testing against a <<real>> Chef Server without mocking the entire Internet.

\subsection{Installing Chef Zero}

Let's cover our cookbook by using Chef Zero. First we should add this gem in Gemfile:

\begin{lstlisting}[label=lst:testing-chef-zero1]
source 'https://rubygems.org'

gem 'berkshelf'
gem 'foodcritic'
gem 'thor-foodcritic'
gem 'chefspec'
gem 'test-kitchen'
gem 'kitchen-vagrant'
gem 'chef-zero'
\end{lstlisting}

And you should to execute \inline{bundle} command to install this gem. We can check what Chef Zero installed:

\begin{lstlisting}[language=Bash,label=lst:testing-chef-zero2]
$ chef-zero
>> Starting Chef Zero (v1.7.3)...
>> Puma (v1.6.3) is listening at http://127.0.0.1:8889
>> Press CTRL+C to stop
^C
>> Stopping Chef Zero ...
\end{lstlisting}

Command \inline{chef-zero} start Chef Zero server in foreground. This is fully functional (empty) Chef Server. To use it in your own repository, create a \inline{knife.rb} like so:

\begin{lstlisting}[label=lst:testing-chef-zero3]
chef_server_url   'http://127.0.0.1:8889'
node_name         'stickywicket'
client_key        'path_to_any_pem_file.pem'
\end{lstlisting}

And use knife like you normally would.

Since Chef Zero does no authentication, any \inline{.pem} file will do. The client just needs something to sign requests with (which will be ignored on the server). Even though it's ignored, the \inline{.pem} must still be a valid format.

If you will stop the Chef Zero server than all the data is gone.

Run \inline{chef-zero --help} to see a list of the supported flags and options:

\begin{lstlisting}[language=Bash,label=lst:testing-chef-zero4]
$ chef-zero --help
Usage: chef-zero [ARGS]
    -H, --host HOST                  Host to bind to (default: 127.0.0.1)
    -p, --port PORT                  Port to listen on
        --socket PATH                Unix socket path to listen on
        --[no-]generate-keys         Whether to generate actual keys or fake it (faster).  Default: false.
    -d, --daemon                     Run as a daemon process
    -l, --log-level LEVEL            Set the output log level
    -h, --help                       Show this message
        --version                    Show version
\end{lstlisting}


\subsection{Using with ChefSpec}

By default, ChefSpec runs in Chef Solo mode, but you can ask ChefSpec to create an in-memory Chef Server during testing using ChefZero. This is especially helpful if you need to support searching or data bags.

To use the ChefSpec server, simply require the module in your \inline{spec_helper}:

\begin{lstlisting}[label=lst:testing-chef-zero5]
# spec_helper.rb
require 'chefspec'
require 'chefspec/server'
\end{lstlisting}

This will automatically create a Chef server, synchronize all the cookbooks in your \inline{cookbook_path}, and wire all the internals of Chef together. Recipe calls to \inline{search}, \inline{data_bag} and \inline{data_bag_item} will now query the ChefSpec server.

The ChefSpec server includes a collection of helpful DSL methods for populating data into the Chef Server.

Create a client:

\begin{lstlisting}[label=lst:testing-chef-zero6]
ChefSpec::Server.create_client('my_client', { admin: true })
\end{lstlisting}

Create a data bag (and items):

\begin{lstlisting}[label=lst:testing-chef-zero7]
ChefSpec::Server.create_data_bag('my_data_bag', {
  'item_1' => {
    'password' => 'abc123'
  },
  'item_2' => {
    'password' => 'def456'
  }
})
\end{lstlisting}

Create an environment:

\begin{lstlisting}[label=lst:testing-chef-zero8]
ChefSpec::Server.create_environment('my_environment', { description: '...' })
\end{lstlisting}

Create a node:

\begin{lstlisting}[label=lst:testing-chef-zero9]
ChefSpec::Server.create_node('my_node', { run_list: ['...'] })
\end{lstlisting}

You may also be interested in the \inline{stub_node} macro, which will create a new \inline{Chef::Node} object and accepts the same parameters as the Chef Runner and a Fauxhai object:

\begin{lstlisting}[label=lst:testing-chef-zero10]
www = stub_node(platform: 'ubuntu', version: '12.04') do |node|
  node.set['attribute'] = 'value'
end

# `www` is now a local Chef::Node object you can use in your test. To push this
# node to the server, call `create_node`:

ChefSpec::Server.create_node(www)
\end{lstlisting}

Create a role:

\begin{lstlisting}[label=lst:testing-chef-zero11]
ChefSpec::Server.create_role('my_role', { default_attributes: {} })

# The role now exists on the Chef Server, you can add it to a node's run_list
# by adding it to the `converge` block:
let(:chef_run) { ChefSpec::Runner.new.converge(described_recipe, 'role[my_role]') }
\end{lstlisting}

\subsubsection{Example}

Let's create recipe \inline{haproxy} in our cookbook:




\subsection{Using with Test Kitchen}
