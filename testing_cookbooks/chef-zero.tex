\section{Chef Zero}
\label{sec:testing-chef-zero}

Chef Zero is a simple, easy-install, in-memory Chef server that can be useful for Chef Client testing and chef-solo-like tasks that require a full Chef Server. It IS intended to be simple, Chef 11 compliant, easy to run and fast to start. It is NOT intended to be secure, scalable, performant or persistent. It does NO input validation, authentication or authorization (it will not throw a 400, 401 or 403). It does not save data, and will start up empty each time you start it.

Because Chef Zero runs in memory, it's super fast and lightweight. This makes it perfect for testing against a <<real>> Chef Server without mocking the entire Internet.

\subsection{Installing}

Let's cover our cookbook by using Chef Zero. First we should add this gem in Gemfile:

\begin{lstlisting}[label=lst:testing-chef-zero1]
source 'https://rubygems.org'

gem 'berkshelf'
gem 'foodcritic'
gem 'thor-foodcritic'
gem 'chefspec'
gem 'test-kitchen'
gem 'kitchen-vagrant'
gem 'chef-zero'
\end{lstlisting}

And you should to execute \inline{bundle} command to install this gem. We can check what Chef Zero installed:

\begin{lstlisting}[language=Bash,label=lst:testing-chef-zero2]
$ chef-zero
>> Starting Chef Zero (v1.7.3)...
>> Puma (v1.6.3) is listening at http://127.0.0.1:8889
>> Press CTRL+C to stop
^C
>> Stopping Chef Zero ...
\end{lstlisting}

Command \inline{chef-zero} start Chef Zero server in foreground. This is fully functional (empty) Chef Server. To use it in your own repository, create a \inline{knife.rb} like so:

\begin{lstlisting}[label=lst:testing-chef-zero3]
chef_server_url   'http://127.0.0.1:8889'
node_name         'stickywicket'
client_key        'path_to_any_pem_file.pem'
\end{lstlisting}

And use knife like you normally would.

Since Chef Zero does no authentication, any \inline{.pem} file will do. The client just needs something to sign requests with (which will be ignored on the server). Even though it's ignored, the \inline{.pem} must still be a valid format.

If you will stop the Chef Zero server than all the data is gone.

Run \inline{chef-zero --help} to see a list of the supported flags and options:

\begin{lstlisting}[language=Bash,label=lst:testing-chef-zero4]
$ chef-zero --help
Usage: chef-zero [ARGS]
    -H, --host HOST                  Host to bind to (default: 127.0.0.1)
    -p, --port PORT                  Port to listen on
        --socket PATH                Unix socket path to listen on
        --[no-]generate-keys         Whether to generate actual keys or fake it (faster).  Default: false.
    -d, --daemon                     Run as a daemon process
    -l, --log-level LEVEL            Set the output log level
    -h, --help                       Show this message
        --version                    Show version
\end{lstlisting}


\subsection{Using with ChefSpec}

By default, ChefSpec runs in Chef Solo mode, but you can ask ChefSpec to create an in-memory Chef Server during testing using ChefZero. This is especially helpful if you need to support searching or data bags.

To use the ChefSpec server, simply require the module in your \inline{spec_helper}:

\begin{lstlisting}[label=lst:testing-chef-zero5]
# spec_helper.rb
require 'chefspec'
require 'chefspec/server'
\end{lstlisting}

This will automatically create a Chef server, synchronize all the cookbooks in your \inline{cookbook_path}, and wire all the internals of Chef together. Recipe calls to \inline{search}, \inline{data_bag} and \inline{data_bag_item} will now query the ChefSpec server.

The ChefSpec server includes a collection of helpful DSL methods for populating data into the Chef Server.

Create a client:

\begin{lstlisting}[label=lst:testing-chef-zero6]
ChefSpec::Server.create_client('my_client', { admin: true })
\end{lstlisting}

Create a data bag (and items):

\begin{lstlisting}[label=lst:testing-chef-zero7]
ChefSpec::Server.create_data_bag('my_data_bag', {
  'item_1' => {
    'password' => 'abc123'
  },
  'item_2' => {
    'password' => 'def456'
  }
})
\end{lstlisting}

Create an environment:

\begin{lstlisting}[label=lst:testing-chef-zero8]
ChefSpec::Server.create_environment('my_environment', { description: '...' })
\end{lstlisting}

Create a node:

\begin{lstlisting}[label=lst:testing-chef-zero9]
ChefSpec::Server.create_node('my_node', { run_list: ['...'] })
\end{lstlisting}

You may also be interested in the \inline{stub_node} macro, which will create a new \inline{Chef::Node} object and accepts the same parameters as the Chef Runner and a Fauxhai object:

\begin{lstlisting}[label=lst:testing-chef-zero10]
www = stub_node(platform: 'ubuntu', version: '12.04') do |node|
  node.set['attribute'] = 'value'
end

# `www` is now a local Chef::Node object you can use in your test. To push this
# node to the server, call `create_node`:

ChefSpec::Server.create_node(www)
\end{lstlisting}

Create a role:

\begin{lstlisting}[label=lst:testing-chef-zero11]
ChefSpec::Server.create_role('my_role', { default_attributes: {} })

# The role now exists on the Chef Server, you can add it to a node's run_list
# by adding it to the `converge` block:
let(:chef_run) { ChefSpec::Runner.new.converge(described_recipe, 'role[my_role]') }
\end{lstlisting}

\subsubsection{Example}

Let's create recipe \inline{haproxy} in our cookbook:

\begin{lstlisting}[label=lst:testing-chef-zero12]
# install and setup haproxy

package "haproxy"

# get nodes of some roles
# get nodes of some roles
if Chef::Config[:solo]
  Chef::Log.warn("This recipe uses search. Chef Solo does not support search.")
  pool_members = []
else
  pool_members = search("node", "role:#{node['my_cool_app']['haproxy']['app_server_role']} AND chef_environment:#{node.chef_environment}") || []
end

pool_members.map! do |member|
  {:ipaddress => member['ipaddress'], :hostname => member['hostname']}
end

if pool_members.length > 0

  http_clients = pool_members.uniq.map do |s|
    "server #{s[:hostname]} #{s[:ipaddress]}:80 weight 1 maxconn 1024 check"
  end
  http_clients = ["mode http"] + http_clients + ["option httpchk GET /healthcheck"]

  https_clients = pool_members.uniq.map do |s|
    "server #{s[:hostname]} #{s[:ipaddress]}:443 weight 1 maxconn 1024 check"
  end
  https_clients = ["mode http"] + https_clients + ["option httpchk GET /ssl-healthcheck"]

else

  http_clients = https_clients = []

end

listeners = {
  "listen" => {},
  "frontend" => {
    "ft_http" => [
      "bind *:80",
      "mode tcp",
      "default_backend bk_http"
    ],
    "ft_https" => [
      "bind *:443",
      "mode tcp",
      "default_backend bk_https"
    ]
  },
  "backend" => {
    "bk_http" => http_clients,
    "bk_https" => https_clients
  }
}

template "/etc/haproxy/haproxy.cfg" do
  source "haproxy.cfg.erb"
  owner "root"
  group "root"
  mode 00644
  notifies :reload, "service[haproxy]"
  variables(
    :listeners => listeners
  )
end

cookbook_file '/etc/default/haproxy' do
  source 'haproxy-default'
  owner 'root'
  group 'root'
  mode 00644
  notifies :restart, 'service[haproxy]'
end

service "haproxy" do
  supports :restart => true, :status => true, :reload => true
  action [:enable, :start]
end
\end{lstlisting}

And default attributes for it:

\begin{lstlisting}[label=lst:testing-chef-zero13]
default['my_cool_app']['haproxy']['app_server_role']       = 'web'
\end{lstlisting}

In our recipe used template \inline{haproxy.cfg.erb}:

\begin{lstlisting}[label=lst:testing-chef-zero14]
global
  log 127.0.0.1   local0
  log 127.0.0.1   local1 notice
  maxconn 5000
  user haproxy
  group haproxy

defaults
  log     global
  mode    http
  retries 3
  timeout client 60000
  timeout server 50000
  timeout connect 25000
  balance roundrobin

<% @listeners.each do |type, listeners | %>
<% listeners.each do |name, listen| %>
<%= type %> <%= name %>
<% listen.each do |option| %>
  <%= option %>
<% end %>
<% end %>
<% end %>
\end{lstlisting}

As you can see, this recipe install and configure haproxy. It using \inline{search} command to get all nodes with role \inline{web} (by default attributes) and chef environment. After this it collect from all selected nodes hostname and ip address. This data is used to create config for haproxy. This is very usefull way, because if you will add or remove node with role \inline{web} from Chef server, it will automatically update haproxy config. But because we using \inline{search} command, we need test our cookbook with Chef Zero. Here how it look our tests:

\begin{lstlisting}[label=lst:testing-chef-zero15]
require 'spec_helper'

describe 'my_cool_app::haproxy' do
  let(:platfom) { 'ubuntu' }
  let(:platfom_version) { '12.04' }
  let(:chef_run) { ChefSpec::Runner.new(platform: platfom, version: platfom_version).converge(described_recipe) }

  it "install haproxy package" do
    expect(chef_run).to install_package('haproxy')
  end

  it 'enable haproxy service' do
    expect(chef_run).to enable_service('haproxy')
  end

  it 'create config /etc/haproxy/haproxy.cfg with empty backends' do
    expect(chef_run).to render_file('/etc/haproxy/haproxy.cfg').with_content(/#{Regexp.quote("backend bk_http\nbackend bk_https\n")}/)
  end

  context 'with env, role and one node' do
    let(:node_env) { 'test' }
    let(:chef_run) do
      ChefSpec::Runner.new(platform: platfom, version: platfom_version) do |node|
        # Create a new environment (you could also use a different :let block or :before block)
        env = Chef::Environment.new
        env.name node_env

        # Stub the node to return this environment
        node.stub(:chef_environment).and_return(env.name)

        # Stub any calls to Environment.load to return this environment
        Chef::Environment.stub(:load).and_return(env)
      end.converge(described_recipe)
    end

    before do
      ChefSpec::Server.create_environment(node_env, { description: 'Test env' })
      ChefSpec::Server.create_role('web', { default_attributes: {} })
      ChefSpec::Server.create_node('first-node', {
        run_list: ['role[web]'],
        chef_environment: node_env,
        normal: { fqdn: '127.0.0.1', hostname: 'test.org', ipaddress: '127.0.0.1' }
      })
    end

    it 'create config /etc/haproxy/haproxy.cfg with first-node on 80 port' do
      expect(chef_run).to render_file('/etc/haproxy/haproxy.cfg').with_content(/#{Regexp.quote('server test.org 127.0.0.1:80 weight 1 maxconn 1024 check')}/)
    end

    it 'create config /etc/haproxy/haproxy.cfg with first-node on 443 port' do
      expect(chef_run).to render_file('/etc/haproxy/haproxy.cfg').with_content(/#{Regexp.quote('server test.org 127.0.0.1:443 weight 1 maxconn 1024 check')}/)
    end

    context 'with two nodes' do
      before do
        ChefSpec::Server.create_node('second-node', {
          run_list: ['role[web]'],
          chef_environment: node_env,
          normal: { fqdn: '192.168.1.2', hostname: 'test2.org', ipaddress: '192.168.1.2' }
        })
      end

      it 'create config /etc/haproxy/haproxy.cfg with first-node on 80 port' do
        expect(chef_run).to render_file('/etc/haproxy/haproxy.cfg').with_content(/#{Regexp.quote('server test.org 127.0.0.1:80 weight 1 maxconn 1024 check')}/)
      end

      it 'create config /etc/haproxy/haproxy.cfg with first-node on 443 port' do
        expect(chef_run).to render_file('/etc/haproxy/haproxy.cfg').with_content(/#{Regexp.quote('server test.org 127.0.0.1:443 weight 1 maxconn 1024 check')}/)
      end

      it 'create config /etc/haproxy/haproxy.cfg with second-node on 80 port' do
        expect(chef_run).to render_file('/etc/haproxy/haproxy.cfg').with_content(/#{Regexp.quote('server test2.org 192.168.1.2:80 weight 1 maxconn 1024 check')}/)
      end

      it 'create config /etc/haproxy/haproxy.cfg with second-node on 443 port' do
        expect(chef_run).to render_file('/etc/haproxy/haproxy.cfg').with_content(/#{Regexp.quote('server test2.org 192.168.1.2:443 weight 1 maxconn 1024 check')}/)
      end

    end
  end

end
\end{lstlisting}

And we can check our tests:

\begin{lstlisting}[language=Bash,label=lst:testing-chef-zero16]
$ rspec spec/unit/recipes/haproxy_spec.rb
.........

Finished in 33.34 seconds
9 examples, 0 failures
\end{lstlisting}

All works.

\subsection{Using with Test Kitchen}

To use Chef Zero with test kitchen, you should change \inline{provisioner} type in \inline{.kitchen.yml}:

\begin{lstlisting}[label=lst:testing-chef-zero17]
---
driver:
  name: vagrant

provisioner:
  name: chef_zero
  roles_path: "test/chef-zero/roles"
  environments_path: "test/chef-zero/environments"
  nodes_path: "test/chef-zero/nodes"
  client_rb:
    environment: test

platforms:
  - name: ubuntu-12.04
  - name: ubuntu-10.04
  - name: centos-6.4

suites:
  - name: default
    run_list:
      - recipe[my_cool_app::default]
    attributes:
  - name: node
    run_list:
      - recipe[my_cool_app::default]
      - recipe[my_cool_app::node]
    attributes:
  - name: haproxy
    run_list:
      - recipe[my_cool_app::haproxy]
    attributes:
    excludes:
      - centos-6.4
\end{lstlisting}

As you can see, we changed \inline{provisioner name} to \inline{chef_zero}. Also we set \inline{roles_path}, \inline{environments_path} and \inline{nodes_path}. This folders will used to upload to Chef Zero test data - roles, environments and nodes. And we set \inline{client_rb} attribute, which allow add attributes for chef client. Here we set environment as <<test>>.

Next we create new test suite, called <<haproxy>>. Let's check it:

\begin{lstlisting}[language=Bash,label=lst:testing-chef-zero18]
$ kitchen list
Instance             Driver   Provisioner  Last Action
default-ubuntu-1204  Vagrant  ChefZero     <Not Created>
default-ubuntu-1004  Vagrant  ChefZero     <Not Created>
default-centos-64    Vagrant  ChefZero     <Not Created>
node-ubuntu-1204     Vagrant  ChefZero     <Not Created>
node-ubuntu-1004     Vagrant  ChefZero     <Not Created>
node-centos-64       Vagrant  ChefZero     <Not Created>
haproxy-ubuntu-1204  Vagrant  ChefZero     <Not Created>
haproxy-ubuntu-1004  Vagrant  ChefZero     <Not Created>
\end{lstlisting}

Now lets create folder <<integration/haproxy/serverspec>> and add to it tests for haproxy recipe:

\begin{lstlisting}[label=lst:testing-chef-zero19]
require 'serverspec'

include Serverspec::Helper::Exec
include Serverspec::Helper::DetectOS

RSpec.configure do |c|
  c.before :all do
    c.path = '/sbin:/usr/sbin'
  end
end

describe "Haproxy Daemon" do

  describe package('haproxy') do
    it { should be_installed }
  end

  describe service('haproxy') do
    it { should be_enabled   }
    it { should be_running   }
  end

  [80, 443].each do |port|
    it "is listening on port #{port}" do
      expect(port(port)).to be_listening
    end
  end

  describe file('/etc/haproxy/haproxy.cfg') do
    it { should be_file }
    its(:content) { should match /#{Regexp.quote('server leopard.in.ua 127.0.0.1:80 weight 1 maxconn 1024 check')}/ }
    its(:content) { should match /#{Regexp.quote('server leopard.in.ua 127.0.0.1:443 weight 1 maxconn 1024 check')}/ }
  end

end
\end{lstlisting}

Almoust ready. As you can see, we should generate by tests haproxy config with one node (hostname <<leopard.in.ua>> and ip address <<127.0.0.1>>). We should prepare this node for tests. In folder <<test/chef-zero/nodes>> we create node:

\begin{lstlisting}[label=lst:testing-chef-zero20]
{
  "name": "first-node",
  "json_class": "Chef::Node",
  "chef_type": "node",
  "chef_environment": "test",
  "normal": {
    "fqdn": "127.0.0.1",
    "hostname": "leopard.in.ua",
    "ipaddress": "127.0.0.1"
  },
  "run_list": ["role[web]"]
}
\end{lstlisting}

It should have <<test>> environment and use <<web>> role, because only by this conditions we will select this node for haproxy config. Also we set <<hostname>> and <<ipaddress>>, because this is not real node and Ohai will not fill this attributes. After this we should create <<test>> environment and <<web>> role:

\begin{lstlisting}[label=lst:testing-chef-zero21]
{
  "name": "test",
  "description": "test environment",
  "chef_type": "environment",
  "json_class": "Chef::Environment",
  "default_attributes": {}
}
\end{lstlisting}

\begin{lstlisting}[label=lst:testing-chef-zero22]
{
  "name": "web",
  "description": "The web role",
  "chef_type": "role",
  "json_class": "Chef::Role",
  "default_attributes": {
  },
  "run_list": []
}
\end{lstlisting}

All this data will load automatically into Chef Zero by test kitchen. And now we cat test our recipe by test kitchen:

\begin{lstlisting}[language=Bash,label=lst:testing-chef-zero23]
$ kitchen test haproxy-ubuntu-1204
-----> Starting Kitchen (v1.2.1)
-----> Cleaning up any prior instances of <haproxy-ubuntu-1204>
-----> Destroying <haproxy-ubuntu-1204>...
...
Uploading /tmp/busser/suites/serverspec/haproxy_spec.rb (mode=0644)
-----> Running serverspec test suite
/opt/chef/embedded/bin/ruby -I/tmp/busser/suites/serverspec -S /opt/chef/embedded/bin/rspec /tmp/busser/suites/serverspec/haproxy_spec.rb --color --format documentation

       Haproxy Daemon
  is listening on port 80
  is listening on port 443
  Package "haproxy"
    should be installed
  Service "haproxy"
    should be enabled
    should be running
  File "/etc/haproxy/haproxy.cfg"
    should be file
    content
      should match /server\ leopard\.in\.ua\ 127\.0\.0\.1:80\ weight\ 1\ maxconn\ 1024\ check/
    content
      should match /server\ leopard\.in\.ua\ 127\.0\.0\.1:443\ weight\ 1\ maxconn\ 1024\ check/

Finished in 0.55591 seconds
8 examples, 0 failures
       Finished verifying <haproxy-ubuntu-1204> (0m2.20s).
...
       Finished testing <haproxy-ubuntu-1204> (3m26.69s).
-----> Kitchen is finished. (3m28.05s)
\end{lstlisting}

As you can see, all works as expected.