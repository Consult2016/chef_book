\section{Cucumber}
\label{sec:testing-cucumber-spec}

\href{http://cukes.info/}{Cucumber} is a tool for running automated tests written in plain language. Because they're written in plain language, they can be read by anyone on your team. Because they can be read by anyone, you can use them to help improve communication, collaboration and trust on your team.

\subsection{Example}

Therefore we need only add the following three gems to a Gemfile:

\begin{lstlisting}[label=lst:testing-cucumber-spec1]
gem 'cucumber'
gem 'rspec-expectations'
gem 'leibniz'
\end{lstlisting}

And you should to execute \inline!bundle! command to install this gem.

Next we should create directory for our tests:

\begin{lstlisting}[language=Bash,label=lst:testing-cucumber-spec2]
$ mkdir -p features/step_definitions
$ mkdir -p features/step_definitions/support
\end{lstlisting}

And require needed libs in \inline!features/step_definitions/support/env.rb! file:

\begin{lstlisting}[language=Bash,label=lst:testing-cucumber-spec3]
require 'leibniz'
require 'faraday'
\end{lstlisting}

Now we should create file, which will contain tests definitions. Let's call it \inline!features/working_web_page.feature!:

\begin{lstlisting}[label=lst:testing-cucumber-spec4]
Feature: Customer can use my cool web app

  In order to get more payment customers
  As a business owner
  I want web users to be able use my cool web app

  Background:
    Given I have provisioned the following infrastructure:
    | Server Name | Operating System | Version | Chef Version | Run List              |
    | localhost   | ubuntu           | 12.04   | 11.4.4       | my_cool_app::default  |

    And I have run Chef

  Scenario: User visits home page
    Given a url "http://example.org"
    When a web user browses to the URL
    Then the user should see "This is my cool web app"
    And cleanup test env
\end{lstlisting}

Next, we can try run cucumber:

\begin{lstlisting}[language=Bash,label=lst:testing-cucumber-spec5]
$ cucumber
Feature: Customer can use my cool web app

  In order to get more payment customers
  As a business owner
  I want web users to be able use my cool web app

  Background:                                              # features/working_web_page.feature:7
    Given I have provisioned the following infrastructure: # features/working_web_page.feature:8
      | Server Name | Operating System | Version | Chef Version | Run List             |
      | localhost   | ubuntu           | 12.04   | 11.4.4       | my_cool_app::default |
    And I have run Chef                                    # features/working_web_page.feature:12

  Scenario: User visits home page                          # features/working_web_page.feature:14
    Given a url http://example.org                         # features/working_web_page.feature:15
    When a web user browses to the URL                     # features/working_web_page.feature:16
    Then the user should see "This is my cool web app"     # features/working_web_page.feature:17

1 scenario (1 undefined)
5 steps (5 undefined)
0m0.012s

You can implement step definitions for undefined steps with these snippets:

Given(/^I have provisioned the following infrastructure:$/) do |table|
  # table is a Cucumber::Ast::Table
  pending # express the regexp above with the code you wish you had
end

Given(/^I have run Chef$/) do
  pending # express the regexp above with the code you wish you had
end

Given(/^a url http:\/\/example\.org$/) do
  pending # express the regexp above with the code you wish you had
end

When(/^a web user browses to the URL$/) do
  pending # express the regexp above with the code you wish you had
end

Then(/^the user should see "(.*?)"$/) do |arg1|
  pending # express the regexp above with the code you wish you had
end

If you want snippets in a different programming language,
just make sure a file with the appropriate file extension
exists where cucumber looks for step definitions.
\end{lstlisting}

As you can see, we dont have any line of real tests. Let's add tests in file \inline!features/step_definitions/working_web_page.rb!:

\begin{lstlisting}[label=lst:testing-cucumber-spec6]
Given(/^I have provisioned the following infrastructure:$/) do |specification|
  @infrastructure = Leibniz.build(specification)
end

Given(/^I have run Chef$/) do
  @infrastructure.destroy
  @infrastructure.converge
end

Given(/^a url "(.*?)"$/) do |url|
  @host_header = url.split('/').last
end

When(/^a web user browses to the URL$/) do
  connection = Faraday.new(:url => "http://#{@infrastructure['localhost'].ip}", :headers => {'Host' => @host_header}) do |faraday|
    faraday.adapter Faraday.default_adapter
  end
  @page = connection.get('/').body
end

Then(/^the user should see "(.*?)"$/) do |content|
  expect(@page).to match /#{content}/
end

Then(/^cleanup test env$/) do
  @infrastructure.destroy if @infrastructure
end
\end{lstlisting}

Leibniz gem read infrastructure configuration from our specs inside <<Background>> and use Test Kitchen to create it. Next, in <<I have run Chef>> we cleanup old and create new node, install chef client and run it inside node. After this we using Faraday gem to do HTTP request inside node and get root page content. We are checking, what this content should contain <<This is my cool web app>>. In this case we will be sure, what nginx installed, running and server our web page. In the end we added <<cleanup test env>>, which would remove node after tests.

Finally, we cat test our cucumber tests:

\begin{lstlisting}[language=Bash,label=lst:testing-cucumber-spec7]
$ cucumber
Feature: Customer can use my cool web app

  In order to get more payment customers
  As a business owner
  I want web users to be able use my cool web app

  Background:                                              # features/working_web_page.feature:7
    Given I have provisioned the following infrastructure: # features/step_definitions/working_web_page.rb:1
      | Server Name | Operating System | Version | Chef Version | Run List             |
      | localhost   | ubuntu           | 12.04   | 11.4.4       | my_cool_app::default |
-----> Destroying <leibniz-localhost>...
       ==> default: Forcing shutdown of VM...
       ==> default: Destroying VM and associated drives...
...

Chef Client finished, 28 resources updated
       Finished converging <leibniz-localhost> (1m6.98s).
    And I have run Chef                                    # features/step_definitions/working_web_page.rb:5

  Scenario: User visits home page                          # features/working_web_page.feature:14
    Given a url "http://example.org"                       # features/step_definitions/working_web_page.rb:10
    When a web user browses to the URL                     # features/step_definitions/working_web_page.rb:14
    Then the user should see "This is my cool web app"     # features/step_definitions/working_web_page.rb:21
-----> Destroying <leibniz-localhost>...
       ==> default: Forcing shutdown of VM...
       ==> default: Destroying VM and associated drives...
       Vagrant instance <leibniz-localhost> destroyed.
       Finished destroying <leibniz-localhost> (0m7.15s).
    And cleanup test env                                   # features/step_definitions/working_web_page.rb:25

1 scenario (1 passed)
6 steps (6 passed)
2m41.004s
\end{lstlisting}

As a result, we cover our default recipe by using cucumber.
