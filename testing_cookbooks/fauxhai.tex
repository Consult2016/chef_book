\section{Fauxhai}
\label{sec:testing-fauxhai}

Ohai is a tool that is used to detect attributes on a node, and then provide these attributes to the chef-client at the start of every chef-client run. Ohai is required by the chef-client and must be present on a node. It's awesome, but this can be problem for testing. What is why exist Fauxhai. \href{http://technology.customink.com/fauxhai/}{Fauxhai} is a gem for mocking out ohai data in your chef testing.

\subsection{Testing}

As you can see from our node specs, we can set \inline!platform! and \inline!version!, but we can stub additional Ohai attribute by using Fauxhai. Let's look at example:

\begin{lstlisting}[label=lst:testing-fauxhai1]
require 'chefspec'

describe 'my_cool_app::default' do
  [1, 2, 4].each do |kernels|
    context "on Ubuntu with #{kernels} kernels" do
      let(:chef_run) { ChefSpec::ChefRunner.new.converge(described_recipe) }

      before do
        Fauxhai.mock(platform: 'ubuntu', version: '12.04', fqdn: 'example.com', cpu: { 'real' => kernels, 'total' => kernels })
      end

      it 'install htop' do
        expect(chef_run).to install_package('htop')
      end

      it 'create file /tmp/cpu_count' do
        expect(chef_run).to render_file('/tmp/cpu_count').with_content(kernels.to_s)
      end
    end
  end
end
\end{lstlisting}

As you can see from example, we mock platform, fqdn and cpu numbers from Ohai attributes. And in our recipe we create \inline!/tmp/cpu_count! file, which contain number of cpu on node. By tests we check what this works on different values of Ohai attributes.

Also we can set node attributes by \inline!ChefSpec::Runner!:

\begin{lstlisting}[label=lst:testing-fauxhai2,title=my-server-cloud/site-cookbooks/my\_cool\_app/spec/unit/recipes/node\_spec.rb]
...

  context 'node versions' do
    let(:system_node_version) { nil }
    let(:chef_run) do
      ChefSpec::Runner.new(platform: platfom, version: platfom_version) do |node|
        node.automatic['system_node_js'] = { 'version' => system_node_version } if system_node_version
      end.converge(described_recipe)
    end

    it 'install node if version not specified' do
      expect(chef_run).to run_execute('make install').with(cwd: "/usr/local/src/node-v#{nodejs_version}")
    end

    context 'installed different version' do
      let(:system_node_version) { '0.8.0' }

      it 'install node if version is not the same' do
        expect(chef_run).to run_execute('make install').with(cwd: "/usr/local/src/node-v#{nodejs_version}")
      end
    end

    context 'installed same version' do
      let(:system_node_version) { nodejs_version }

      it 'do not install node' do
        expect(chef_run).not_to run_execute('make install').with(cwd: "/usr/local/src/node-v#{nodejs_version}")
      end
    end
  end

...
\end{lstlisting}

And check what this new tests is pass:

\begin{lstlisting}[language=Bash,label=lst:testing-fauxhai3]
$ rspec

...............

Finished in 3.76 seconds
15 examples, 0 failures
\end{lstlisting}

We covered different installation (or not installation) of node.js on node.