\section{Minitest}
\label{sec:testing-minitest}

\href{https://github.com/seattlerb/minitest}{Minitest} provides a complete suite of testing facilities supporting \href{http://en.wikipedia.org/wiki/Test-driven\_development}{TDD}, \href{http://en.wikipedia.org/wiki/Behavior-driven\_development}{BDD}, mocking, and benchmarking. Minitest doesn't reinvent anything that ruby already provides, like: classes, modules, inheritance, methods. This means you only have to learn ruby to use minitest and all of your regular OO practices like extract-method refactorings still apply.

Exists two ways of minitest usage:

\begin{itemize}
  \item With Test Kitchen (like bats or serverspec)
  \item Testing node after cooking by minitest-chef-handler
\end{itemize}

Second way is very interesting, because allows you to check tests even on production environment. This allows to be sure, that tests pass on check run of chef client on node.

Let's consider each example.

\subsection{Test Kitchen}

As you remember, we used bats and serverspec with Test Kitchen. But inside it you can use any test framework, for which exists <<busser>>. Right now you can find:

\begin{itemize}
  \item busser-minitest
  \item busser-bats
  \item busser-bash
  \item busser-serverspec
  \item busser-rspec
  \item busser-cucumber
\end{itemize}

Let's add little test inside our cookbook:

\begin{lstlisting}[label=lst:testing-minitest1]
require 'minitest/autorun'

describe "my_cool_app::default" do

  it "has created /var/www/my_cool_app/index.html" do
    assert File.exists?("/var/www/my_cool_app/index.html")
  end

end
\end{lstlisting}

And check how it works:

\begin{lstlisting}[language=Bash,label=lst:testing-minitest2]
$ kitchen test default-ubuntu-1204
-----> Starting Kitchen (v1.2.1)
-----> Cleaning up any prior instances of <default-ubuntu-1204>
-----> Destroying <default-ubuntu-1204>...
...
       Plugin minitest installed (version 0.2.0)
-----> Running postinstall for minitest plugin
...
-----> Running minitest test suite
/opt/chef/embedded/bin/ruby  -I"/opt/chef/embedded/lib/ruby/1.9.1" "/opt/chef/embedded/lib/ruby/1.9.1/rake/rake_test_loader.rb" "/tmp/busser/suites/minitest/test_default.rb"
Run options: --seed 59554

# Running tests:

.

Finished tests in 0.002407s, 415.5347 tests/s, 415.5347 assertions/s.

1 tests, 1 assertions, 0 failures, 0 errors, 0 skips
       Finished verifying <default-ubuntu-1204> (0m2.39s).
-----> Kitchen is finished. (1m27.97s)
\end{lstlisting}

As you can see, minitest doesn't have additional helpers, like have serverspec (\lstinline!be_listening!, \lstinline!be_running!, etc). What is why let's consider second way of using it.

\subsection{Minitest Chef Handler}

Minitest-chef-handler run minitest suites after your Chef recipes to check the status of your system. It can be very useful, because you testing inside your real environment. Exists 2 option to use it:

Option 1: Add the report handler to your client.rb or solo.rb file:

\begin{lstlisting}[label=lst:testing-minitest3]
require 'minitest-chef-handler'

report_handlers << MiniTest::Chef::Handler.new
\end{lstlisting}

Option 2: Using \href{https://github.com/btm/minitest-handler-cookbook}{minitest-handler} cookbook, which should be added in the end of \lstinline!run_list!:

\begin{lstlisting}[label=lst:testing-minitest4]
chef.run_list = [
  "your-recipes",
  "minitest-handler"
]
\end{lstlisting}

I prefer second variant. Just add in \lstinline!Berksfile! cookbook \lstinline!minitest-handler!:

\begin{lstlisting}[label=lst:testing-minitest5]
source "http://api.berkshelf.com"

metadata

cookbook "minitest-handler"
\end{lstlisting}

Add it in \lstinline!run_list!:

\begin{lstlisting}[label=lst:testing-minitest6]
suites:
  - name: default
    run_list:
      - recipe[my_cool_app::default]
      - recipe[minitest-handler]
\end{lstlisting}

Next we should write some tests. Let's create folder to its:

\begin{lstlisting}[language=Bash,label=lst:testing-minitest7]
$ mkdir -p files/default/test
\end{lstlisting}

And add some tests to cover default recipe:

\begin{lstlisting}[label=lst:testing-minitest8,title=my-server-cloud/site-cookbooks/my\_cool\_app/files/default/test/default\_test.rb]
require 'minitest/spec'

describe_recipe 'my_cool_app::default' do

  it "install ntp package" do
    package('ntp').must_be_installed
  end

  it "install git package" do
    if "ubuntu" == node['platform'] && node['platform_version'].to_f <= 10.04
      pack = "git-core"
    else
      pack = "git"
    end
    package(pack).must_be_installed
  end

  it "nginx must running" do
    service("nginx").must_be_running
  end

end
\end{lstlisting}

Finally run our chef-client by Test Kitchen <<converge>> command:

\begin{lstlisting}[label=lst:testing-minitest9]
$ kitchen converge default-ubuntu-1204
...
Running handlers:
[2014-03-27T20:30:03+00:00] INFO: Running report handlers
Run options: -v --seed 42106

# Running tests:

recipe::my_cool_app::default#test_0002_install git package = 0.14 s = .
recipe::my_cool_app::default#test_0001_install ntp package = 0.08 s = .
recipe::my_cool_app::default#test_0003_nginx must running = 0.09 s = .
\end{lstlisting}

As a result, after launching all recipes in \lstinline!run_list!, last recipe \lstinline!minitest-handler::default! run minitest to check status of the node.
