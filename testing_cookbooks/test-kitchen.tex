\section{Test Kitchen}
\label{sec:testing-test-kitchen}

\href{http://kitchen.ci/}{Test Kitchen} is a test harness tool to execute your configured code on one or more platforms in isolation. A driver plugin architecture is used which lets you run your code on various cloud providers and virtualization technologies such as Amazon EC2, Blue Box, CloudStack, Digital Ocean, Rackspace, OpenStack, Vagrant, Docker, LXC containers, and more. Many testing frameworks are already supported out of the box including Bats, shUnit2, RSpec, Serverspec, etc.

\subsection{Example}

\subsubsection{Installing Test Kitchen}

Let's cover our cookbook by test kitchen tests. First we should add this gem in Gemfile:

\begin{lstlisting}[label=lst:testing-test-kitchen1]
source 'https://rubygems.org'

gem 'berkshelf'
gem 'foodcritic'
gem 'thor-foodcritic'
gem 'chefspec'
gem 'test-kitchen'
gem 'kitchen-vagrant'
\end{lstlisting}

And you should to execute \inline{bundle} command to install this gem. We can check what kitchen installed:

\begin{lstlisting}[language=Bash,label=lst:testing-test-kitchen2]
$ kitchen version
Test Kitchen version 1.2.1
$ kitchen help
Commands:
  kitchen console                         # Kitchen Console!
  kitchen converge [INSTANCE|REGEXP|all]  # Converge one or more instances
  kitchen create [INSTANCE|REGEXP|all]    # Create one or more instances
  kitchen destroy [INSTANCE|REGEXP|all]   # Destroy one or more instances
  kitchen diagnose [INSTANCE|REGEXP|all]  # Show computed diagnostic configuration
  kitchen driver                          # Driver subcommands
  kitchen driver create [NAME]            # Create a new Kitchen Driver gem project
  kitchen driver discover                 # Discover Test Kitchen drivers published on RubyGems
  kitchen driver help [COMMAND]           # Describe subcommands or one specific subcommand
  kitchen help [COMMAND]                  # Describe available commands or one specific command
  kitchen init                            # Adds some configuration to your cookbook so Kitchen can rock
  kitchen list [INSTANCE|REGEXP|all]      # Lists one or more instances
  kitchen login INSTANCE|REGEXP           # Log in to one instance
  kitchen setup [INSTANCE|REGEXP|all]     # Setup one or more instances
  kitchen test [INSTANCE|REGEXP|all]      # Test one or more instances
  kitchen verify [INSTANCE|REGEXP|all]    # Verify one or more instances
  kitchen version                         # Print Kitchen's version information
\end{lstlisting}

Now we'll add Test Kitchen to our project by using the init subcommand:

\begin{lstlisting}[language=Bash,label=lst:testing-test-kitchen3]
$ kitchen init
    create  .kitchen.yml
    append  Thorfile
    create  test/integration/default
\end{lstlisting}

What's going on here? The \inline{kitchen init} subcommand will create an initial configuration file for Test Kitchen called \inline{.kitchen.yml}. We'll look at that in a moment.

A few directories were created but these are only a convenience–you don't strictly need <<test/integration/default>> in your project.

You can see that you have a \inline{.gitignore} file in your project's root which will tell Git to never commit a directory called \inline{.kitchen} and something called \inline{.kitchen.local.yml}. Don't worry about these for now, just some housekeeping details.

Finally, a gem called \inline{kitchen-vagrant} was installed. By itself Test Kitchen can't do very much. It needs one or more Drivers which are responsible for managing the virtual machines we need for testing. At present there are many different Test Kitchen Drivers but we're going to stick with the \href{https://github.com/opscode/kitchen-vagrant}{Kitchen Vagrant Driver} for now.

Let's turn our attention to the \inline{.kitchen.yml} file for a minute. While Test Kitchen may have created the initial file automatically, it's expected that you will read and edit this file. Opening this file in your editor of choice we see something like the following:

\begin{lstlisting}[label=lst:testing-test-kitchen4]
---
driver:
  name: vagrant

provisioner:
  name: chef_solo

platforms:
  - name: ubuntu-12.04
  - name: centos-6.4

suites:
  - name: default
    run_list:
      - recipe[my_cool_app::default]
    attributes:
\end{lstlisting}

Very briefly we can cover the 4 main sections you're likely to find in a \inline{.kitchen.yml} file:

\begin{itemize}
  \item \inline{driver}: This is where we configure the behaviour of the Kitchen Driver - the component that is responsible for creating a machine that we'll use to test our cookbook. Here we set up basics like credentials, ssh usernames, sudo requirements, etc. Each Driver is reponsible for requiring and using the configuration here. Under this section we have \inline{driver.name}: This tells Test Kitchen that we want to use the \inline{kitchen-vagrant} driver by default unless otherwise specified.
  \item \inline{provisioner}: This tells Test Kitchen how to run Chef, to apply the code in our cookbook to the machine under test. The default and simplest approach is to use \inline{chef-solo}, but other options are available, and ultimately Test Kitchen doesn't care how the infrastructure is built - it could theoretically be with Puppet, Ansible, or Perl for all it cares.
  \item \inline{platforms}: This is a list of operation systems on which we want to run our code. Note that the operating system's version, architecture, cloud environment, etc. might be relevant to what Test Kitchen considers a Platform.
  \item \inline{suites}: This section defines what we want to test. It includes the Chef run-list and any node attribute setups that we want run on each Platform above. For example, we might want to test the MySQL client cookbook code seperately from the server cookbook code for maximum isolation.
\end{itemize}

Let's say for argument's sake that we only care about running our Chef cookbook on Ubuntu 12.04 distributions. In that case, we can edit the \inline{.kitchen.yml} file so that the list of platforms has only one entry like so:

\begin{lstlisting}[label=lst:testing-test-kitchen5]
---
driver:
  name: vagrant

provisioner:
  name: chef_solo

platforms:
  - name: ubuntu-12.04

suites:
  - name: default
    run_list:
      - recipe[my_cool_app::default]
    attributes:
\end{lstlisting}

To see the results of our work, let's run the \inline{kitchen list} subcommand:

\begin{lstlisting}[language=Bash,label=lst:testing-test-kitchen6]
$ kitchen list
Instance             Driver   Provisioner  Last Action
default-ubuntu-1204  Vagrant  ChefSolo     <Not Created>
\end{lstlisting}

So what's this <<default-ubuntu-1204>> thing and what's an <<Instance>>? A Test Kitchen Instance is a pairwise combination of a Suite and a Platform as laid out in your \inline{.kitchen.yml} file. Test Kitchen has auto-named your only instance by combining the Suite name (<<default>>) and the Platform name (<<ubuntu-12.04>>) into a form that is safe for DNS and hostname records, namely <<default-ubuntu-1204>>.

Okay, let's spin this Instance up to see what happens. Test Kitchen calls this the Create Action. We're going to be painfully explicit and ask Test Kitchen to only create the <<default-ubuntu-1204>> instance:

\begin{lstlisting}[language=Bash,label=lst:testing-test-kitchen7]
$ kitchen create default-ubuntu-1204
-----> Starting Kitchen (v1.2.1)
-----> Creating <default-ubuntu-1204>...
       Bringing machine 'default' up with 'virtualbox' provider...
       [default] Importing base box 'opscode-ubuntu-12.04'...
       [default] Matching MAC address for NAT networking...
       [default] Setting the name of the VM...
       [default] Clearing any previously set forwarded ports...
       [default] Clearing any previously set network interfaces...
       [default] Preparing network interfaces based on configuration...
       [default] Forwarding ports...
       [default] -- 22 => 2222 (adapter 1)
       [default] Running 'pre-boot' VM customizations...
       [default] Booting VM...
       [default] Waiting for machine to boot. This may take a few minutes...
       [default] Machine booted and ready!
       [default] Setting hostname...
       Vagrant instance <default-ubuntu-1204> created.
       Finished creating <default-ubuntu-1204> (3m56.11s).
-----> Kitchen is finished. (3m57.66s)
\end{lstlisting}

If you are a Vagrant user then the line containing \inline{vagrant up --no-provision} will look familiar. Let's check the status of our instance now:

\begin{lstlisting}[language=Bash,label=lst:testing-test-kitchen8]
$ kitchen list
Instance             Driver   Provisioner  Last Action
default-ubuntu-1204  Vagrant  ChefSolo     Created
\end{lstlisting}





\subsubsection{Running Kitchen Converge}

Now let's let Test Kitchen run it for us on our Ubuntu 12.04 instance:

\begin{lstlisting}[language=Bash,label=lst:testing-test-kitchen9]
$ kitchen converge default-ubuntu-1204
-----> Starting Kitchen (v1.2.1)
-----> Converging <default-ubuntu-1204>...
       Preparing files for transfer
       Resolving cookbook dependencies with Berkshelf 2.0.14...
       Removing non-cookbook files before transfer
       Transfering files to <default-ubuntu-1204>
       ...
       Finished converging <default-ubuntu-1204> (10m24.13s).
-----> Kitchen is finished. (10m26.31s)
\end{lstlisting}

To quote our Chef run, that was too easy. If you are a Chef user then part of the output above should look familiar to you. Here's what happened at a high level:

\begin{itemize}
  \item Chef was installed on the instance by performing an \href{http://www.opscode.com/chef/install/}{Omnibus package installation}
  \item Your Git cookbook files and a minimal Chef Solo configuration were built and uploaded to the instance
  \item A Chef run was initiated using the run-list and node attributes specified in the \inline{.kitchen.yml} file
\end{itemize}

There's nothing to stop you from running this command again (or over-and-over for that matter) so, let's see what happens:

\begin{lstlisting}[language=Bash,label=lst:testing-test-kitchen10]
$ kitchen converge default-ubuntu-1204
-----> Starting Kitchen (v1.2.1)
-----> Converging <default-ubuntu-1204>...
       Preparing files for transfer
       Resolving cookbook dependencies with Berkshelf 2.0.14...
       Removing non-cookbook files before transfer
       Transfering files to <default-ubuntu-1204>
       ...
       Finished converging <default-ubuntu-1204> (0m13.13s).
-----> Kitchen is finished. (0m15.23s)
\end{lstlisting}

That ran a lot faster didn't it? Here's what happened this time:

\begin{itemize}
  \item Test Kitchen found that Chef was present and installed so skipped a re-installation.
  \item The same Chef cookbook files and Chef Solo configuration was uploaded to the instance. Test Kitchen is optimizing for freshness of code and configuration over speed. Although we all like speed wherever possible.
  \item A Chef run is initiated and runs very quickly as we are in the desired state.
\end{itemize}

Let's check the status of our instance:

\begin{lstlisting}[language=Bash,label=lst:testing-test-kitchen11]
$ kitchen list
Instance             Driver   Provisioner  Last Action
default-ubuntu-1204  Vagrant  ChefSolo     Converged
\end{lstlisting}

A clean converge run, success!





\subsubsection{Manually Verifying}

Test Kitchen has a \inline{login} subcommand for just these kinds of situations:

\begin{lstlisting}[language=Bash,label=lst:testing-test-kitchen12]
$ kitchen login default-ubuntu-1204
Welcome to Ubuntu 12.04.3 LTS (GNU/Linux 3.8.0-29-generic x86_64)

 * Documentation:  https://help.ubuntu.com/
Last login: Thu Mar  6 22:01:08 2014 from 10.0.2.2
vagrant@default-ubuntu-1204:~$ ps ax | grep nginx
 6151 ?        Ss     0:00 nginx: master process /usr/sbin/nginx
26282 ?        S      0:00 nginx: worker process
26758 pts/0    S+     0:00 grep --color=auto nginx
\end{lstlisting}

As you can see by the prompt above we are now in the <<default-ubuntu-1204>> instance and nginx installed inside it.




\subsubsection{Writing a Test}

Now if you are interested in writing some tests, Test Kitchen has several options available to you. The component that helps facilitate testing on your instances is called Busser. Just like Test Kitchen it is a RubyGem library and it provides a plugin system so that you can wire in whatever testing framework your heart desires. A quick \href{https://rubygems.org/search?utf8=%E2%9C%93&amp;query=busser-}{search} on the RubyGems website returns several testing frameworks currently available to you.

To keep things simple we're going to use the \inline{busser-bats} runner plugin which uses the \href{https://github.com/sstephenson/bats}{Bash Automated Testing System} also known as bats.

We need to put our test files in a specifc location, so let's create the directory:

\begin{lstlisting}[language=Bash,label=lst:testing-test-kitchen13]
$ mkdir -p test/integration/default/bats
\end{lstlisting}

It looks long and dense, but each directory has some meaning to Test Kitchen and the Busser helper:

\begin{itemize}
  \item \inline{test/integration}: Test Kitchen will look for tests to run under this directory. It allows you to put unit or other tests in test/unit, spec, acceptance, or wherever without mixing them up. This is configurable, if desired.
  \item \inline{default}: This corresponds exactly to the Suite name we set up in the .kitchen.yml file. If we had a suite called <<server-only>>, then you would put tests for the server only suite under test/integration/server-only.
  \item \inline{bats}: This tells Test Kitchen (and Busser) which Busser runner plugin needs to be installed on the remote instance. In other words the bats directory name will cause Busser to install busser-bats from RubyGems.
\end{itemize}

Let's write a test. Create a new file called \inline{test/integration/default/bats/nginx_installed.bats} with the following:

\begin{lstlisting}[language=Bash,label=lst:testing-test-kitchen14]
#!/usr/bin/env bats

@test "nginx binary is found in PATH" {
  run which nginx
  [ "$status" -eq 0 ]
}
\end{lstlisting}

Now to put our test to the test. For this we'll use the verify subcommand:

\begin{lstlisting}[language=Bash,label=lst:testing-test-kitchen15]
$ kitchen verify default-ubuntu-1204
-----> Starting Kitchen (v1.2.1)
-----> Setting up <default-ubuntu-1204>...
Fetching: thor-0.18.1.gem (100%)
Fetching: busser-0.6.0.gem (100%)
...
Uploading /tmp/busser/suites/bats/nginx_installed.bats (mode=0755)
-----> Running bats test suite
  nginx binary is found in PATH

       1 test, 0 failures
       Finished verifying <default-ubuntu-1204> (0m1.39s).
-----> Kitchen is finished. (0m16.51s)
\end{lstlisting}

All right!

A few things of note from the output above:

\begin{itemize}
  \item Notice the Setting up <<default-ubuntu-1204>> line? That's another action called the Setup Action. Usually not a big for most users but this action is responsible for installing the Busser RubyGem and any test runner plugins required. In our case the busser-bats RubyGem was installed which helped to install the bats testing framework.
  \item The Verifying <<default-ubuntu-1204>> line corresponds to the start of the Verify Action and the nginx binary is found in PATH line is output from a bats test run.
  \item The last command (\inline{echo \$?}) is a way to print the exit code of the last run shell command. This shows that the kitchen command exited cleanly.
\end{itemize}

Let's check the status of our instance again:

\begin{lstlisting}[language=Bash,label=lst:testing-test-kitchen16]
$ kitchen list
Instance             Driver   Provisioner  Last Action
default-ubuntu-1204  Vagrant  ChefSolo     Verified
\end{lstlisting}

Let's add more tests:

\begin{lstlisting}[label=lst:testing-test-kitchen17]
@test "nginx config is valid" {
  run nginx -t
  [ "$status" -eq 0 ]
}

@test "nginx is running" {
  run service nginx status
  [ "$status" -eq 0 ]
  [ "$output" == " * nginx is running" ]
}
\end{lstlisting}





\subsubsection{Running Kitchen Test}

Now it's time to introduce to the \inline{test} meta-action which helps you automate all the previous actions so far into one command. Recall that we currently have our instance in a <<verified>> state. With this in mind, let's run kitchen test:

\begin{lstlisting}[language=Bash,label=lst:testing-test-kitchen18]
$ kitchen test default-ubuntu-1204
-----> Starting Kitchen (v1.2.1)
-----> Cleaning up any prior instances of <default-ubuntu-1204>
-----> Destroying <default-ubuntu-1204>...
       Finished destroying <default-ubuntu-1204> (0m0.00s).
-----> Testing <default-ubuntu-1204>
...
Uploading /tmp/busser/suites/bats/nginx_installed.bats (mode=0755)
-----> Running bats test suite
  nginx binary is found in PATH

       1 test, 0 failures
       Finished verifying <default-ubuntu-1204> (0m1.39s).
-----> Kitchen is finished. (0m16.51s)
\end{lstlisting}

There's only one remaining action left that needs a mention: the Destroy Action which destroys the instance. With this in mind, here's what Test Kitchen is doing in the Test Action:

\begin{itemize}
  \item Destroys the instance if it exists (\inline{Cleaning up any prior instances of <default-ubuntu-1204>})
  \item Creates the instance (\inline{Creating <default-ubuntu-1204>})
  \item Converges the instance (\inline{Converging <default-ubuntu-1204>})
  \item Sets up Busser and runner plugins on the instance (\inline{Setting up <default-ubuntu-1204>})
  \item Verifies the instance by running Busser tests (\inline{Verifying <default-ubuntu-1204>})
  \item Destroys the instance (\inline{Destroying <default-ubuntu-1204>})
\end{itemize}

A few details with regards to test:

\begin{itemize}
  \item Test Kitchen will abort the run on the instance at the first sign of trouble. This means that if your Chef run fails then Busser won't be installed and the instance won't be destroyed. This gives you a chance to inspect the state of the instance and fix any issues.
  \item The behavior of the final destroy action can be overridden if desired. Check out the documentation for the \inline{--destroy} flag using \inline{kitchen help test}.
  \item The primary use case in mind for this meta-action is in a Continuous Integration environment or a command for developers to run before check in or after a fresh clone. If you're using this in your test-code-verify development cycle it's going to quickly become very slow and frustrating. You're better off running the \inline{converge} and \inline{verify} subcommands in development and save the test subcommand when you need to verify the end-to-end run of your code.
\end{itemize}

Finally, let's check the status of the instance:

\begin{lstlisting}[language=Bash,label=lst:testing-test-kitchen19]
$ kitchen list
Instance             Driver   Provisioner  Last Action
default-ubuntu-1204  Vagrant  ChefSolo     <Not Created>
\end{lstlisting}




\subsubsection{Adding a Platform}

Now that we are masters of the Ubuntu platform, let's add support for CentOS to our cookbook. This shouldn't be too bad. Open \inline{.kitchen.yml} in your editor and the \inline{centos-6.4} line to your platforms list so that it resembles:

\begin{lstlisting}[label=lst:testing-test-kitchen20]
---
driver:
  name: vagrant

provisioner:
  name: chef_solo

platforms:
  - name: ubuntu-12.04
  - name: centos-6.4

suites:
  - name: default
    run_list:
      - recipe[my_cool_app::default]
    attributes:
\end{lstlisting}

Now let's check the status of our instances:

\begin{lstlisting}[language=Bash,label=lst:testing-test-kitchen21]
$ kitchen list
Instance             Driver   Provisioner  Last Action
default-ubuntu-1204  Vagrant  Chef Solo    <Not Created>
default-centos-64    Vagrant  Chef Solo    <Not Created>
\end{lstlisting}

We're going to use two shortcuts here in the next command:

\begin{itemize}
  \item Each Test Kitchen instance has a simple state machine that tracks where it is in its lifecyle. Given its current state and the desired state, the instance is smart enough to run all actions in between current and desired. In our next example we will take an instance from not created to verified in one command.
  \item Any \inline{kitchen} subcommand that takes an instance name as an argument can take a Ruby regular expression that will be used to glob a list of instances together. This means that \inline{kitchen test ubuntu} would run the test action only the ubuntu instance. Note that the list subcommand also takes the regex-globbing argument so feel free to experiment there. In our next example we'll select the \inline{default-centos-64} instance with simply \inline{64}.
\end{itemize}

Let's see how CentOS runs our cookbook:

\begin{lstlisting}[language=Bash,label=lst:testing-test-kitchen22]
$ kitchen verify 64
-----> Starting Kitchen (v1.2.1)
-----> Creating <default-centos-64>...
...
-----> Running bats test suite
   nginx binary is found in PATH
   nginx config is valid                                                    2/3
   nginx config is valid
   nginx is running                                                         3/3
   nginx is running
          (in test file /tmp/busser/suites/bats/nginx_installed.bats, line 16)

       3 tests, 1 failure
\end{lstlisting}

We must fix last test:

\begin{lstlisting}[language=Bash,label=lst:testing-test-kitchen23]
@test "nginx is running" {
   run service nginx status
   [ "$status" -eq 0 ]
-  [ "$output" == " * nginx is running" ]
+  [ $(expr "$output" : ".*nginx.*running") -ne 0 ]
}
\end{lstlisting}

And check again on all instances:

\begin{lstlisting}[language=Bash,label=lst:testing-test-kitchen24]
$ kitchen verify
-----> Starting Kitchen (v1.2.1)
-----> Creating <default-centos-64>...
...
-----> Verifying <default-ubuntu-1204>...
       Suite path directory /tmp/busser/suites does not exist, skipping.
Uploading /tmp/busser/suites/bats/nginx_installed.bats (mode=0755)
-----> Running bats test suite
   nginx binary is found in PATH
   nginx config is valid
   nginx is running

3 tests, 0 failures
       Finished verifying <default-ubuntu-1204> (0m1.46s).
-----> Verifying <default-centos-64>...
       Removing /tmp/busser/suites/bats
       Uploading /tmp/busser/suites/bats/nginx_installed.bats (mode=0755)
-----> Running bats test suite
   nginx binary is found in PATH
   nginx config is valid                                                    2/3
   nginx config is valid
   nginx is running                                                         3/3
   nginx is running

       3 tests, 0 failures
       Finished verifying <default-centos-64> (0m1.81s).
\end{lstlisting}

Nice! We've verified that our cookbook works on Ubuntu 12.04 and CentOS 6.4. Since the CentOS instance will hang out for no good reason, let's kill it for now:

\begin{lstlisting}[language=Bash,label=lst:testing-test-kitchen25]
$ kitchen destroy
-----> Starting Kitchen (v1.2.1)
-----> Destroying <default-ubuntu-1204>...
       ...
       Finished destroying <default-ubuntu-1204> (0m5.82s).
-----> Destroying <default-centos-64>...
       ...
       Finished destroying <default-centos-64> (0m5.41s).
-----> Kitchen is finished. (0m12.95s)
\end{lstlisting}

Any \inline{kitchen} subcommand without an instance argument will apply to all instances.



\subsubsection{Fixing Converge}

Now a colleague has expressed interest in running the cookbook on a fleet of older Ubuntu 10.04 systems. Open \inline{.kitchen.yml} in your editor and add a \inline{ubuntu-10.04} entry to the platforms list:

\begin{lstlisting}[label=lst:testing-test-kitchen26]
---
driver:
  name: vagrant

provisioner:
  name: chef_solo

platforms:
  - name: ubuntu-12.04
  - name: ubuntu-10.04
  - name: centos-6.4

suites:
  - name: default
    run_list:
      - recipe[my_cool_app::default]
    attributes:
\end{lstlisting}

And run \inline{kitchen list} to confirm the introduction of our latest instance:

\begin{lstlisting}[language=Bash,label=lst:testing-test-kitchen27]
$ kitchen list
Instance             Driver   Provisioner  Last Action
default-ubuntu-1204  Vagrant  ChefSolo     <Not Created>
default-ubuntu-1004  Vagrant  ChefSolo     <Not Created>
default-centos-64    Vagrant  ChefSolo     <Not Created>
\end{lstlisting}

Now we'll run the \inline{test} subcommand and go grab a coffee:

\begin{lstlisting}[language=Bash,label=lst:testing-test-kitchen28]
$ kitchen test 10
-----> Starting Kitchen (v1.2.1)
-----> Cleaning up any prior instances of <default-ubuntu-1004>
-----> Destroying <default-ubuntu-1004>...
       Finished destroying <default-ubuntu-1004> (0m0.00s).
-----> Testing <default-ubuntu-1004>
...
Chef::Exceptions::Package
       -------------------------
       git has no candidate in the apt-cache
\end{lstlisting}

Oh noes! Argh, why!? Let's login to the instance and see if we can figure out what the correct package is:

\begin{lstlisting}[language=Bash,label=lst:testing-test-kitchen29]
$  kitchen login 10

vagrant@default-ubuntu-1004:~$ sudo apt-cache search git | grep ^git
git-buildpackage - Suite to help with Debian packages in Git repositories
git-cola - highly caffeinated git gui
git-load-dirs - Import upstream archives into git
gitg - git repository viewer for gtk+/GNOME
gitmagic - guide about Git version control system
gitosis - git repository hosting application
gitpkg - tools for maintaining Debian packages with git
gitstats - statistics generator for git repositories
git-core - fast, scalable, distributed revision control system
git-doc - fast, scalable, distributed revision control system (documentation)
gitk - fast, scalable, distributed revision control system (revision tree visualizer)
git-arch - fast, scalable, distributed revision control system (arch interoperability)
git-cvs - fast, scalable, distributed revision control system (cvs interoperability)
git-daemon-run - fast, scalable, distributed revision control system (git-daemon service)
git-email - fast, scalable, distributed revision control system (email add-on)
git-gui - fast, scalable, distributed revision control system (GUI)
git-svn - fast, scalable, distributed revision control system (svn interoperability)
gitweb - fast, scalable, distributed revision control system (web interface)
\end{lstlisting}

Okay, it looks like we want to install the \inline{git-core} package for this release of Ubuntu. Let's fix this up back in the default recipe. Open up \inline{recipes/default.rb} and edit to something like:

\begin{lstlisting}[language=Ruby,label=lst:testing-test-kitchen30]
# install needed package
packages = %w(git)

if "ubuntu" == node['platform'] && node['platform_version'].to_f <= 10.04
  packages << "git-core"
else
  packages << "git"
end

packages.each do |pack|
  package pack
end
\end{lstlisting}

This may not be pretty but let's verify that it works first on Ubuntu 10.04:

\begin{lstlisting}[language=Bash,label=lst:testing-test-kitchen31]
$ kitchen verify 10
...
-----> Running bats test suite
   nginx binary is found in PATH
   nginx config is valid                                                    2/3
   nginx config is valid
   nginx is running                                                         3/3
   nginx is running

       3 tests, 0 failures
       Finished verifying <default-ubuntu-1004> (0m2.51s).
-----> Kitchen is finished. (0m53.48s)
\end{lstlisting}

Back to green, good. Let's verify that the all instances are still good.

\begin{lstlisting}[language=Bash,label=lst:testing-test-kitchen32]
$ kitchen verify -c 3
-----> Starting Kitchen (v1.2.1)
-----> Verifying <default-centos-64>...
-----> Verifying <default-ubuntu-1004>...
-----> Verifying <default-ubuntu-1204>...
       Removing /tmp/busser/suites/bats
       Removing /tmp/busser/suites/bats
       Uploading /tmp/busser/suites/bats/nginx_installed.bats (mode=0755)
Uploading /tmp/busser/suites/bats/nginx_installed.bats (mode=0755)
       Removing /tmp/busser/suites/bats
       Uploading /tmp/busser/suites/bats/nginx_installed.bats (mode=0755)
-----> Running bats test suite
 -----> Running bats test suite                                             1/3
   nginx binary is found in PATH
   nginx config is valid                                                    2/3
   nginx config is valid
   nginx is running                                                         3/3
   nginx is running

       3 tests, 0 failures
        Finished verifying <default-ubuntu-1004> (0m2.94s).                 1/3
   nginx binary is found in PATH
   nginx config is valid
   nginx is running

       3 tests, 0 failures
       Finished verifying <default-ubuntu-1204> (0m3.09s).
-----> Running bats test suite
   nginx binary is found in PATH
   nginx config is valid                                                    2/3
   nginx config is valid
   nginx is running                                                         3/3
   nginx is running

       3 tests, 0 failures
       Finished verifying <default-centos-64> (0m3.86s).
-----> Kitchen is finished. (0m6.12s)
\end{lstlisting}

We used \inline{-c} to run a test against all matching instances concurrently, where next argument mean number of instances run at the same time.

We've successfully verified all three instances, so let's shut them down.

\begin{lstlisting}[language=Bash,label=lst:testing-test-kitchen33]
$ kitchen destroy
-----> Starting Kitchen (v1.2.1)
...
-----> Kitchen is finished. (0m19.86s)
\end{lstlisting}


