\subsection{Installing}

Let's cover our cookbook by test kitchen. First we should add this gem in Gemfile:

\begin{lstlisting}[label=lst:testing-test-kitchen1]
source 'https://rubygems.org'

gem 'berkshelf'
gem 'foodcritic'
gem 'thor-foodcritic'
gem 'chefspec'
gem 'test-kitchen'
gem 'kitchen-vagrant'
\end{lstlisting}

And you should to execute \lstinline!bundle! command to install this gems. We can check what kitchen installed:

\begin{lstlisting}[language=Bash,label=lst:testing-test-kitchen2]
$ kitchen version
Test Kitchen version 1.2.1
$ kitchen help
Commands:
  kitchen console                         # Kitchen Console!
  kitchen converge [INSTANCE|REGEXP|all]  # Converge one or more instances
  kitchen create [INSTANCE|REGEXP|all]    # Create one or more instances
  kitchen destroy [INSTANCE|REGEXP|all]   # Destroy one or more instances
  kitchen diagnose [INSTANCE|REGEXP|all]  # Show computed diagnostic configuration
  kitchen driver                          # Driver subcommands
  kitchen driver create [NAME]            # Create a new Kitchen Driver gem project
  kitchen driver discover                 # Discover Test Kitchen drivers published on RubyGems
  kitchen driver help [COMMAND]           # Describe subcommands or one specific subcommand
  kitchen help [COMMAND]                  # Describe available commands or one specific command
  kitchen init                            # Adds some configuration to your cookbook so Kitchen can rock
  kitchen list [INSTANCE|REGEXP|all]      # Lists one or more instances
  kitchen login INSTANCE|REGEXP           # Log in to one instance
  kitchen setup [INSTANCE|REGEXP|all]     # Setup one or more instances
  kitchen test [INSTANCE|REGEXP|all]      # Test one or more instances
  kitchen verify [INSTANCE|REGEXP|all]    # Verify one or more instances
  kitchen version                         # Print Kitchen's version information
\end{lstlisting}

Now we'll add Test Kitchen to our project by using the init subcommand:

\begin{lstlisting}[language=Bash,label=lst:testing-test-kitchen3]
$ kitchen init
    create  .kitchen.yml
    append  Thorfile
    create  test/integration/default
\end{lstlisting}

What's going on here? The \lstinline!kitchen init! subcommand will create an initial configuration file for Test Kitchen called \lstinline!.kitchen.yml!. A few directories were created but these are only a convenience – you don't strictly need <<test/integration/default>> in your project. You can see that you have a \lstinline!.gitignore! file in your project's root which will tell Git to never commit a directory called \lstinline!.kitchen! and something called \lstinline!.kitchen.local.yml!. Finally, a gem called \lstinline!kitchen-vagrant! was installed. By itself Test Kitchen can't do very much. It needs one or more Drivers which are responsible for managing the virtual machines we need for testing. At present there are many different Test Kitchen Drivers but we're going to stick with the \href{https://github.com/opscode/kitchen-vagrant}{Kitchen Vagrant Driver} for now.

Let's turn our attention to the \lstinline!.kitchen.yml! file. While Test Kitchen may have created the initial file automatically, it's expected that you will read and edit this file. Opening this file in your editor of choice we see something like the following:

\begin{lstlisting}[label=lst:testing-test-kitchen4]
---
driver:
  name: vagrant

provisioner:
  name: chef_solo

platforms:
  - name: ubuntu-12.04
  - name: centos-6.4

suites:
  - name: default
    run_list:
      - recipe[my_cool_app::default]
    attributes:
\end{lstlisting}

Very briefly we can cover the 4 main sections you're likely to find in a \lstinline!.kitchen.yml! file:

\begin{itemize}
  \item \lstinline!driver!: This is where we configure the behaviour of the Kitchen Driver - the component that is responsible for creating a machine that we'll use to test our cookbook. Here we set up basics like credentials, ssh usernames, sudo requirements, etc. Each Driver is responsible for requiring and using the configuration here. Under this section we have \lstinline!driver.name!: This tells Test Kitchen that we want to use the \lstinline!kitchen-vagrant! driver by default unless otherwise specified.
  \item \lstinline!provisioner!: This tells Test Kitchen how to run Chef, to apply the code in our cookbook to the machine under test. The default and simplest approach is to use \lstinline!chef-solo!, but other options are available, and ultimately Test Kitchen doesn't care how the infrastructure is built - it could theoretically be with Puppet, Ansible, or Perl for all it cares.
  \item \lstinline!platforms!: This is a list of operation systems on which we want to run our code. Note that the operating system's version, architecture, cloud environment, etc. might be relevant to what Test Kitchen considers a Platform.
  \item \lstinline!suites!: This section defines what we want to test. It includes the Chef run-list and any node attribute setups that we want run on each Platform above. For example, we might want to test the MySQL client cookbook code separately from the server cookbook code for maximum isolation.
\end{itemize}

Let's say for argument's sake that we only care about running our Chef cookbook on Ubuntu 12.04 distributions. In that case, we can edit the \lstinline!.kitchen.yml! file so that the list of platforms has only one entry like so:

\begin{lstlisting}[label=lst:testing-test-kitchen5]
---
driver:
  name: vagrant

provisioner:
  name: chef_solo

platforms:
  - name: ubuntu-12.04

suites:
  - name: default
    run_list:
      - recipe[my_cool_app::default]
    attributes:
\end{lstlisting}

To see the results of our work, let's run the \lstinline!kitchen list! subcommand:

\begin{lstlisting}[language=Bash,label=lst:testing-test-kitchen6]
$ kitchen list
Instance             Driver   Provisioner  Last Action
default-ubuntu-1204  Vagrant  ChefSolo     <Not Created>
\end{lstlisting}

So what's this <<default-ubuntu-1204>> thing and what's an <<Instance>>? A Test Kitchen Instance is a pairwise combination of a Suite and a Platform as laid out in your \lstinline!.kitchen.yml! file. Test Kitchen has auto-named your only instance by combining the Suite name (<<default>>) and the Platform name (<<ubuntu-12.04>>) into a form that is safe for DNS and hostname records, namely <<default-ubuntu-1204>>.

Okay, let's spin this Instance up to see what happens. Test Kitchen calls this the Create Action. We're going to be painfully explicit and ask Test Kitchen to only create the <<default-ubuntu-1204>> instance:

\begin{lstlisting}[language=Bash,label=lst:testing-test-kitchen7]
$ kitchen create default-ubuntu-1204
-----> Starting Kitchen (v1.2.1)
-----> Creating <default-ubuntu-1204>...
       Bringing machine 'default' up with 'virtualbox' provider...
       [default] Importing base box 'opscode-ubuntu-12.04'...
       [default] Matching MAC address for NAT networking...
       [default] Setting the name of the VM...
       [default] Clearing any previously set forwarded ports...
       [default] Clearing any previously set network interfaces...
       [default] Preparing network interfaces based on configuration...
       [default] Forwarding ports...
       [default] -- 22 => 2222 (adapter 1)
       [default] Running 'pre-boot' VM customizations...
       [default] Booting VM...
       [default] Waiting for machine to boot. This may take a few minutes...
       [default] Machine booted and ready!
       [default] Setting hostname...
       Vagrant instance <default-ubuntu-1204> created.
       Finished creating <default-ubuntu-1204> (3m56.11s).
-----> Kitchen is finished. (3m57.66s)
\end{lstlisting}

If you are a Vagrant user then the line containing \lstinline!vagrant up --no-provision! will look familiar. Let's check the status of our instance now:

\begin{lstlisting}[language=Bash,label=lst:testing-test-kitchen8]
$ kitchen list
Instance             Driver   Provisioner  Last Action
default-ubuntu-1204  Vagrant  ChefSolo     Created
\end{lstlisting}
