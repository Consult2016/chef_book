\subsection{Adding a Platform}

Now that we are masters of the Ubuntu platform, let's add support for CentOS to our cookbook. This shouldn't be too bad. Open \inline!.kitchen.yml! in your editor and the \inline!centos-6.4! line to your platforms list so that it resembles:

\begin{lstlisting}[label=lst:testing-test-kitchen20]
---
driver:
  name: vagrant

provisioner:
  name: chef_solo

platforms:
  - name: ubuntu-12.04
  - name: centos-6.4

suites:
  - name: default
    run_list:
      - recipe[my_cool_app::default]
    attributes:
\end{lstlisting}

Now let's check the status of our instances:

\begin{lstlisting}[language=Bash,label=lst:testing-test-kitchen21]
$ kitchen list
Instance             Driver   Provisioner  Last Action
default-ubuntu-1204  Vagrant  Chef Solo    <Not Created>
default-centos-64    Vagrant  Chef Solo    <Not Created>
\end{lstlisting}

We're going to use two shortcuts here in the next command:

\begin{itemize}
  \item Each Test Kitchen instance has a simple state machine that tracks where it is in its lifecycle. Given its current state and the desired state, the instance is smart enough to run all actions in between current and desired. In our next example we will take an instance from not created to verified in one command.
  \item Any \inline!kitchen! subcommand that takes an instance name as an argument can take a Ruby regular expression that will be used to glob a list of instances together. This means that \inline!kitchen test ubuntu! would run the test action only the ubuntu instance. Note that the list subcommand also takes the regex-globbing argument so feel free to experiment there. In our next example we'll select the \inline!default-centos-64! instance with simply \inline!64!.
\end{itemize}

Let's see how CentOS runs our cookbook:

\begin{lstlisting}[language=Bash,label=lst:testing-test-kitchen22]
$ kitchen verify 64
-----> Starting Kitchen (v1.2.1)
-----> Creating <default-centos-64>...
...
-----> Running bats test suite
   nginx binary is found in PATH
   nginx config is valid                                                    2/3
   nginx config is valid
   nginx is running                                                         3/3
   nginx is running
          (in test file /tmp/busser/suites/bats/nginx_installed.bats, line 16)

       3 tests, 1 failure
\end{lstlisting}

We should fix failed test:

\begin{lstlisting}[language=Bash,label=lst:testing-test-kitchen23]
@test "nginx is running" {
   run service nginx status
   [ "$status" -eq 0 ]
-  [ "$output" == " * nginx is running" ]
+  [ $(expr "$output" : ".*nginx.*running") -ne 0 ]
}
\end{lstlisting}

And check again on all instances:

\begin{lstlisting}[language=Bash,label=lst:testing-test-kitchen24]
$ kitchen verify
-----> Starting Kitchen (v1.2.1)
-----> Creating <default-centos-64>...
...
-----> Verifying <default-ubuntu-1204>...
       Suite path directory /tmp/busser/suites does not exist, skipping.
Uploading /tmp/busser/suites/bats/nginx_installed.bats (mode=0755)
-----> Running bats test suite
   nginx binary is found in PATH
   nginx config is valid
   nginx is running

3 tests, 0 failures
       Finished verifying <default-ubuntu-1204> (0m1.46s).
-----> Verifying <default-centos-64>...
       Removing /tmp/busser/suites/bats
       Uploading /tmp/busser/suites/bats/nginx_installed.bats (mode=0755)
-----> Running bats test suite
   nginx binary is found in PATH
   nginx config is valid                                                    2/3
   nginx config is valid
   nginx is running                                                         3/3
   nginx is running

       3 tests, 0 failures
       Finished verifying <default-centos-64> (0m1.81s).
\end{lstlisting}

Nice! We've verified that our cookbook works on Ubuntu 12.04 and CentOS 6.4. Since the CentOS instance will hang out for no good reason, let's kill it for now:

\begin{lstlisting}[language=Bash,label=lst:testing-test-kitchen25]
$ kitchen destroy
-----> Starting Kitchen (v1.2.1)
-----> Destroying <default-ubuntu-1204>...
       ...
       Finished destroying <default-ubuntu-1204> (0m5.82s).
-----> Destroying <default-centos-64>...
       ...
       Finished destroying <default-centos-64> (0m5.41s).
-----> Kitchen is finished. (0m12.95s)
\end{lstlisting}

Any \inline!kitchen! subcommand without an instance argument will apply to all instances.



\subsection{Fixing Converge}

Now a colleague has expressed interest in running the cookbook on a fleet of older Ubuntu 10.04 systems. Open \inline!.kitchen.yml! in your editor and add a \inline!ubuntu-10.04! entry to the platforms list:

\begin{lstlisting}[label=lst:testing-test-kitchen26]
---
driver:
  name: vagrant

provisioner:
  name: chef_solo

platforms:
  - name: ubuntu-12.04
  - name: ubuntu-10.04
  - name: centos-6.4

suites:
  - name: default
    run_list:
      - recipe[my_cool_app::default]
    attributes:
\end{lstlisting}

And run \inline!kitchen list! to confirm the introduction of our latest instance:

\begin{lstlisting}[language=Bash,label=lst:testing-test-kitchen27]
$ kitchen list
Instance             Driver   Provisioner  Last Action
default-ubuntu-1204  Vagrant  ChefSolo     <Not Created>
default-ubuntu-1004  Vagrant  ChefSolo     <Not Created>
default-centos-64    Vagrant  ChefSolo     <Not Created>
\end{lstlisting}

Now we'll run the \inline!test! subcommand and go grab a coffee:

\begin{lstlisting}[language=Bash,label=lst:testing-test-kitchen28]
$ kitchen test 10
-----> Starting Kitchen (v1.2.1)
-----> Cleaning up any prior instances of <default-ubuntu-1004>
-----> Destroying <default-ubuntu-1004>...
       Finished destroying <default-ubuntu-1004> (0m0.00s).
-----> Testing <default-ubuntu-1004>
...
Chef::Exceptions::Package
       -------------------------
       git has no candidate in the apt-cache
\end{lstlisting}

Oh noes! Argh, why!? Let's login to the instance and see if we can figure out what the correct package is:

\begin{lstlisting}[language=Bash,label=lst:testing-test-kitchen29]
$  kitchen login 10

vagrant@default-ubuntu-1004:~$ sudo apt-cache search git | grep ^git
git-buildpackage - Suite to help with Debian packages in Git repositories
git-cola - highly caffeinated git gui
git-load-dirs - Import upstream archives into git
gitg - git repository viewer for gtk+/GNOME
gitmagic - guide about Git version control system
gitosis - git repository hosting application
gitpkg - tools for maintaining Debian packages with git
gitstats - statistics generator for git repositories
git-core - fast, scalable, distributed revision control system
git-doc - fast, scalable, distributed revision control system (documentation)
gitk - fast, scalable, distributed revision control system (revision tree visualizer)
git-arch - fast, scalable, distributed revision control system (arch interoperability)
git-cvs - fast, scalable, distributed revision control system (cvs interoperability)
git-daemon-run - fast, scalable, distributed revision control system (git-daemon service)
git-email - fast, scalable, distributed revision control system (email add-on)
git-gui - fast, scalable, distributed revision control system (GUI)
git-svn - fast, scalable, distributed revision control system (svn interoperability)
gitweb - fast, scalable, distributed revision control system (web interface)
\end{lstlisting}

Okay, it looks like we want to install the \inline!git-core! package for this release of Ubuntu. Let's fix this up back in the default recipe. Open up \inline!recipes/default.rb! and edit to something like:

\begin{lstlisting}[language=Ruby,label=lst:testing-test-kitchen30]
# install needed package
packages = %w(ntp)

if "ubuntu" == node['platform'] && node['platform_version'].to_f <= 10.04
  packages << "git-core"
else
  packages << "git"
end

packages.each do |pack|
  package pack
end
\end{lstlisting}

This may not be pretty but let's verify that it works first on Ubuntu 10.04:

\begin{lstlisting}[language=Bash,label=lst:testing-test-kitchen31]
$ kitchen verify 10
...
-----> Running bats test suite
   nginx binary is found in PATH
   nginx config is valid                                                    2/3
   nginx config is valid
   nginx is running                                                         3/3
   nginx is running

       3 tests, 0 failures
       Finished verifying <default-ubuntu-1004> (0m2.51s).
-----> Kitchen is finished. (0m53.48s)
\end{lstlisting}

Back to green, good. Let's verify that the all instances are still good.

\begin{lstlisting}[language=Bash,label=lst:testing-test-kitchen32]
$ kitchen verify -c 3
-----> Starting Kitchen (v1.2.1)
-----> Verifying <default-centos-64>...
-----> Verifying <default-ubuntu-1004>...
-----> Verifying <default-ubuntu-1204>...
       Removing /tmp/busser/suites/bats
       Removing /tmp/busser/suites/bats
       Uploading /tmp/busser/suites/bats/nginx_installed.bats (mode=0755)
Uploading /tmp/busser/suites/bats/nginx_installed.bats (mode=0755)
       Removing /tmp/busser/suites/bats
       Uploading /tmp/busser/suites/bats/nginx_installed.bats (mode=0755)
-----> Running bats test suite
 -----> Running bats test suite                                             1/3
   nginx binary is found in PATH
   nginx config is valid                                                    2/3
   nginx config is valid
   nginx is running                                                         3/3
   nginx is running

       3 tests, 0 failures
        Finished verifying <default-ubuntu-1004> (0m2.94s).                 1/3
   nginx binary is found in PATH
   nginx config is valid
   nginx is running

       3 tests, 0 failures
       Finished verifying <default-ubuntu-1204> (0m3.09s).
-----> Running bats test suite
   nginx binary is found in PATH
   nginx config is valid                                                    2/3
   nginx config is valid
   nginx is running                                                         3/3
   nginx is running

       3 tests, 0 failures
       Finished verifying <default-centos-64> (0m3.86s).
-----> Kitchen is finished. (0m6.12s)
\end{lstlisting}

We used \inline!-c! to run a test against all matching instances concurrently, where next argument mean number of instances run at the same time.

We've successfully verified all three instances, so let's shut them down.

\begin{lstlisting}[language=Bash,label=lst:testing-test-kitchen33]
$ kitchen destroy
-----> Starting Kitchen (v1.2.1)
...
-----> Kitchen is finished. (0m19.86s)
\end{lstlisting}

