\section{What are the core principles?}
\label{sec:what-principles}

\subsection{Idempotence}

A recipe can run multiple times on the same system and the results will always be identical. A resource is defined in a recipe, which then defines the actions to be performed on the system. The chef-client ensures that actions are not performed if the resources have not changed and that any action that is performed is done the same way each time. If a recipe is re-run and nothing has changed, then the chef-client will not do anything.

\subsection{Thick Clients, Thin Server}

Chef does as much work as possible on the node and as little as possible on the server. A chef-client runs on each node and it only interacts with the server when it needs to. The server is designed to distribute the data to each node easily, including all cookbooks, recipes, templates, files, and so on. The server also retains a copy of the state of the node at the conclusion of every chef-client run. This approach ensures that the actual work needed to configure each node in your infrastructure is distributed across the organization, rather than centralized on smaller set of configuration management servers.

\subsection{Order Matters}

When the chef-client configures each node in the system, the order in which that configuration occurs is very important. For example, if Apache is not installed, then it cannot be configured and the daemon cannot be started. Configuration management tools have struggled with this problem for a long time. For each node a list of recipes is applied. Within a recipe, resources are applied in the order in which they are listed. At any point in a recipe other recipes may be included, which assures that all resources are applied. The chef-client will never apply the same recipe twice. Dependencies are only applied at the recipe level (and never the resource level). This means that dependencies can be tracked between high-level concepts like <<I need to install Apache before I can start my Django application!>> It also means that given the same set of cookbooks, the chef-client will always execute resources in the same exact order.
