\section{Attributes}

An attribute can be defined in a cookbook (or a recipe) and then used to override the default settings on a node. When a cookbook is loaded during a chef-client run, these attributes are compared to the attributes that are already present on the node. When the cookbook attributes take precedence over the default attributes, the chef-client will apply those new settings and values during the chef-client run on the node.

An attribute file is located in the <<attributes/>> sub-directory for a cookbook. When a cookbook is run against a node, the attributes contained in all attribute files are evaluated in the context of the node object. Node methods (when present) are used to set attribute values on a node. For example, the apache2 cookbook contains an attribute file called <<default.rb>>, which contains the following attributes:

\begin{lstlisting}[label=lst:cookbook-attributes1]
default["apache"]["dir"]          = "/etc/apache2"
default["apache"]["listen_ports"] = [ "80","443" ]
\end{lstlisting}

The use of the node object (node) is implicit in the previous example; the following example defines the node object itself as part of the attribute:

\begin{lstlisting}[label=lst:cookbook-attributes2]
node.default["apache"]["dir"]          = "/etc/apache2"
node.default["apache"]["listen_ports"] = [ "80","443" ]
\end{lstlisting}
