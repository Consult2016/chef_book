\section{Definitions}
\label{sec:cookbook-definitions}

A definition is used to declare resources so they can be added to the resource collection. A definition is not a resource or a lightweight resource. A definition does not have an associated provider. A definition groups two (or more) resource declarations. There is no limit to the number of resources that can be part of a definition. All definitions within a cookbook must be located in the <<definitions/>> folder. A definition is never declared into a cookbook. A definition is best-used when:

\begin{itemize}
  \item Data needs to be passed from one (or more) recipes into a single definition
  \item A repeating usage pattern exists for one (or more) resources
  \item An action does not need to be sent directly to a resource (when it does, it should be sent to a provider)
\end{itemize}

In our \inline{my_cool_app} cookbook we are using \inline{nginx_site} definition from nginx cookbook.

Right now definitions continue working in Chef, but better to use LWRP (or HWRP) instead definition (in chapter <<\ref{sec:testing-foodcritic}~\nameref{sec:testing-foodcritic}>> you will see warning about this).

\subsection{Example}

Let's create definition in our \inline{my_cool_app} cookbook. We will move our nginx conf creation into definition. Our code:

\begin{lstlisting}[label=lst:cookbook-definitions-default1,title=my-server-cloud/site-cookbooks/my\_cool\_app/recipes/default.rb]
# create nginx config from temlate nginx.conf.erb
nginx_config = "#{node['nginx']['dir']}/sites-available/#{node['my_cool_app']['name']}.conf"
template nginx_config do
  source "nginx.conf.erb"
  mode "0644"
end

# activate conf in nginx
nginx_site "#{node['my_cool_app']['name']}.conf"
\end{lstlisting}

we moved to definition \inline{enable_web_site}:

\begin{lstlisting}[label=lst:cookbook-definitions-enable-site,title=my-server-cloud/site-cookbooks/my\_cool\_app/definitions/enable\_web\_site.rb]
define :enable_web_site, :enable => true, :template => "site.conf.erb" do
  if params[:enable]
    # create nginx config from temlate nginx.conf.erb
    nginx_config = "#{node['nginx']['dir']}/sites-available/#{params[:name]}.conf"
    template nginx_config do
      source params[:template]
      mode "0644"
    end
    # activate conf in nginx
    nginx_site "#{params[:name]}.conf"
  else
    # deactivateconf in nginx
    nginx_site "#{params[:name]}.conf" do
      enable    false
    end
  end
end
\end{lstlisting}

And now we can replace code in~\ref{lst:cookbook-definitions-default1} by call of \inline{enable_web_site}:

\begin{lstlisting}[label=lst:cookbook-definitions-default2,title=my-server-cloud/site-cookbooks/my\_cool\_app/recipes/default.rb]
# enable website
enable_web_site node['my_cool_app']['name'] do
  template "nginx.conf.erb"
end
\end{lstlisting}

Now it remains to check how it works:

\begin{lstlisting}[language=Bash,label=lst:cookbook-definitions-shell1]
$ knife cookbook upload my_cool_app
Uploading my_cool_app    [0.1.0]
Uploaded 1 cookbook.

$ vagrant provision chef_second_client
...
INFO: template[/etc/nginx/sites-available/my_cool_app.conf] created file /etc/nginx/sites-available/my_cool_app.conf
INFO: template[/etc/nginx/sites-available/my_cool_app.conf] updated file contents /etc/nginx/sites-available/my_cool_app.conf
INFO: template[/etc/nginx/sites-available/my_cool_app.conf] mode changed to 644
INFO: execute[nxensite my_cool_app.conf] ran successfully
...
\end{lstlisting}

As you can see our definition is works as expected.