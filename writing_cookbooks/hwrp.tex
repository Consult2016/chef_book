\section{HWRPs}
\label{sec:cookbook-hwrp}

When Chef first came out, there was no Light Weight Resource Provider (LWRP) syntax and any hardcore extension to Chef had to be written in Ruby. However, Chef team saw a need to be filled and created LWRP, making it easier to create your own Resources. The problem comes when LWRP cannot fulfill all of your needs. This means you need to fall back to writing pure ruby code. For lack of a better term, I'll call this method a HWRP, or Heavy Weight Resource Provider.

While writing a LWRP is meant to be simple and elegant, writing a HWRP is meant to be flexible. It gives you the full power of ruby in exchange for elegance.

\subsection{HWRPs and LWRPS}

With LWRP you are taught to create a Resource and a Provider together. This is the simplest way. However, just because you need to convert a resource definition or a provider into a HWRP you do not need to convert both.

The LWRP syntax <<compiles>> into real ruby code, so Chef will not know the difference in how they were defined. A valid cookbook directory structure:

\begin{lstlisting}[language=Bash,label=lst:cookbook-hwrp1]
libraries/
    provider_default.rb
providers/
resources/
    default.rb
recipes/
    default.rb
metadata.rb
\end{lstlisting}

Anything you put in <<resources/>> or <<providers/>> Chef will attempt to parse at runtime. We don't want Chef trying to read our HWRP as the Chef DSL, we want it to interpret it as code. Luckily, anything stored in the <<libraries/>> folder Chef will try to import at runtime. A good example of this can be seen in the runit cookbook.

\subsection{Example}

Let's go through an example. We are going to create a HWRP that is very simple, it already written as a LWRP. Our HWRP will called \inline{my_cool_app_known_host}, to not conflict with already existed in cookbook LWRP \inline{my_cool_app_know_host}.

\subsubsection{Resource}

We need to inherit from the appropriate Chef classes in our HWRP. Note the class hierarchy as well as the inheritance:

\begin{lstlisting}[label=lst:cookbook-hwrp2]
require 'chef/resource'

class Chef
  class Resource
    class MyCoolAppKnownHost < Chef::Resource

    # Some Magic Happens

    end
  end
end
\end{lstlisting}

We need to override the initialize method to make sure we have some defaults. We aren't defining all of our resource attributes here, just the ones that need defaults.

\begin{lstlisting}[label=lst:cookbook-hwrp3]
require 'chef/resource'

class Chef
  class Resource
    class MyCoolAppKnownHost < Chef::Resource

      def initialize(name, run_context=nil)
        super
        # Bind ourselves to the name with an underscore
        @resource_name = :my_cool_app_known_host
        # We need to tie to our provider
        @provider = Chef::Provider::MyCoolAppKnownHost
        # Default Action Goes here
        @action = :create
        @allowed_actions = [:create, :delete]

        # Now we need to set up any resource defaults
        @port = 22
        @known_hosts_file = '/etc/ssh/ssh_known_hosts'
        @host = name  # This is equivalent to setting :name_attribute => true
      end

    end
  end
end
\end{lstlisting}

Now lets set up some attribute methods in our HWRP. Make sure to read the code comments for an explanation of what is going on.

\begin{lstlisting}[label=lst:cookbook-hwrp4]
require 'chef/resource'

class Chef
  class Resource
    class MyCoolAppKnownHost < Chef::Resource

      def initialize(name, run_context=nil)
        super
        # Bind ourselves to the name with an underscore
        @resource_name = :my_cool_app_known_host
        # We need to tie to our provider
        @provider = Chef::Provider::MyCoolAppKnownHost
        # Default Action Goes here
        @action = :create
        @allowed_actions = [:create, :delete]

        # Now we need to set up any resource defaults
        @port = 22
        @known_hosts_file = '/etc/ssh/ssh_known_hosts'
        @host = name  # This is equivalent to setting :name_attribute => true
      end

      # Define the attributes we set defaults for
      def key(arg=nil)
        set_or_return(:key, arg, :kind_of => String)
      end

      def host(arg=nil)
        set_or_return(:host, arg, :kind_of => String)
      end

      def port(arg=nil)
        set_or_return(:port, arg, :kind_of => Integer)
      end

      def known_hosts_file(arg=nil)
        set_or_return(:known_hosts_file, arg, :kind_of => String)
      end

    end
  end
end
\end{lstlisting}

\subsubsection{Providers}

Very similar to resources, here is the basic class structure for a provider.

\begin{lstlisting}[label=lst:cookbook-hwrp5]
require 'chef/provider'

class Chef
  class Provider
    class MyCoolAppKnownHost < Chef::Provider

    # Magic Happens

    end
  end
end
\end{lstlisting}

While we don’t need to write an initialize method (we can), we do need to override \inline{load_current_resource}.

\begin{lstlisting}[label=lst:cookbook-hwrp6]
require 'chef/provider'

class Chef
  class Provider
    class MyCoolAppKnownHost < Chef::Provider

      # We MUST override this method in our custom provider
      def load_current_resource
        # Here we keep the existing version of the resource
        # if none exists we create a new one from the resource we defined earlier
        @current_resource ||= Chef::Resource::MyCoolAppKnownHost.new(new_resource.name)

        # New resource represents the chef DSL block that is being run (from a recipe for example)
        @current_resource.key(new_resource.key)
        @current_resource.port(new_resource.port)
        @current_resource.known_hosts_file(new_resource.known_hosts_file)
        # Although you can reference @new_resource throughout the provider it is best to
        # only make modifications to the current version
        @current_resource.host(new_resource.host)
        @current_resource
      end

    end
  end
end
\end{lstlisting}

Now it is time to define what we do in our actions, with our HWRP we need to define methods like \inline{action_create} to define a \inline{:create} action. Chef will do some introspection to find these methods and hook them up.

\begin{lstlisting}[label=lst:cookbook-hwrp7]
require 'chef/provider'

class Chef
  class Provider
    class MyCoolAppKnownHost < Chef::Provider

      # We MUST override this method in our custom provider
      def load_current_resource
        # Here we keep the existing version of the resource
        # if none exists we create a new one from the resource we defined earlier
        @current_resource ||= Chef::Resource::MyCoolAppKnownHost.new(new_resource.name)

        # New resource represents the chef DSL block that is being run (from a recipe for example)
        @current_resource.key(new_resource.key)
        @current_resource.port(new_resource.port)
        @current_resource.known_hosts_file(new_resource.known_hosts_file)
        # Although you can reference @new_resource throughout the provider it is best to
        # only make modifications to the current version
        @current_resource.host(new_resource.host)
        @current_resource
      end

      def action_create
        Chef::Log.debug("#{@new_resource}: Create #{new_resource.host}")
      end

      def action_delete
        Chef::Log.debug("#{@new_resource}: Delete #{new_resource.host}")
      end

    end
  end
end
\end{lstlisting}

Now we can test it.

\begin{lstlisting}[label=lst:cookbook-hwrp8]
...
my_cool_app_known_host 'bitbucket.org'
\end{lstlisting}

To run chef-client in debug mode you should use \inline{-l} attribute with settings \inline{debug}. In vagrant you need set \inline{log_level} for node in Vagrantfile:

\begin{lstlisting}[label=lst:cookbook-hwrp9]
chef_client.vm.provision :chef_client do |chef|

  chef.log_level = :debug

  ...
end
\end{lstlisting}

And see our HWRP execution:

\begin{lstlisting}[language=Bash,label=lst:cookbook-hwrp10]
$ knife cookbook upload my_cool_app
Uploading my_cool_app    [0.1.0]
Uploaded 1 cookbook.

$ vagrant provision chef_second_client
[chef_second_client] Running provisioner: chef_client...
Creating folder to hold client key...
Uploading chef client validation key...
...
DEBUG: my_cool_app_known_host[bitbucket.org]: Create bitbucket.org
DEBUG: Saving the current state of node second.example.com
INFO: Chef Run complete in 15.491575653 seconds
...
\end{lstlisting}

Now let's add code for actions \inline{:create} and \inline{:delete}:

\begin{lstlisting}[label=lst:cookbook-hwrp11]
require 'chef/provider'

class Chef
  class Provider
    class MyCoolAppKnownHost < Chef::Provider

      ...

      def action_create
        Chef::Log.debug("#{@new_resource}: Create #{new_resource.host}")

        key, comment = insure_for_file(new_resource)
        # Use a Ruby block to edit the file
        ruby_block "add #{new_resource.host} to #{new_resource.known_hosts_file}" do
          block do
            file = ::Chef::Util::FileEdit.new(new_resource.known_hosts_file)
            file.insert_line_if_no_match(/#{Regexp.escape(comment)}|#{Regexp.escape(key)}/, key)
            file.write_file
          end
        end
        new_resource.updated_by_last_action(true)
      end

      def action_delete
        Chef::Log.debug("#{@new_resource}: Delete #{new_resource.host}")

        key, comment = insure_for_file(new_resource)
        # Use a Ruby block to edit the file
        ruby_block "del #{new_resource.host} from #{new_resource.known_hosts_file}" do
          block do
            file = ::Chef::Util::FileEdit.new(new_resource.known_hosts_file)
            file.search_file_delete_line(/#{Regexp.escape(comment)}|#{Regexp.escape(key)}/)
            file.write_file
          end
        end
        new_resource.updated_by_last_action(true)
      end

      private

      def insure_for_file(new_resource)
        key = (new_resource.key || `ssh-keyscan -H -p #{new_resource.port} #{new_resource.host} 2>&1`)
        comment = key.split("\n").first || ""

        Chef::Application.fatal! "Could not resolve #{new_resource.host}" if key =~ /getaddrinfo/

        # Ensure that the file exists and has minimal content (required by Chef::Util::FileEdit)
        file new_resource.known_hosts_file do
          action        :create
          backup        false
          content       '# This file must contain at least one line. This is that line.'
          only_if do
            !::File.exists?(new_resource.known_hosts_file) || ::File.new(new_resource.known_hosts_file).readlines.length == 0
          end
        end
        [key, comment]
      end

    end
  end
end
\end{lstlisting}

And check how it works:

\begin{lstlisting}[language=Bash,label=lst:cookbook-hwrp12]
$ knife cookbook upload my_cool_app
Uploading my_cool_app    [0.1.0]
Uploaded 1 cookbook.

$ vagrant provision chef_second_client
[chef_second_client] Running provisioner: chef_client...
Creating folder to hold client key...
Uploading chef client validation key...
...
WARN: Cloning resource attributes for file[/etc/ssh/ssh_known_hosts] from prior resource (CHEF-3694)
WARN: Previous file[/etc/ssh/ssh_known_hosts]: /var/chef/cache/cookbooks/my_cool_app/providers/know_host.rb:39:in `insure_for_file'
WARN: Current  file[/etc/ssh/ssh_known_hosts]: /var/chef/cache/cookbooks/my_cool_app/libraries/provider_known_host.rb:62:in `insure_for_file'
INFO: ruby_block[add bitbucket.org to /etc/ssh/ssh_known_hosts] called
...

$ vagrant ssh chef_second_client

vagrant@precise64:~$ cat /etc/ssh/ssh_known_hosts
# This file must contain at least one line. This is that line.
# github.com SSH-2.0-OpenSSH_6.2p2 Ubuntu-6ubuntu0.1+github2
|1|Daaa5BVIzI52zRmJ2ifMfOkkLJE=|Y611mqhJbVkdJ1onVkaqfV+3iks= ...
# bitbucket.org SSH-2.0-OpenSSH_5.3
|1|jycCFcglLajRKYIlyAJaD+zmOjw=|51hFV+x1XIZNdiMGG6K0Xz+Nkds= ...
\end{lstlisting}

So it works.