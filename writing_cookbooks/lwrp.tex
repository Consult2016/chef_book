\section{LWRPs}

A LWRP is a part of a cookbook that is used to extend the chef-client in a way that allows custom actions to be defined, and then used in recipes in much the same way as any platform resource. A LWRP has two principal components:

\begin{itemize}
  \item A lightweight resource that defines a set of actions and attributes
  \item A lightweight provider that tells the chef-client how to handle each action, what to do if certain conditions are met, and so on
\end{itemize}

In addition, most lightweight providers are built using platform resources and some lightweight providers are built using custom Ruby code.

Once created, a LWRP becomes a Ruby class within the organization. During each chef-client run, the chef-client will read the lightweight resources from recipes and process them alongside all of the other resources. When it is time to configure the node, the chef-client will use the corresponding lightweight provider to determine the steps required to bring the system into the desired state.

Where the lightweight resource represents a piece of the system, its current state, and the action that is needed to move it to the desired state, a lightweight provider defines the steps that are required to bring that piece of the system from its current state to the desired state. A LWRP behaves similar to platform resources and providers:

\begin{itemize}
  \item A lightweight resource is a key part of a recipe
  \item A lightweight resource defines the actions that can be taken
  \item During a chef-client run, each lightweight resource is identified, and then associated with a lightweight provider
  \item A lightweight provider does the work to complete the action requested by the lightweight resource
\end{itemize}

Lightweight resources and providers are loaded from files that are saved in the following cookbook sub-directories:

\begin{tabular}{ | l | l | }
  \hline
  Directory	& Description \\
  \hline
  providers/ & The sub-directory in which lightweight providers are located. \\
  resources/ & The sub-directory in which lightweight resources are located. \\
  \hline
\end{tabular}

The naming patterns of lightweight resources and providers are determined by the name of the cookbook and by the name of the files in the <<resources/>> and <<providers/>> sub-directories. For example, if a cookbook named <<example>> was downloaded to the chef-repo, it would be located at <</cookbooks/example/>>. If that cookbook contained two resources and two providers, the following files would be part of the <<resources/>> directory:

\begin{tabular}{ | l | l | l | }
  \hline
  Files	& Resource Name	& Generated Class \\
  \hline
  default.rb & example & Chef::Resource::Example \\
  custom.rb	& example\_custom & Chef::Resource::ExampleCustom \\
  \hline
\end{tabular}

And the following files would be part of the <<providers/>> directory:

\begin{tabular}{ | l | l | l | }
  \hline
  Files	& Provider Name	& Generated Class \\
  \hline
  default.rb & example & Chef::Provider::Example \\
  custom.rb	& custom & Chef::Provider::ExampleCustom \\
  \hline
\end{tabular}

Let's add in our <<my\_cool\_app>> LWRP, which will add in \inline{/etc/ssh/ssh_known_hosts} host. First of all we should create direcotry <<resources>> and add to it \inline{know_host.rb} file with content:

\begin{lstlisting}[label=lst:cookbook-lwrp1,title=my-server-cloud/site-cookbooks/my\_cool\_app/resources/know\_host.rb]
actions :create, :delete
default_action :create

attribute :host, :kind_of => String, :name_attribute => true, :required => true
attribute :key, :kind_of => String
attribute :port, :kind_of => Fixnum, :default => 22
attribute :known_hosts_file, :kind_of => String, :default => '/etc/ssh/ssh_known_hosts'

# Needed for Chef versions < 0.10.10
def initialize(*args)
  super
  @action = :create
end
\end{lstlisting}

Let's take it line by line. The first line specifies the allowed actions. Actions are what your resource can do, e.g. start, stop, create, delete, etc. In this case, you can \inline{:create} or \inline{:delete} known host. The next line defines the \inline{default_action} for our resource, in this case \inline{:create}. If you don't specify an action when you use the resource in a recipe, it will default to creating a known host, which is what you probably want. A general philosophy of Chef is to define intelligent or <<sane>> defaults.

Lines 4-7 define attributes, or properties of the known host resource we are creating. Line 4 defines an \inline{:host} attribute. Its \inline{:name_attribute} is true, which means that this attribute will be set to the string between \inline{my_cool_app_know_host} and \inline{do}. Example:

\begin{lstlisting}[label=lst:cookbook-lwrp2]
my_cool_app_know_host "Add github host" do
  host 'github.com'
end

my_cool_app_know_host 'github.com' do
  # The :host attribute will be set to 'github.com'
end
\end{lstlisting}

In the second example above, the \inline{:host} attribute wll be set to <<github.com>>.

Also, on line 4, we are definining the \inline{kind_of} validation parameter to tell the resource which kind of data we should expect (in this case, a string), whether this attribute is required (yes). Line 5 defines a \inline{:key} attribute, which is an optional string with no default. Line 6 defines a \inline{:port} attribute, a Ruby Fixnum (i.e. an integer) with a default of 22, which is the default when you create a known host. Line 7 defines a \inline{:known_hosts_file} attribute, a string with a default of \inline{/etc/ssh/ssh_known_hosts}, which is the default file with known hosts for ssh client.

Now we need to create file \inline{know_host.rb} in <<providers>> directory:

\begin{lstlisting}[label=lst:cookbook-lwrp3,title=my-server-cloud/site-cookbooks/my\_cool\_app/providers/know\_host.rb]
action :create do
  key = (new_resource.key || `ssh-keyscan -H -p #{new_resource.port} #{new_resource.host} 2>&1`)
  comment = key.split("\n").first || ""

  Chef::Application.fatal! "Could not resolve #{new_resource.host}" if key =~ /getaddrinfo/

  # Ensure that the file exists and has minimal content (required by Chef::Util::FileEdit)
  file new_resource.known_hosts_file do
    action        :create
    backup        false
    content       '# This file must contain at least one line. This is that line.'
    only_if do
      !::File.exists?(new_resource.known_hosts_file) || ::File.new(new_resource.known_hosts_file).readlines.length == 0
    end
  end

  # Use a Ruby block to edit the file
  ruby_block "add #{new_resource.host} to #{new_resource.known_hosts_file}" do
    block do
      file = ::Chef::Util::FileEdit.new(new_resource.known_hosts_file)
      file.insert_line_if_no_match(/#{Regexp.escape(comment)}|#{Regexp.escape(key)}/, key)
      file.write_file
    end
  end
  new_resource.updated_by_last_action(true)
end

action :delete do
  key = (new_resource.key || `ssh-keyscan -H -p #{new_resource.port} #{new_resource.host} 2>&1`)
  comment = key.split("\n").first || ""

  Chef::Application.fatal! "Could not resolve #{new_resource.host}" if key =~ /getaddrinfo/

  # Use a Ruby block to edit the file
  ruby_block "del #{new_resource.host} from #{new_resource.known_hosts_file}" do
    block do
      file = ::Chef::Util::FileEdit.new(new_resource.known_hosts_file)
      file.search_file_delete_line(/#{Regexp.escape(comment)}|#{Regexp.escape(key)}/, key)
      file.write_file
    end
    only_if do
      ::File.exists?(new_resource.known_hosts_file)
    end
  end
  new_resource.updated_by_last_action(true)
end
\end{lstlisting}

\begin{lstlisting}
http://docs.opscode.com/lwrp_custom_provider.html
http://dougireton.com/blog/2013/01/07/creating-an-lwrp-part-2/
https://github.com/opscode-cookbooks/ssh_known_hosts/blob/master/providers/entry.rb
\end{lstlisting}