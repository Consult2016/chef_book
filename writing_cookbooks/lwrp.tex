\section{LWRPs}
\label{sec:cookbook-lwrp}

A LWRP (Lightweight Resources and Providers) is a part of a cookbook that is used to extend the chef-client in a way that allows custom actions to be defined, and then used in recipes in much the same way as any platform resource. A LWRP has two principal components:

\begin{itemize}
  \item A lightweight resource that defines a set of actions and attributes
  \item A lightweight provider that tells the chef-client how to handle each action, what to do if certain conditions are met, and so on
\end{itemize}

In addition, most lightweight providers are built using platform resources and some lightweight providers are built using custom Ruby code.

Once created, a LWRP becomes a Ruby class within the organization. During each chef-client run, the chef-client will read the lightweight resources from recipes and process them alongside all of the other resources. When it is time to configure the node, the chef-client will use the corresponding lightweight provider to determine the steps required to bring the system into the desired state.

Where the lightweight resource represents a piece of the system, its current state, and the action that is needed to move it to the desired state, a lightweight provider defines the steps that are required to bring that piece of the system from its current state to the desired state. A LWRP behaves similar to platform resources and providers:

\begin{itemize}
  \item A lightweight resource is a key part of a recipe
  \item A lightweight resource defines the actions that can be taken
  \item During a chef-client run, each lightweight resource is identified, and then associated with a lightweight provider
  \item A lightweight provider does the work to complete the action requested by the lightweight resource
\end{itemize}

Lightweight resources and providers are loaded from files that are saved in the following cookbook sub-directories:

\begin{tabular}{ | l | l | }
  \hline
  Directory	& Description \\
  \hline
  providers/ & The sub-directory in which lightweight providers are located. \\
  resources/ & The sub-directory in which lightweight resources are located. \\
  \hline
\end{tabular}

The naming patterns of lightweight resources and providers are determined by the name of the cookbook and by the name of the files in the <<resources/>> and <<providers/>> sub-directories. For example, if a cookbook named <<example>> was downloaded to the chef-repo, it would be located at <</cookbooks/example/>>. If that cookbook contained two resources and two providers, the following files would be part of the <<resources/>> directory:

\begin{tabular}{ | l | l | l | }
  \hline
  Files	& Resource Name	& Generated Class \\
  \hline
  default.rb & example & Chef::Resource::Example \\
  custom.rb	& example\_custom & Chef::Resource::ExampleCustom \\
  \hline
\end{tabular}

And the following files would be part of the <<providers/>> directory:

\begin{tabular}{ | l | l | l | }
  \hline
  Files	& Provider Name	& Generated Class \\
  \hline
  default.rb & example & Chef::Provider::Example \\
  custom.rb	& custom & Chef::Provider::ExampleCustom \\
  \hline
\end{tabular}

Let's add in our <<my\_cool\_app>> LWRP, which will add in \inline!/etc/ssh/ssh_known_hosts! host.

\subsection{Resources}

First of all we should create directory <<resources>> and add to it \inline!know_host.rb! file with content:

\begin{lstlisting}[label=lst:cookbook-lwrp1,title=my-server-cloud/site-cookbooks/my\_cool\_app/resources/know\_host.rb]
actions :create, :delete
default_action :create

attribute :host, :kind_of => String, :name_attribute => true, :required => true
attribute :key, :kind_of => String
attribute :port, :kind_of => Fixnum, :default => 22
attribute :known_hosts_file, :kind_of => String, :default => '/etc/ssh/ssh_known_hosts'

# Needed for Chef versions < 0.10.10
def initialize(*args)
  super
  @action = :create
end
\end{lstlisting}

Let's take it line by line. The first line specifies the allowed actions. Actions are what your resource can do, e.g. start, stop, create, delete, etc. In this case, you can \inline!:create! or \inline!:delete! known host. The next line defines the \inline!default_action! for our resource, in this case \inline!:create!. If you don't specify an action when you use the resource in a recipe, it will default to creating a known host, which is what you probably want. A general philosophy of Chef is to define intelligent or <<sane>> defaults.

Lines 4-7 define attributes, or properties of the known host resource we are creating. Line 4 defines an \inline!:host! attribute. Its \inline!:name_attribute! is true, which means that this attribute will be set to the string between \inline!my_cool_app_know_host! and \inline!do!. Example:

\begin{lstlisting}[label=lst:cookbook-lwrp2]
my_cool_app_know_host "Add github host" do
  host 'github.com'
end

my_cool_app_know_host 'github.com' do
  # The :host attribute will be set to 'github.com'
end
\end{lstlisting}

In the second example above, the \inline!:host! attribute wll be set to <<github.com>>.

Also, on line 4, we are defining the \inline!kind_of! validation parameter to tell the resource which kind of data we should expect (in this case, a string), whether this attribute is required (yes). Line 5 defines a \inline!:key! attribute, which is an optional string with no default. Line 6 defines a \inline!:port! attribute, a Ruby Fixnum (i.e. an integer) with a default of 22, which is the default when you create a known host. Line 7 defines a \inline!:known_hosts_file! attribute, a string with a default of \inline!/etc/ssh/ssh_known_hosts!, which is the default file with known hosts for ssh client.

For example, the \inline!cron_d! lightweight resource (found in the cron cookbook) can be used to manage files located in \inline!/etc/cron.d!:

\begin{lstlisting}[label=lst:cookbook-lwrp3]
actions :create, :delete
default_action :create

attribute :name, :kind_of => String, :name_attribute => true
attribute :cookbook, :kind_of => String, :default => "cron"
attribute :minute, :kind_of => [Integer, String], :default => "*"
attribute :hour, :kind_of => [Integer, String], :default => "*"
attribute :day, :kind_of => [Integer, String], :default => "*"
attribute :month, :kind_of => [Integer, String], :default => "*"
attribute :weekday, :kind_of => [Integer, String], :default => "*"
attribute :command, :kind_of => String, :required => true
attribute :user, :kind_of => String, :default => "root"
attribute :mailto, :kind_of => [String, NilClass]
attribute :path, :kind_of => [String, NilClass]
attribute :home, :kind_of => [String, NilClass]
attribute :shell, :kind_of => [String, NilClass]
\end{lstlisting}

where

\begin{itemize}
  \item the \inline!actions! allow a recipe to manage entries in a crontab file (create entry, delete entry)
  \item \inline!:create! is the default action
  \item \inline!:minute!, \inline!:hour!, \inline!:day!, \inline!:month!, and \inline!:weekday! are the collection of attributes used to schedule a cron job, assigned a default value of <<*>>
  \item \inline!:command! is the command that will be run (and also required)
  \item \inline!:user! is the user by which the command is run
  \item \inline!:mailto!, \inline!:path!, \inline!:home!, and \inline!:shell! are optional environment variables that do not have default value, which each being defined as an array that supports the String and NilClass Ruby classes
\end{itemize}

\subsection{Providers}

Now we need to create file \inline!know_host.rb! in <<providers>> directory:

\begin{lstlisting}[label=lst:cookbook-lwrp4,title=my-server-cloud/site-cookbooks/my\_cool\_app/providers/know\_host.rb]
# Support whyrun
def whyrun_supported?
  true
end

use_inline_resources

action :create do
  key, comment = insure_for_file(new_resource)
  # Use a Ruby block to edit the file
  ruby_block "add #{new_resource.host} to #{new_resource.known_hosts_file}" do
    block do
      file = ::Chef::Util::FileEdit.new(new_resource.known_hosts_file)
      file.insert_line_if_no_match(/#{Regexp.escape(comment)}|#{Regexp.escape(key)}/, key)
      file.write_file
    end
  end
  new_resource.updated_by_last_action(true)
end

action :delete do
  key, comment = insure_for_file(new_resource)
  # Use a Ruby block to edit the file
  ruby_block "del #{new_resource.host} from #{new_resource.known_hosts_file}" do
    block do
      file = ::Chef::Util::FileEdit.new(new_resource.known_hosts_file)
      file.search_file_delete_line(/#{Regexp.escape(comment)}|#{Regexp.escape(key)}/)
      file.write_file
    end
  end
  new_resource.updated_by_last_action(true)
end

def insure_for_file(new_resource)
  key = (new_resource.key || `ssh-keyscan -H -p #{new_resource.port} #{new_resource.host} 2>&1`)
  comment = key.split("\n").first || ""

  Chef::Application.fatal! "Could not resolve #{new_resource.host}" if key =~ /getaddrinfo/

  # Ensure that the file exists and has minimal content (required by Chef::Util::FileEdit)
  file new_resource.known_hosts_file do
    action        :create
    backup        false
    content       '# This file must contain at least one line. This is that line.'
    only_if do
      !::File.exists?(new_resource.known_hosts_file) || ::File.new(new_resource.known_hosts_file).readlines.length == 0
    end
  end
  [key, comment]
end
\end{lstlisting}

\subsection{DSL Methods}

\subsubsection{action}

The action method is used to define the steps that will be taken for each of the possible actions defined by the lightweight resource. Each action must be defined in separate action blocks within the same file. The syntax for the action method is as follows:

\begin{lstlisting}[label=lst:cookbook-lwrp-dsl-action]
action :action_name do
  if @current_resource.exists
    Chef::Log.info "#{ @new_resource } already exists - nothing to do."
  else
    resource "resource_name" do
      Chef::Log.info "#{ @new_resource } created."
    end
  end
  new_resource.updated_by_last_action(true)
end
\end{lstlisting}

where:

\begin{itemize}
  \item \inline!:action_name! corresponds to an action defined by a lightweight resource
  \item if \inline!@current_resource.exists! is a condition test that is using an instance variable to see if the object already exists on the node; this is an example of a test for idempotence
  \item If the object already exists, a \inline!@new_resource! already exists - nothing to do. log entry is created
  \item If the object does not already exists, the resource block is run. This block is a recipe that tells the chef-client what to do. A \inline!@new_resource! created. log entry is created
\end{itemize}

\subsubsection{current\_resource}

The \inline!current_resource! method is used to represent a resource as it exists on the node at the beginning of the chef-client run. In other words: what the resource is currently. The chef-client compares the resource as it exists on the node to the resource that is created during the chef-client run to determine what steps need to be taken to bring the resource into the desired state. This method is often used as an instance variable (\inline!@current_resource!).

For example:

\begin{lstlisting}[label=lst:cookbook-lwrp-dsl-current-resource]
action :add do
  unless @current_resource.exists
    cmd = "#{appcmd} add app /site.name:\"#{@new_resource.app_name}\""
    cmd << " /path:\"#{@new_resource.path}\""
    cmd << " /applicationPool:\"#{@new_resource.application_pool}\"" if @new_resource.application_pool
    cmd << " /physicalPath:\"#{@new_resource.physical_path}\"" if @new_resource.physical_path
    Chef::Log.debug(cmd)
    shell_out!(cmd)
    Chef::Log.info("App created")
  else
    Chef::Log.debug("#{@new_resource} app already exists - nothing to do")
  end
end
\end{lstlisting}

where the \inline!unless! conditional statement checks to make sure the resource doesn't already exist on a node, and then runs a series of commands when it doesn't. If the resource already exists, the log entry would be <<Foo app already exists - nothing to do.>>

\subsubsection{load\_current\_resource}

The \inline!load_current_resource! method is used to find a resource on a node based on a collection of attributes. These attributes are defined in a lightweight resource and are loaded by the chef-client when processing a recipe during a chef-client run. This method will ask the chef-client to look on the node to see if a resource exists with specific matching attributes.

For example:

\begin{lstlisting}[label=lst:cookbook-lwrp-dsl-load-current-resource]
def load_current_resource
  @current_resource = Chef::Resource::TransmissionTorrentFile.new(@new_resource.name)
  Chef::Log.debug("#{@new_resource} torrent hash = #{torrent_hash}")
  path = "foo:#{@new_resource.att1}@#{@new_resource.att2}:#{@new_resource.att3}/path"
  @transmission = Opscode::Transmission::Client.new(path)
  @torrent = nil
  begin
    @torrent = @transmission.get_torrent(torrent_hash)
    info = "Found existing #{@new_resource} in swarm "+
    "with name of '#{@torrent.name}' and status of '#{@torrent.status_message}'"
    Chef::Log.info(info)
    @current_resource.torrent(@new_resource.torrent)
  rescue
    Chef::Log.debug("Cannot find #{@new_resource} in the swarm")
  end
  @current_resource
end
\end{lstlisting}

In the previous example, if a resource exists with matching attributes, the chef-client does nothing and if a resource does not exist with matching attributes, the chef-client will enforce the state declared in \inline!new_resource!.

\subsubsection{new\_resource}

The \inline!new_resource! method is used to represent a resource as loaded by the chef-client during the chef-client run. In other words: what the resource should be. The chef-client compares the resource as it exists on the node to the resource that is created during the chef-client run to determine what steps need to be taken to bring the resource into the desired state.

For example:

\begin{lstlisting}[label=lst:cookbook-lwrp-dsl-new-resource]
action :delete do
  if exists?
    if ::File.writable?(new_resource.path)
      Chef::Log.info("Deleting #{new_resource} at #{new_resource.path}")
      ::File.delete(new_resource.path)
      new_resource.updated_by_last_action(true)
    else
      raise "Cannot delete #{new_resource} at #{new_resource.path}!"
    end
  end
end
\end{lstlisting}

where the chef-client checks to see if the file exists, then if the file is writable, and then attempts to delete the resource. \inline!path! is an attribute of the new resource that is defined by the lightweight resource.

\subsubsection{updated\_by\_last\_action}

The \inline!updated_by_last_action! method is used to notify a lightweight resource that a node was updated successfully. For example, the \inline!cron_d! lightweight resource in the cron cookbook:

\begin{lstlisting}[label=lst:cookbook-lwrp-dsl-updated]
action :create do
  t = template "/etc/cron.d/#{new_resource.name}" do
    cookbook new_resource.cookbook
    source "cron.d.erb"
    mode "0644"
    variables({
        :name => new_resource.name,
        :minute => new_resource.minute,
        :hour => new_resource.hour,
        :day => new_resource.day,
        :month => new_resource.month,
        :weekday => new_resource.weekday,
        :command => new_resource.command,
        :user => new_resource.user,
        :mailto => new_resource.mailto,
        :path => new_resource.path,
        :home => new_resource.home,
        :shell => new_resource.shell
      })
    action :create
  end
  new_resource.updated_by_last_action(t.updated_by_last_action?)
end
\end{lstlisting}

where \inline!t.updated_by_last_action?! uses a variable to check whether a new crontab entry was created.

\subsubsection{use\_inline\_resources}

A lightweight resource should be set to inline compile mode by adding the \inline!use_inline_resources! method at the top of the provider. This ensures that notifications work properly across the resource collection. The \inline!use_inline_resources! method was added to the chef-client starting in version 11.0 to address the behavior described below. The \inline!use_inline_resources! method should be considered a requirement for any lightweight resource authored against the 11.0+ versions of the chef-client. This behavior will become the default behavior in an upcoming version of the chef-client.

The reason why the \inline!use_inline_resources method! exists at all is due to how the chef-client processes resources. Currently, the default behavior of the chef-client processes a single collection of resources, converged on the node in order.

A lightweight resource is often implemented using the core chef-client resources — \inline!file!, \inline!template!, \inline!package!, and so on—as building blocks. A lightweight resource is then added to a recipe using the short name of the lightweight resource in the recipe (and not by using any of the building block resource components).

This situation can create problems with notifications because the chef-client includes embedded resources in the <<single collection of resources>> after the parent resource has been fully evaluated.

For example:

\begin{lstlisting}[label=lst:cookbook-lwrp-inline-resources1]
custom_resource "something" do
  action :run
  notifies :restart, "service[whatever]", :immediately
end

service "whatever" do
  action :nothing
end
\end{lstlisting}

If the \inline!custom_resource! is built using the file resource, what happens during the chef-client run is:

\begin{lstlisting}[language=Bash,label=lst:cookbook-lwrp-inline-resources2]
custom_resource (not updated)
  file (updated)
service (skipped, due to ``:nothing``)
\end{lstlisting}

The \inline!custom_resource! is converged completely, its state set to not updated before the \inline!file! resource is evaluated. The \inline!notifies :restart! is ignored and the service is not restarted.

If the author of the custom resource knows in advance what notification is required, then the \inline!file! resource can be configured for the notification in the provider. For example:

\begin{lstlisting}[label=lst:cookbook-lwrp-inline-resources3]
action :run do
  file "/tmp/foo" do
    owner "root"
    group "root"
    mode "0644"
    notifies :restart, "service[whatever]", :immediately
  end
end
\end{lstlisting}

And then in the recipe:

\begin{lstlisting}[label=lst:cookbook-lwrp-inline-resources4]
service "whatever" do
  action :nothing
end
\end{lstlisting}

This approach works, but only when the author of the lightweight resource knows what should be notified in advance of the chef-client run. Consequently, this is less-than-ideal for most situations.

Using the \inline!use_inline_resources! method will ensure that the chef-client processes a lightweight resource as if it were its own resource collection—a <<mini chef-client run>>, effectively—that is converged before the chef-client finishes evaluating the parent lightweight resource. This ensures that any notifications that may exist in the embedded resources are processed as if they were notifications on the parent lightweight resource. For example:

\begin{lstlisting}[label=lst:cookbook-lwrp-inline-resources5]
custom_resource "something" do
  action :run
  notifies :restart, "service[whatever]", :immediately
end

service "whatever" do
  action :nothing
end
\end{lstlisting}

If the \inline!custom_resource! is built using the \inline!file! resource, what happens during the chef-client run is:

\begin{lstlisting}[language=Bash,label=lst:cookbook-lwrp-inline-resources6]
custom_resource (starts converging)
  file (updated)
custom_resource (updated, because ``file`` updated)
service (updates, because ``:immediately`` is set in the custom resource)
\end{lstlisting}

The \inline!use_inline_resources! method should be considered a default method for any provider that defines a custom resource. It's the correct behavior. And it will soon become the default behavior in a future version of the chef-client.

Because inline compile mode makes it impossible for embedded resources to notify resources in the parent resource collection, inline compile mode may cause issues with some provider implementations. In these cases, use a definition to work around inline compile mode.

\subsubsection{whyrun\_supported?}

why-run mode is a way to see what the chef-client would have configured, had an actual chef-client run occurred. This approach is similar to the concept of <<no-operation>> (or <<no-op>>): decide what should be done, but then don't actually do anything until it's done right. This approach to configuration management can help identify where complexity exists in the system, where inter-dependencies may be located, and to verify that everything will be configured in the desired manner.

When why-run mode is enabled, a chef-client run will occur that does everything up to the point at which configuration would normally occur. This includes getting the configuration data, authenticating to the server, rebuilding the node object, expanding the run list, getting the necessary cookbook files, resetting node attributes, identifying the resources, and building the resource collection and does not include mapping each resource to a provider or configuring any part of the system.

When the chef-client is run in why-run mode, certain assumptions are made:

\begin{itemize}
  \item If the service resource cannot find the appropriate command to verify the status of a service, why-run mode will assume that the command would have been installed by a previous resource and that the service would not be running
  \item For \inline!not_if! and \inline!only_if! attribute, why-run mode will assume these are commands or blocks that are safe to run. These conditions are not designed to be used to change the state of the system, but rather to help facilitate idempotency for the resource itself. That said, it may be possible that these attributes are being used in a way that modifies the system state
  \item The closer the current state of the system is to the desired state, the more useful why-run mode will be. For example, if a full run-list is run against a fresh system, that run-list may not be completely correct on the first try, but also that run-list will produce more output than smaller run-list
\end{itemize}

The \inline!whyrun_supported?! method is used to set a lightweight provider to support why-run mode. The syntax for the \inline!whyrun_supported?! method is as follows:

\begin{lstlisting}[label=lst:cookbook-lwrp-why-run]
def whyrun_supported?
  true
end
\end{lstlisting}

where \inline!whyrun_supported?! is set to true for any lightweight provider that supports using why-run mode. When why-run mode is supported by the a lightweight provider, the \inline!converge_by! method is used to define the strings that are logged by the chef-client when it is run in why-run mode.

\subsubsection{Log entries and rescue}

Use the \inline!Chef::Log! class in a lightweight provider to define log entries that are created during a chef-client run. The syntax for a log message is as follows:

\begin{lstlisting}[label=lst:cookbook-lwrp-logs1]
Chef::Log.log_type("message")
\end{lstlisting}

where

\begin{itemize}
  \item \inline!log_type! can be .debug, .info, .warn, .error, or .fatal
  \item <<message>> is the message that is logged
\end{itemize}

For example, from the <<repository.rb>> provider in the yum cookbook:

\begin{lstlisting}[label=lst:cookbook-lwrp-logs2]
action :add do
  unless ::File.exists?("/etc/yum.repos.d/#{new_resource.repo_name}.repo")
    Chef::Log.info "Adding #{new_resource.repo_name} repository to /etc/yum.repos.d/#{new_resource.repo_name}.repo"
    repo_config
  end
end
\end{lstlisting}

where the \inline!Chef::Log! class appends .info as the log type. If the name of the repo was <<foo>>, then the log message would be <<Adding foo repository to /etc/yum.repos.d/foo.repo>>.

Another example shows two log entries, one that is triggered when a service is being restarted, and then another that is triggered after the service has been restarted:

\begin{lstlisting}[label=lst:cookbook-lwrp-logs3]
action :restart do
  if @current_resource.running
    Chef::Log.debug "Restarting #{new_resource.service_name}"
    shell_out!(restart_command)
    new_resource.updated_by_last_action(true)
    Chef::Log.debug "Restarted #{new_resource.service_name}"
  end
end
\end{lstlisting}

Use the \inline!rescue! clause to make sure that a log message is always provided. For example:

\begin{lstlisting}[label=lst:cookbook-lwrp-logs4]
def load_current_resource
  ...
  begin
    ...
  rescue
    Chef::Log.debug("Cannot find #{@new_resource} in the swarm")
  end
  ...
end
\end{lstlisting}

\subsection{Using LWRP}

We finished with our LWRP. Let's test it. Just add to <<default.rb>> you new LWRP, which will add to known hosts <<github.com>>:

\begin{lstlisting}[label=lst:cookbook-lwrp5,title=my-server-cloud/site-cookbooks/my\_cool\_app/recipes/default.rb]
# known hosts for github.com
my_cool_app_know_host 'github.com'
\end{lstlisting}

Upload cookbook on server and run chef-client on node:

\begin{lstlisting}[language=Bash,label=lst:cookbook-lwrp6]
$ knife cookbook upload my_cool_app
  Uploading my_cool_app    [0.1.0]
  Uploaded 1 cookbook.

$ vagrant provision chef_second_client
...
INFO: Found chef-client in /usr/bin/chef-client
INFO: runit_service[nginx] configured
INFO: ruby_block[add github.com to /etc/ssh/ssh_known_hosts] called
INFO: Chef Run complete in 11.920336698 seconds
INFO: Running report handlers
INFO: Report handlers complete

$ vagrant ssh chef_second_client

$ cat /etc/ssh/ssh_known_hosts
# This file must contain at least one line. This is that line.
# github.com SSH-2.0-OpenSSH_5.9p1 Debian-5ubuntu1+github5
|1|fvGKZG+jIkEntM5yBvzJ230TX1o=|9qP2wRdFIS+cAouirLYDb1Ibl7A= ssh-rsa...
\end{lstlisting}

Let's check how it will delete this hosts:

\begin{lstlisting}[label=lst:cookbook-lwrp7,title=my-server-cloud/site-cookbooks/my\_cool\_app/recipes/default.rb]
my_cool_app_know_host 'github.com' do
  action :delete
end
\end{lstlisting}

Again upload on server and run chef-client:

\begin{lstlisting}[language=Bash,label=lst:cookbook-lwrp8]
$ knife cookbook upload my_cool_app
  Uploading my_cool_app    [0.1.0]
  Uploaded 1 cookbook.

$ vagrant provision chef_second_client
...
INFO: Found chef-client in /usr/bin/chef-client
INFO: runit_service[nginx] configured
INFO: ruby_block[del github.com from /etc/ssh/ssh_known_hosts] called
INFO: Chef Run complete in 17.109379291 seconds
INFO: Running report handlers
INFO: Report handlers complete

$ vagrant ssh chef_second_client

$ cat /etc/ssh/ssh_known_hosts
# This file must contain at least one line. This is that line.
\end{lstlisting}

As you can see, all works as expected.
