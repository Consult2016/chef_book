\section{Metadata}

Metadata is file, which contain all main information about cookbook. Let's consider our generated example:

\begin{lstlisting}[label=lst:cookbook-metadata1,title=my-server-cloud/site-cookbooks/my\_cool\_app/metadata.rb]
name             'my_cool_app'
maintainer       'YOUR_NAME'
maintainer_email 'YOUR_EMAIL'
license          'All rights reserved'
description      'Installs/Configures my_cool_app'
long_description IO.read(File.join(File.dirname(__FILE__), 'README.md'))
version          '0.1.0'
\end{lstlisting}

This file written on Ruby and can have such settings:

\begin{itemize}
  \item \textbf{name} - the name of the cookbook
  \item \textbf{maintainer} - the name of the person responsible for maintaining a cookbook, either an individual or an organization
  \item \textbf{maintainer\_email} - the email address for the person responsible for maintaining a cookbook. Only one email can be listed here
  \item \textbf{license} - the type of license under which a cookbook is distributed: <<Apache v2.0>>, <<GPL v2>>, <<GPL v3>>, <<MIT>>, or license <<Proprietary - All Rights Reserved>> (default)
  \item \textbf{description} - a short description of a cookbook and its functionality
  \item \textbf{long\_description} - a longer description that ideally contains full instructions on the proper use of a cookbook, including definitions, libraries, dependencies, and so on. In example the contents pulled from <<README.md>> file
  \item \textbf{version} - the current version of a cookbook. Version numbers always follow a simple three-number version sequence
\end{itemize}
