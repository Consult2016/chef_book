\section{Metadata}

Metadata is file, which contain all main information about cookbook. Let's consider our generated example:

\begin{lstlisting}[label=lst:cookbook-metadata1,title=my-server-cloud/site-cookbooks/my\_cool\_app/metadata.rb]
name             'my_cool_app'
maintainer       'YOUR_NAME'
maintainer_email 'YOUR_EMAIL'
license          'All rights reserved'
description      'Installs/Configures my_cool_app'
long_description IO.read(File.join(File.dirname(__FILE__), 'README.md'))
version          '0.1.0'
\end{lstlisting}

This file written on Ruby and can have such settings:

\begin{itemize}
  \item \textbf{name} - the name of the cookbook
  \item \textbf{maintainer} - the name of the person responsible for maintaining a cookbook, either an individual or an organization
  \item \textbf{maintainer\_email} - the email address for the person responsible for maintaining a cookbook. Only one email can be listed here
  \item \textbf{license} - the type of license under which a cookbook is distributed: <<Apache v2.0>>, <<GPL v2>>, <<GPL v3>>, <<MIT>>, or license <<Proprietary - All Rights Reserved>> (default)
  \item \textbf{description} - a short description of a cookbook and its functionality
  \item \textbf{long\_description} - a longer description that ideally contains full instructions on the proper use of a cookbook, including definitions, libraries, dependencies, and so on. In example the contents pulled from <<README.md>> file
  \item \textbf{version} - the current version of a cookbook. Version numbers always follow a simple three-number version sequence
  \item \textbf{attribute} - the list of attributes that are required to configure a cookbook
  \item \textbf{depends} - indicates that a cookbook has a dependency on another cookbook
  \item \textbf{recommends} - adds a dependency on another cookbook that is recommended, but not required
  \item \textbf{suggests} - adds a dependency on another cookbook that is suggested, but not required
  \item \textbf{conflicts} - indicates that a cookbook conflicts with another cookbook or cookbook version
  \item \textbf{grouping} - adds a title and description to a group of attributes within a namespace
  \item \textbf{provides} - adds a recipe, definition, or resource that is provided by this cookbook, should the auto-populated list be insufficient
  \item \textbf{recipe} - a description for a recipe, mostly for cosmetic value within the server user interface
  \item \textbf{replaces} - indicates that this cookbook should replace another (and can be used in-place of that cookbook)
  \item \textbf{supports} - indicates that a cookbook has a supported platform
\end{itemize}

For our cookbook we need little modify it:

\begin{lstlisting}[label=lst:cookbook-metadata2,title=my-server-cloud/site-cookbooks/my\_cool\_app/metadata.rb]
name             'my_cool_app'
maintainer       'Alexey Vasiliev'
maintainer_email 'leopard_ne@inbox.ru'
license          'MIT'
description      'Installs/Configures my_cool_app'
long_description IO.read(File.join(File.dirname(__FILE__), 'README.md'))
version          '0.1.0'
\end{lstlisting}

As of writing this cookbook, we will be adding information to this file on it.
