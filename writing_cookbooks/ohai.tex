\section{Ohai}
\label{sec:cookbook-ohai}

Ohai detects data about your operating system. It can be used standalone, but its primary purpose is to provide node data to Chef.

When invoked, it collects detailed, extensible information about the machine it's running on, including Chef configuration, hostname, FQDN, networking, memory, CPU, platform, and kernel data.

When Chef configures the node object during each Chef run, these attributes are used by the chef-client to ensure that certain properties remain unchanged. These properties are also referred to as automatic attributes (which, as you remember, impossible to override by attributes from cookbooks, environments, roles and nodes). For example:

\begin{lstlisting}[language=Bash,label=lst:cookbook-ohai1]
node['platform'] # The platform on which a node is running. This attribute helps determine which providers will be used.
node['platform_version']  # The version of the platform. This attribute helps determine which providers will be used.
node['hostname']  # The host name for the node.
\end{lstlisting}

\subsection{Example}

Let's create new recipe, which will use Ohai attributes and create our own Ohai attributes.

\subsubsection{Recipe node}

First of all, we will create new recipe <<node>>, which will install \href{http://nodejs.org/}{Node.js} on nodes in our \lstinline!my_cool_app! cookbook.

New default attributes:

\begin{lstlisting}[label=lst:cookbook-ohai-attributes,title=my-server-cloud/site-cookbooks/my\_cool\_app/attributes/node.rb]
default['my_cool_app']['nodejs']['version'] = '0.10.26'
default['my_cool_app']['nodejs']['checksum'] = '2340ec2dce1794f1ca1c685b56840dd515a271b2'
default['my_cool_app']['nodejs']['dir'] = '/usr/local'
default['my_cool_app']['nodejs']['src_url'] = "http://nodejs.org/dist"
\end{lstlisting}

And recipe:

\begin{lstlisting}[label=lst:cookbook-ohai-recipe-before,title=my-server-cloud/site-cookbooks/my\_cool\_app/recipes/node.rb]
include_recipe "build-essential"

case node['platform_family']
  when 'rhel','fedora'
    package "openssl-devel"
  when 'debian'
    package "libssl-dev"
end

nodejs_tar = "node-v#{node['my_cool_app']['nodejs']['version']}.tar.gz"
nodejs_tar_path = nodejs_tar
if node['my_cool_app']['nodejs']['version'].split('.')[1].to_i >= 5
  nodejs_tar_path = "v#{node['my_cool_app']['nodejs']['version']}/#{nodejs_tar_path}"
end
# Let the user override the source url in the attributes
nodejs_src_url = "#{node['my_cool_app']['nodejs']['src_url']}/#{nodejs_tar_path}"

remote_file "/usr/local/src/#{nodejs_tar}" do
  source nodejs_src_url
  checksum node['my_cool_app']['nodejs']['checksum']
  mode 0644
  action :create_if_missing
end

# --no-same-owner required overcome "Cannot change ownership" bug
# on NFS-mounted filesystem
execute "tar --no-same-owner -zxf #{nodejs_tar}" do
  cwd "/usr/local/src"
  creates "/usr/local/src/node-v#{node['my_cool_app']['nodejs']['version']}"
end

bash "compile node.js" do
  cwd "/usr/local/src/node-v#{node['my_cool_app']['nodejs']['version']}"
  code <<-EOH
    PATH="/usr/local/bin:$PATH"
    ./configure --prefix=#{node['my_cool_app']['nodejs']['dir']} && \
    make
  EOH
  creates "/usr/local/src/node-v#{node['my_cool_app']['nodejs']['version']}/node"
end

execute "nodejs make install" do
  environment({"PATH" => "/usr/local/bin:/usr/bin:/bin:$PATH"})
  command "make install"
  cwd "/usr/local/src/node-v#{node['my_cool_app']['nodejs']['version']}"
  not_if do
    File.exists?("#{node['my_cool_app']['nodejs']['dir']}/bin/node") &&
    `#{node['my_cool_app']['nodejs']['dir']}/bin/node --version`.chomp == "v#{node['my_cool_app']['nodejs']['version']}"
  end
end
\end{lstlisting}

As you can see, we use \lstinline!node['platform_family']! Ohai variable, which help for us understand type of OS on node and install needed package.

Also we used cookbook \lstinline!build-essential!, because we will install node.js from source. In this case we should add it as dependency in our metadata.rb file:

\begin{lstlisting}[label=lst:cookbook-ohai-meta]
depends 'nginx',           '~> 2.2.0'
depends 'build-essential'
\end{lstlisting}

To use this recipe we should add it in \lstinline!run_list!:

\begin{lstlisting}[language=JSON,label=lst:cookbook-ohai2]
{
  "run_list": [
    "recipe[my_cool_app]",
    "recipe[my_cool_app::node]"
  ]
}
\end{lstlisting}

But I want run this recipe with default recipe, so I will leave \lstinline!run_list! as is and add node recipe in default recipe:

\begin{lstlisting}[label=lst:cookbook-ohai3]
...
include_recipe 'my_cool_app::node'
\end{lstlisting}

Now it remains to check how it works:

\begin{lstlisting}[language=Bash,label=lst:cookbook-ohai4]
$ knife cookbook upload my_cool_app
Uploading my_cool_app    [0.1.0]
Uploaded 1 cookbook.

$ vagrant provision chef_second_client
INFO: remote_file[/usr/local/src/node-v0.10.26.tar.gz] created file /usr/local/src/node-v0.10.26.tar.gz
INFO: remote_file[/usr/local/src/node-v0.10.26.tar.gz] updated file contents /usr/local/src/node-v0.10.26.tar.gz
INFO: remote_file[/usr/local/src/node-v0.10.26.tar.gz] mode changed to 644
INFO: execute[tar --no-same-owner -zxf node-v0.10.26.tar.gz] ran successfully
INFO: bash[compile node.js] ran successfully
INFO: execute[nodejs make install] ran successfully

$ vagrant ssh chef_second_client

vagrant@precise64:~$ node -v
v0.10.26
\end{lstlisting}

As you can see node recipe installed Node.js on our node.

\subsubsection{Ohai plugin}

In our node recipe we used maybe not best way to check node version, which already installed on node (if version mismatch - we should install needed). Let's create Ohai plugin, which will give for use node.js version from server. First of all create in \lstinline!my_cool_app! new recipe \lstinline!ohai_plugin.rb! with content:

\begin{lstlisting}[label=lst:cookbook-ohai-recipe1,title=my-server-cloud/site-cookbooks/my\_cool\_app/recipe/ohai\_plugin.rb]
template "#{node['ohai']['plugin_path']}/system_node_js.rb" do
  source "plugins/system_node_js.rb.erb"
  owner "root"
  group "root"
  mode 00755
  variables(
    :node_js_bin => "#{node['my_cool_app']['nodejs']['dir']}/bin/node"
  )
end

include_recipe "ohai"
\end{lstlisting}

This recipe will generate ohai plugin from \lstinline!system_node_js.rb.erb! template. Next we should create this template in folder <<templates/default/plugins>>:

\begin{lstlisting}[label=lst:cookbook-ohai-template1,title=my-server-cloud/site-cookbooks/my\_cool\_app/templates/default/plugin/system\_node\_js.rb.erb]
provides "system_node_js"
provides "system_node_js/version"

system_node_js Mash.new unless system_node_js
system_node_js[:version] = nil unless system_node_js[:version]

status, stdout, stderr = run_command(:no_status_check => true, :command => "<%= @node_js_bin %> --version")

system_node_js[:version] = stdout[1..-1] if 0 == status
\end{lstlisting}

In first two lines we set by method \lstinline!provides! automatic attributes, which will provide for us this plugin.
Most of the information we want to lookup would be nested in some way, and ohai tends to do this by storing the data in a Mash (similar to Ruby hash type). This can be done by creating a new mash and setting the attribute to it. We did this with \lstinline!system_node_js!. In the end of code, plugin set the version of node.js, if node.js installed on node. That's it!

Also we should add new dependency for our cookbook - ohai cookbook:

\begin{lstlisting}[label=lst:cookbook-ohai5]
depends 'nginx',           '~> 2.2.0'
depends 'build-essential'
depends 'ohai'
\end{lstlisting}

Next, let's try this plugin by adding node.rb recipe this content:

\begin{lstlisting}[label=lst:cookbook-ohai6]
include_recipe "build-essential"

include_recipe "my_cool_app::ohai_plugin"
Chef::Log.info "Installed Node version: #{node['system_node_js']['version']}" if node['system_node_js']

case node['platform_family']
  when 'rhel','fedora'
    package "openssl-devel"
  when 'debian'
    package "libssl-dev"
end
\end{lstlisting}

In this case we can little change our node.js recipe:

\begin{lstlisting}[label=lst:cookbook-ohai7]
execute "nodejs make install" do
  environment({"PATH" => "/usr/local/bin:/usr/bin:/bin:$PATH"})
  command "make install"
  cwd "/usr/local/src/node-v#{node['my_cool_app']['nodejs']['version']}"
  not_if {node['system_node_js'] && node['system_node_js']['version'] == node['my_cool_app']['nodejs']['version'] }
end
\end{lstlisting}

Now we can check how it works:

\begin{lstlisting}[language=Bash,label=lst:cookbook-ohai8]
$ knife cookbook upload my_cool_app
Uploading my_cool_app    [0.1.0]
Uploaded 1 cookbook.

$ vagrant provision chef_second_client
...
INFO: template[/etc/chef/ohai_plugins/system_node_js.rb] created file /etc/chef/ohai_plugins/system_node_js.rb
INFO: template[/etc/chef/ohai_plugins/system_node_js.rb] updated file contents /etc/chef/ohai_plugins/system_node_js.rb
INFO: template[/etc/chef/ohai_plugins/system_node_js.rb] owner changed to 0
INFO: template[/etc/chef/ohai_plugins/system_node_js.rb] group changed to 0
INFO: template[/etc/chef/ohai_plugins/system_node_js.rb] mode changed to 755
...
$ vagrant provision chef_second_client
...
WARN: Current  service[nginx]: /var/chef/cache/cookbooks/nginx/recipes/source.rb:123:in `from_file'
INFO: Installed Node version: 0.10.26
WARN: Cloning resource attributes for package[libssl-dev] from prior resource (CHEF-3694)
...
\end{lstlisting}

As you can see, after first \lstinline!provision! (\lstinline!knife cook!, \lstinline!knife ssh 'sudo chef-client'!) execution Ohai plugin will be installed, but not executed. Only on second execution chef-client will use Ohai plugin (because Ohai plugins load before running \lstinline!run_list!).

\subsubsection{Ohai 7}

Ohai 6 had served us well. However, it had an important architectural limitation that prevented us from implementing some cool ideas, such as differentiating collected data as critical or optional. This limitation was because Ohai 6 treated plugins as monolithic blocks of code.

Ohai 7 introduces a new DSL, which makes it easier to write custom plugins, provides better code organization, and sets us up for the future. Here is what an Ohai 7 plugin looks like:

\begin{lstlisting}[label=lst:cookbook-ohai9]
Ohai.plugin(:Name) do
  provides "attribute", "attribute/subattribute"
  depends "kernel", "users"

  def a_shared_method
    # some Ruby code that defines the shared method
    attribute Mash.new
  end

  collect_data(:default) do
    # some Ruby code
    attribute Mash.new
  end

  collect_data(:windows) do
    # some Ruby code that gets run only on Windows
    attribute Mash.new
  end

end
\end{lstlisting}

Two important pieces of the new DSL are:

\begin{itemize}
  \item \lstinline!collect_data()! blocks, which enable better organization of platform-specific code
  \item \lstinline!depends! / \lstinline!provides! statements, which enable easier dependency management among plugins
\end{itemize}

Read more about the new DSL \href{http://docs.opscode.com/ohai_custom.html}{here}.

To migrate our plugin on Ohai 7 is very simple:

\begin{lstlisting}[label=lst:cookbook-ohai10,title=my-server-cloud/site-cookbooks/my\_cool\_app/templates/default/plugin/system\_node\_js.rb.erb]
Ohai.plugin(:SystemNodeJs) do
  provides 'system_node_js'
  provides 'system_node_js/version'

  collect_data do
    system_node_js Mash.new
    system_node_js[:version] = nil unless system_node_js[:version]
    status, stdout, stderr = run_command(:no_status_check => true, :command => "<%= @node_js_bin %> --version")
    system_node_js[:version] = stdout[1..-1] if 0 == status
  end
end
\end{lstlisting}

Ohai 7 is backwards compatible with existing Ohai 6 plugins. But none of the new (or future) functionality will be available to version 6 plugins. All of your existing plugins will continue to work with Ohai 7.