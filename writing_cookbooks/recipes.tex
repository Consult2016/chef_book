\section{Recipes}
\label{sec:cookbook-recipes}

Any cookbook contains recipes. The default recipe inside cookbook have name <<default>>. Let's add our default recipe, which will install \href{http://git-scm.com/}{git}:

\begin{lstlisting}[label=lst:cookbook-recipes1,title=my-server-cloud/site-cookbooks/my\_cool\_app/recipes/default.rb]
#
# Cookbook Name:: my_cool_app
# Recipe:: default
#
# Copyright (C) 2014 Alexey Vasiliev
#
# MIT
#

package 'git'
\end{lstlisting}

As you can see, at the beginning of recipe we have comments about this recipe. Next we add resource <<package>> with argument <<git>>. The <<package>> resource is used to manage packages on the system. For example, on Debian or Ubuntu resource <<package>> will use <<apt-get>> command to install git on system.

Now you should add <<my\_cool\_app>> into run-list to use this cookbook:

\begin{lstlisting}[label=lst:cookbook-recipes2,title=my-server-cloud/nodes/second.example.com.json]
{
  "name": "second.example.com",
  "json_class": "Chef::Node",
  "chef_type": "node",
  "chef_environment": "development",
  "normal": {
    "fqdn": "10.33.33.35"
  },
  "default": {},
  "override": {},
  "run_list": [
    "role[chef-client]",
    "role[nginx]",
    "recipe[my_cool_app]"
  ]
}
\end{lstlisting}

If you using Chef Server, don't forget upload this cookbook and update node on Chef Server by knife.

\begin{lstlisting}[language=Bash,label=lst:cookbook-recipes3]
$ knife cookbook upload my_cool_app
Uploading my_cool_app    [0.1.0]
Uploaded 1 cookbook.
$ knife node from file nodes/second.example.com.json
Updated Node second.example.com!
// on real environment you will execute "knife ssh 'name:second.example.com' 'sudo chef-client' -i ../keys/production.pem -x ubuntu"
$ vagrant provision chef_second_client
INFO: Chef Run complete in 26.935610739 seconds
INFO: Running report handler
\end{lstlisting}

Let's install also \href{http://en.wikipedia.org/wiki/Network\_Time\_Protocol}{ntp} package in the same recipe. Because we have in recipe Ruby syntax, we can little \href{http://ru.wikipedia.org/wiki/Dont\_repeat\_yourself}{DRY} our code:

\begin{lstlisting}[label=lst:cookbook-recipes4,title=my-server-cloud/site-cookbooks/my\_cool\_app/recipes/default.rb]
%w(git ntp).each do |pack|
  package pack
end
\end{lstlisting}

Again upload cookbook and run chef-client:

\begin{lstlisting}[language=Bash,label=lst:cookbook-recipes5]
$ knife cookbook upload my_cool_app
Uploading my_cool_app    [0.1.0]
Uploaded 1 cookbook.
// on real environment you will execute "knife ssh 'name:second.example.com' 'sudo chef-client' -i ../keys/production.pem -x ubuntu"
$ vagrant provision chef_second_client
INFO: Chef Run complete in 26.935610739 seconds
INFO: Running report handler
$ vagrant ssh chef_second_client
...
vagrant@precise64:~$ ps ax | grep ntp
 1115 ?        Ss     0:00 /usr/sbin/ntpd -p /var/run/ntpd.pid -g -u 103:108
13839 pts/2    S+     0:00 grep --color=auto ntp
vagrant@precise64:~$ git --version
git version 1.7.9.5
\end{lstlisting}

As you can see our simple cookbook is working.


\subsection{Assign Dependencies}

If a cookbook has a dependency on a recipe that is located in another cookbook, that dependency must be declared in the metadata.rb file for that cookbook using the depends keyword.

For example, if the following recipe is included in a cookbook named <<my\_app>>:

\begin{lstlisting}[language=Bash,label=lst:cookbook-recipes6]
include_recipe "apache2::mod_ssl"
\end{lstlisting}

Then the metadata.rb file for that cookbook would have:

\begin{lstlisting}[language=Bash,label=lst:cookbook-recipes7]
depends "apache2"
\end{lstlisting}

\subsection{Create Exceptions}

A recipe can write events to a log file and can cause exceptions using \inline{Chef::Log}. The levels include debug, info, warn, error, and fatal. For example, to just capture information:

\begin{lstlisting}[language=Bash,label=lst:cookbook-recipes7]
Chef::Log.info('some useful information')
\end{lstlisting}

Or to trigger a fatal exception:

\begin{lstlisting}[language=Bash,label=lst:cookbook-recipes8]
Chef::Log.fatal!('something bad')
\end{lstlisting}

\subsection{Include Recipes}

A recipe can include one (or more) recipes located in external cookbooks by using the \inline{include_recipe} method. When a recipe is included, the resources found in that recipe will be inserted (in the same exact order) at the point where the \inline{include_recipe} keyword is located. The syntax for including a recipe is like this:

\begin{lstlisting}[language=Bash,label=lst:cookbook-recipes9]
include_recipe "recipe"
\end{lstlisting}

For example:

\begin{lstlisting}[language=Bash,label=lst:cookbook-recipes10]
include_recipe "apache2::mod_ssl"
\end{lstlisting}

If the \inline{include_recipe} method is used more than once to include a recipe, only the first inclusion is processed and any subsequent inclusions are ignored.

\subsection{Reload Attributes}

Attributes sometimes depend on actions taken from within recipes, so it may be necessary to reload a given attribute from within a recipe. For example:

\begin{lstlisting}[language=Bash,label=lst:cookbook-recipes11]
ruby_block 'some_code' do
  block do
    node.from_file(run_context.resolve_attribute("COOKBOOK_NAME", "ATTR_FILE"))
  end
  action :nothing
end
\end{lstlisting}

\subsection{Accessor Methods}

Attribute accessor methods are automatically created and the method invocation can be used interchangeably with the keys. For example:

\begin{lstlisting}[language=Bash,label=lst:cookbook-recipes12]
default.apache.dir          = "/etc/apache2"
default.apache.listen_ports = [ "80","443" ]
ion :nothing
end
\end{lstlisting}

This is a matter of style and preference for how attributes are reloaded from recipes, and may be seen when retrieving the value of an attribute.