\section{Resources and Providers}

As you read from previous chapter, Chef inside have resources (in example we used <<package>> resource). A resource defines the actions that can be taken, such as when a package should be installed, whether a service should be enabled or restarted, which groups, users, or groups of users should be created, where to put a collection of files, what the name of a new directory should be, and so on. During a chef-client run, each resource is identified and then associated with a provider. The provider then does the work to complete the action defined by the resource. Each resource is processed in the same order as they appear in a recipe. The chef-client ensures that the same actions are taken the same way everywhere and that actions produce the same result every time. A resource is implemented within a recipe using Ruby.

Let's look at the most necessary resources.

\subsection{Bash}

The bash resource is used to execute scripts using the Bash interpreter and includes all of the actions and attributes that are available to the execute resource. Example:

\begin{lstlisting}[label=lst:cookbook-resources-bash]
bash "install_something" do
  user "root"
  cwd "/tmp"
  code <<-EOH
  wget http://www.example.com/tarball.tar.gz
  tar -zxf tarball.tar.gz
  cd tarball
  ./configure
  make
  make install
  EOH
end
\end{lstlisting}

\subsection{Cron}

The cron resource is used to manage cron entries for time-based job scheduling. Attributes for a schedule will default to * if not provided. The cron resource requires access to a crontab program, typically cron. Example:

\begin{lstlisting}[label=lst:cookbook-resources-cron1]
cron "name_of_cron_entry" do
  hour "8"
  weekday "6"
  mailto "admin@opscode.com"
  action :create
end
\end{lstlisting}

\begin{lstlisting}[label=lst:cookbook-resources-cron2]
cron "noop" do
  hour "5"
  minute "0"
  command "/bin/true"
end
\end{lstlisting}

\subsection{Git}

The git resource is used to manage source control resources that exist in a git repository. git version 1.6.5 (or higher) is required to use all of the functionality in the git resource. Example:

\begin{lstlisting}[label=lst:cookbook-resources-git1]
git "/opt/mysources/couch" do
  repository "git://git.apache.org/couchdb.git"
  reference "master"
  action :sync
end
\end{lstlisting}

\begin{lstlisting}[label=lst:cookbook-resources-git2]
git "#{Chef::Config[:file_cache_path]}/ruby-build" do
 repository "git://github.com/sstephenson/ruby-build.git"
 reference "master"
 action :sync
end

bash "install_ruby_build" do
 cwd "#{Chef::Config[:file_cache_path]}/ruby-build"
 user "rbenv"
 group "rbenv"
 code <<-EOH
   ./install.sh
   EOH
 environment 'PREFIX' => "/usr/local"
end
\end{lstlisting}

\subsection{Cookbook\_file}

The cookbook\_file resource is used to transfer files from a sub-directory of the files/ directory in a cookbook to a specified path that is located on the host running the chef-client or chef-solo. The file in a cookbook is selected according to file specificity, which allows different source files to be used based on the hostname, host platform (operating system, distro, or as appropriate), or platform version. Files that are located under COOKBOOK\_NAME/files/default can be used on any platform. Example:

\begin{lstlisting}[label=lst:cookbook-resources-cookbook-file1]
cookbook_file "/tmp/testfile" do
  source "testfile"
  mode 00644
end
\end{lstlisting}

\begin{lstlisting}[label=lst:cookbook-resources-cookbook-file2]
cookbook_file "/etc/yum.repos.d/custom.repo" do
  source "custom"
  mode 00644
  notifies :run, "execute[create-yum-cache]", :immediately
  notifies :create, "ruby_block[reload-internal-yum-cache]", :immediately
end
\end{lstlisting}

\subsection{Template}

The template resource is used to manage file contents with an embedded Ruby (erb) template. This resource includes actions and attributes from the file resource. Template files managed by the template resource follow the same file specificity rules as the remote\_file and file resources. Example:

\begin{lstlisting}[label=lst:cookbook-resources-cookbook-template1]
template "/tmp/config.conf" do
  source "config.conf.erb"
end
\end{lstlisting}

\begin{lstlisting}[label=lst:cookbook-resources-cookbook-template2]
template "/tmp/somefile" do
  mode 00644
  source "somefile.erb"
  not_if {File.exists?("/etc/passwd")}
end
\end{lstlisting}

\subsection{User}

The user resource is used to add users, update existing users, remove users, and to lock/unlock user passwords. Example:

\begin{lstlisting}[label=lst:cookbook-resources-cookbook-user1]
user "random" do
  supports :manage_home => true
  comment "Random User"
  uid 1234
  gid "users"
  home "/home/random"
  shell "/bin/bash"
  password "$1$JJsvHslV$szsCjVEroftprNn4JHtDi."
end
\end{lstlisting}

\begin{lstlisting}[label=lst:cookbook-resources-cookbook-user2]
user "systemguy" do
  comment "system guy"
  system true
  shell "/bin/false"
end
\end{lstlisting}

\subsection{Deploy}

The deploy resource is used to manage and control deployments. This is a popular resource, but is also complex, having the most attributes, multiple providers, the added complexity of callbacks, plus four attributes that support layout modifications from within a recipe.

The deploy resource is modeled after Capistrano, a utility and framework for executing commands in parallel on multiple remote machines via SSH. The deploy resource is designed to behave in a way that is similar to the deploy and deploy:migration tasks in Capistrano.

TODO
